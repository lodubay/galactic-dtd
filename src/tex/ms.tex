% Define document class
% \documentclass[modern,linenumbers]{aastex631}
\documentclass[twocolumn,twocolappendix,linenumbers]{aastex631}
%twocolappendix

\usepackage{showyourwork}
\usepackage{amsmath}
\usepackage{amssymb}
\usepackage{graphicx}
% \usepackage{layouts}
\usepackage{xcolor}
\usepackage{upgreek}
\let\tablenum\relax
\usepackage{siunitx}

% user-defined commands
\newcommand{\yes}{\textcolor{green}{\checkmark}}
\newcommand{\meh}{\textcolor{black}{$\sim$}}
\newcommand{\no}{\textcolor{red}{$\times$}}
\newcommand{\osuaffil}{Department of Astronomy, The Ohio State University, 140 W. 18th Ave, Columbus OH 43210, USA}

\shorttitle{Galactic SN Ia DTD}
\shortauthors{Dubay, Johnson, \& Johnson}

\begin{document}

% Title
\title{The Galactic Delay-Time Distribution of Type Ia Supernovae:\\
       A Chemical Evolution Perspective}

% Author list
\author[0000-0003-3781-0747]{Liam O. Dubay}
\affiliation{\osuaffil}
\author[0000-0001-7258-1834]{Jennifer A. Johnson}
\affiliation{\osuaffil}
\affiliation{Center for Cosmology and Astroparticle Physics (CCAPP), The Ohio State University, 191 W. Woodruff Ave., Columbus OH 43210, USA}
\author[0000-0002-6534-8783]{James W. Johnson}
\affiliation{\osuaffil}
\affiliation{Observatories of the Carnegie Institution for Science, 813 Santa Barbara St., Pasadena CA 91101, USA}
% \author[0000-0001-7775-7261]{David H. Weinberg}

% Abstract from AAS 241
\begin{abstract}
    Type Ia supernovae (SNe Ia) produce most of Fe-peak elements in the Universe and therefore are a crucial ingredient in galactic chemical evolution models. SNe Ia do not explode immediately after star formation and the delay-time distribution (DTD) has not been definitively determined by supernova surveys or theoretical models. Because the DTD also affects the relationship among age, [Fe/H], and [$\alpha$/Fe] in chemical evolution models, comparison with observations of stars in the Milky Way is an important consistency check for any proposed DTD. We implement several popular forms of the DTD in combination with multiple star formation histories for the Milky Way in one-zone and multi-zone chemical evolution models using the Versatile Integrator for Chemical Evolution (VICE). VICE includes a prescription for radial stellar migration in multi-zone models. We compare our predicted interstellar medium abundance tracks, stellar abundance distributions, and stellar age distributions to the 17th data release of the Apache Point Observatory Galactic Evolution Experiment (APOGEE). We find that the SN Ia DTD has the largest effect on the [$\alpha$/Fe] distribution: a DTD with more prompt SNe Ia produces a stellar abundance distribution that is skewed toward a lower [$\alpha$/Fe] ratio, whereas a DTD with more delayed SNe Ia produces a distribution that is skewed toward higher [$\alpha$/Fe]. While the DTD alone cannot explain the observed bimodality in the [$\alpha$/Fe] distribution, in combination with an appropriate star formation history it affects the goodness of fit between the simulated and observed high-$\alpha$ sequence.
\end{abstract}

\section{Introduction}

[Hook and intro paragraph]
Galactic chemical evolution (GCE) studies seek to reproduce the observed distribution of metals throughout the Milky Way Galaxy.

[Introduction to SNe Ia]
Type Ia supernovae (SNe Ia), the explosions of carbon-oxygen white dwarfs (WDs), are responsible for roughly half of the iron produced in the Galaxy. Despite their importance, the mechanism(s) for producing SNe Ia are not fully understood. Two general production channels have been proposed: in the single-degenerate (SD) case, the WD accretes mass from a close non-degenerate companion until it surpasses a mass of $\sim1.4$ M$_\odot$ and explodes. In the double-degenerate (DD) case, two WDs merge after a gravitational-wave inspiral or head-on collision. Recent observations have placed tight constraints on the SD channel, heavily disfavoring it as the main pathway for producing ``normal'' SNe Ia. Though it is now the preferred model, the DD channel faces issues of its own with matching observed SN Ia rates as well as the fact that not all WD mergers may lead to a thermonuclear explosion.

[What is the delay time distribution]
The delay-time distribution (DTD) is the rate of SNe Ia which explode as a function of time after a hypothetical single star formation event. By convolving the DTD with the star formation rate (SFR) as a function of time, one can produce a function describing the time-variation in the SN Ia rate.

[Analytical DTDs]
Uncertainties in the progenitor channels of SNe Ia translate into uncertainties in the form of the DTD. 
\citet{Greggio2005-AnalyticalRates} used analytical approximations of binary population synthesis to produce theoretical DTDs for the SD and DD channels.

[Observational DTDs]
\citet{Mannucci2005-SNRate} found that the SN Ia rate per unit mass has a strong dependence on host galaxy morphology and color. Using these results, \citet{Mannucci2006-TwoPopulations} obtain a best-fit model for the DTD which consists of both a prompt component ($<100$ Myr) and a long exponential tail. These studies all assume a particular form for the DTD and fit for the specific parameters. They do not quantitatively favor one form over another (e.g., power-law over exponential).

[Why radial migration is important for SNe Ia]
Core-collapse SNe explode on timescales of a few Myr, before the progenitor has a chance to migrate far from its birth location. On the other hand, SNe Ia have a broad distribution of delay times and their progenitors can migrate great distances between birth and explosion. The consequence is that SNe Ia can enrich regions of the Galaxy far removed from regions which have experienced a high rate of star formation.

[Summary of DTDs used in simulations and theoretical studies]. The Feedback In Realistic Environments \citep[FIRE;][]{Hopkins2014-FIRE-1} and FIRE-2 \citep{Hopkins2018-FIRE-2} simulations implement the DTD from \citet{Mannucci2006-TwoPopulations} (explain previously). Following \citet{Schonrich2009-RadialMixing}, \citet{Weinberg2017-ChemicalEquilibrium} assume an exponential DTD for the purposes of producing an analytical solution to the equations of chemical equilibrium, but they show that it can be extended to a double-exponential DTD which closely approximates a power-law. 

Studies of galactic chemical evolution typically assume a single form for the SN Ia DTD. This is the first comprehensive look at the DTD in a multi-zone GCE model which includes the effects of radial migration.

A few chemical evolution studies have compared DTDs with different functional forms. \citet{Palicio2023-AnalyticDTD} derive analytic solutions to one-zone chemical evolution equations for seven DTDs (three theoretical and four empirical), approximating the more complicated DTD functions as a linear combination of exponential, Gaussian, and power-law components. They predict observables such as abundance tracks and MDFs and compare to APOGEE, but comparisons are limited to the Solar neighborhood. \citet{Poulhazan2018-PrecisionPollution} investigate six different DTDs in a smoothed-particle hydrodynamics simulation, but they only present global results for the entire galaxy, they don't compare abundances throughout the disk, and they don't compare to observations.

One curious chemical feature of the Milky Way disk is the existence of two populations separated by their [$\alpha$/Fe] abundance ratios. First noted by \citet{Furhmann1998-NearbyStars}, expanded by \citet{Hayden2015-ChemicalCartography}, definitely others too. The ``high-$\alpha$'' population is associated with the thick disk and contains stars with low enrichment from SNe Ia, so their abundance ratios reflect that of pure CCSN enrichment. The ``low-$\alpha$'' population is associated with the thin disk and contains stars with roughly solar $\alpha$-abundances.
The origin of the [$\alpha$/Fe] bimodality is debated. \citet{Schonrich2009-RadialMixing} argue that it arises due to radial migration (churning) even with a smooth star formation rate. \citet{Johnson2021-Migration} use a smooth SFR and a prescription for radial migration but are unable to reproduce the bimodality. \citet{Spitoni2021-TwoInfall} (definitely earlier papers too) argue that two distinct infall episodes can explain the bimodality. What separates the two sequences is SN Ia enrichment, so the DTD ought to have some impact on the [$\alpha$/Fe] bimodality, perhaps even determining whether or not it exists.

In this paper we comprehensively evaluate a selection of DTD models from the literature with multiple SFHs and a prescription for radial migration in the Versatile Integrator for Chemical Evolution (VICE). In Section \ref{sec:methods}, we present our models for the DTD and SFH, and we describe our chemical evolution simulations. In Section \ref{sec:onezone-results}, we present the results of our one-zone chemical evolution simulations. In Section \ref{sec:multizone-results}, we present the results of our multi-zone simulations and compare to observations. In Section \ref{sec:discussion}, we discuss the implications for the DTD and future surveys. In Section \ref{sec:conclusions}, we conclude.

\section{Methods}
\label{sec:methods}

\begin{deluxetable*}{Cccl}
    \tablecaption{A summary of parameters and their fiducial values for our chemical evolution models. We omit parameters which have been left unchanged from \citet{Johnson2021-Migration}; see their Table 1 for details.\label{tab:multizone-parameters}}
    \tablehead{
        \colhead{Quantity} & \colhead{Fiducial Value(s)} & \colhead{Section} & \colhead{Description}
    }
    \startdata
        % Multi-zone simulation parameters
        R_{\rm gal}     & [0, 20] kpc   & & Galactocentric radius \\
        \delta R_{\rm gal}  & 100 pc    & & Width of each concentric ring \\
        \Delta R_{\rm gal}  & N/A       & \ref{app:migration} & Change in orbital radius due to stellar migration \\
        p(\Delta R_{\rm gal}|\tau,R_{\rm form}) & Equation \ref{eq:radial-migration}    & \ref{app:migration} & Probability density function of radial migration \\
        z                   & [-3, 3] kpc                & \ref{app:migration} & Distance from Galactic midplane at present day \\
        p(z|\tau,R_{\rm final}) & Equation \ref{eq:sech-pdf}            & \ref{app:migration} & Probability density function of Galactic midplane distance\\
        \Delta t        & 10 Myr    & & Time-step size \\
        t_{\rm max}     & 13.2 Gyr  & & Maximum simulation time \\
        n               & 8         & & Number of stellar populations formed per ring per time-step \\
        R_{\rm Ia}(t)   & Equations \ref{eq:powerlaw-dtd}--\ref{eq:triple-dtd}  & \ref{sec:dtd-models}  & Delay time distribution of Type Ia supernovae \\
        t_D             & 40 Myr    & \ref{sec:dtd-models}  & Minimum SN Ia delay time \\
        M_{\rm gas,0}   & 0         & & Initial gas mass \\
        % Nucleosynthetic yields
        y_{\rm O}^{\rm CC}  & 0.015     & & CCSN yield of O    \\
        y_{\rm Fe}^{\rm CC} & 0.0012    & & CCSN yield of Fe   \\
        y_{\rm O}^{\rm Ia}  & 0         & & SN Ia yield of O       \\
        y_{\rm Fe}^{\rm Ia} & 0.00214   & & SN Ia yield of Fe \\
        f_{\rm IO}(t|R_{\rm gal})   & Equation \ref{eq:insideout-sfh}   & & Time-dependence of the SFR in the inside-out model \\
        f_{\rm LB}(t|R_{\rm gal})   & Equation \ref{eq:lateburst-sfh}   & & Time-dependence of the SFR in the late-burst model \\
        f_{\rm EB}(t|R_{\rm gal})   & Equation \ref{eq:earlyburst-ifr}  & & Time-dependence of the IFR in the early-burst model \\
        \uptau_{*,\rm EB}(t|R_{\rm gal})  & Equation \ref{eq:earlyburst-conroy22} & & Time-dependence of the SFE timescale in the early-burst model \\
        f_{\rm TI}(t|R_{\rm gal})   & Equation \ref{eq:twoinfall-ifr}   & & Time-dependence of the IFR in the two-infall model \\
    \enddata
\end{deluxetable*}

\subsection{DTD Models}
\label{sec:dtd-models}

\begin{subequations}
\begin{align}
    \dot M_{\rm Fe}^{\rm Ia} &= y_{\rm Fe}^{\rm Ia} \langle \dot M_*\rangle_{\rm Ia} \\
    &= y_{\rm Fe}^{\rm Ia} \frac{\int_0^t \dot M_*(t')R_{\rm Ia}(t-t')dt'}{\int_{t_D}^{t_{\rm max}} R_{\rm Ia}(t')dt'}
\end{align}
\end{subequations}

The yield $y_{\rm Fe}^{\rm Ia}$ can be expressed in terms of the integral of the DTD:

\begin{equation}
    y_{\rm Fe}^{\rm Ia} = m_{\rm Fe}^{\rm Ia} \int_{t_D}^{t_{\rm max}} R_{\rm Ia}(t')dt' = m_{\rm Fe}^{\rm Ia} \frac{N_{\rm Ia}}{M_*}
    \label{eq:dtd-integral}
\end{equation}
where $m_{\rm Fe}^{\rm Ia}$ is the average mass of Fe produced by a single SN Ia. We take the average number of SNe Ia per mass of stars formed to be $N_{\rm Ia}/M_*=2.2\times10^{-3}\,{\rm M}_\odot^{-1}$ following \citet{MaozMannucci2012-SNeIaReview}, which gives $R_{\rm Ia}$ units of ${\rm M}_\odot^{-1}\,{\rm yr}^{-1}$.

We explore five different functional forms for the SN Ia DTD: a power-law, a broken power-law with an initial flat plateau, an exponential, a two-population model, and a model based on triple-system dynamics, with a one or two variants of each. We present the unnormalized functional forms for $R_{\rm Ia}(t)$ below; VICE automatically normalizes and scales the DTD to produce the correct number of SNe Ia over the lifetime of the Galaxy. Figure \ref{fig:dtds} presents these DTDs as a function of time after star formation, and Table \ref{tab:dtds} summarizes the DTD models and their parameters.

\begin{figure}
    \centering
    \includegraphics{figures/delay_time_distributions.pdf}
    \caption{Selection of models for the SN Ia delay time distribution (DTD) used in this paper. All functions are normalized such that $R_{\rm Ia}(t=1\,{\rm Gyr})=1$. The black squares represent the DTD recovered for the SDSS-II sample of SNe Ia by \citet{Maoz2012-SloanIIDTD} at the same scale as the DTD models. The horizontal and vertical error bars indicate the time range and 1$\sigma$ uncertainties, respectively, of each DTD measurement.}
    \label{fig:dtds}
    \script{delay_time_distributions.py}
\end{figure}

\begin{deluxetable*}{lll}
\tablecaption{Summary of SN Ia DTD models explored in this paper.\label{tab:dtds}}
\tablehead{
\colhead{Model} & \colhead{Parameters} & \colhead{Similar to}
}
\startdata
Power-law   & $\alpha=-1.1$                 & \citet[][field]{Maoz2017-CosmicDTD}; 
                                              \citet{Wiseman2021-DESRates}              \\
Power-law   & $\alpha=-1.4$                 & \citet[][cluster]{Maoz2017-CosmicDTD}; 
                                              \citet{Heringer2019-FieldGalaxyDTD}       \\
Plateau     & $W=0.3$ Gyr, $\alpha=-1.1$    & \citet[][CLOSE DD]{Greggio2005-AnalyticalRates} \\
Plateau     & $W=1.0$ Gyr, $\alpha=-1.1$    & \citet[][WIDE DD]{Greggio2005-AnalyticalRates} \\
Triple-system   & $f_{\rm init}=0.05f_{\rm peak}$, $t_{\rm rise}=0.5$ Gyr, & \citet{Rajamuthukumar2022-TripleEvolution} \\
            & $W=0.5$ Gyr, $\alpha=-1.1$ & \\
Exponential & $\tau=1.5$ Gyr    & \citet[][SD]{Greggio2005-AnalyticalRates};
                                  \citet{Schonrich2009-RadialMixing};       \\
            &                   & \citet{Weinberg2017-ChemicalEquilibrium}  \\
Exponential & $\tau=3.0$ Gyr    & --- \\
Two-population  & $t_{\rm max}=0.05$ Gyr, $\sigma=0.015$ Gyr, & \citet{Mannucci2006-TwoPopulations} \\
            & $\tau=3.0$ Gyr & \\
\enddata
\end{deluxetable*}

% \citet{Stolger2020-ExponentialDTD} perform maximum likelihood estimations based on the cosmic star formation history and the star formation histories of individual galaxies. In both cases their best fit functions were approximately exponential, though they started with a skew-normal function so could not have ended up with a power law even if they wanted to. They don't actually report the timescale of this exponential, but based on the plots I'd say it's about 1.5 - 2 (1.7?) Gyr. Their best fit parameters make no sense to me as they don't report their units.

\paragraph{Power-law} A simple power law with slope $\alpha$:
\begin{equation}
    R_{\rm Ia}(t) \propto (t/1\,\rm{Gyr})^\alpha
    \label{eq:powerlaw-dtd}
\end{equation}
% where $A_\alpha = (\alpha+1)(t_{\rm max}^{\alpha+1} - t_{\rm min}^{\alpha+1})^{-1}$ is the normalization coefficient.
A declining power-law with $\alpha\sim-1$ is a standard assumption in many observational studies of the DTD as well as GCE simulations. \citet{Maoz2017-CosmicDTD} obtain a DTD with $\alpha=-1.07\pm0.09$ based on volumetric rates and an assumed cosmic SFH for field galaxies in redshift range $0\leq z\leq 2.25$. \citet{Wiseman2021-DESRates} obtain a similar slope of $\alpha=-1.13\pm0.05$ for field galaxies in the redshift range $0.2<z<0.6$. For galaxy clusters, \citet{Maoz2017-CosmicDTD} find a steeper DTD slope of $\alpha=-1.39^{+0.32}_{-0.05}$. \citet{Heringer2019-FieldGalaxyDTD} use a SFH-independent method to constrain the DTD for field galaxies within $0.01<z<0.2$ and find a steeper slope of $\alpha=-1.34^{+0.19}_{-0.17}$.
In this paper, we investigate the cases $\alpha=-1.1$ and $\alpha=-1.4$.

\paragraph{Plateau} A ``plateau'' model consisting of a flat slope of width $W$ followed by a power law decline:
\begin{equation}
    R_{\rm Ia}(t) \propto
    \begin{cases}
        1, & t < W \\
        (t/W)^\alpha, & t \ge W
    \end{cases}
    \label{eq:plateau-dtd}
\end{equation}
% where $B_W=[(W-t_{\rm min}) + \frac{t_{\rm max}^{\alpha+1} - W^{\alpha+1}}{(\alpha+1)W^\alpha}]^{-1}$ is the normalization coefficient.
We investigate the cases $W=0.3$ Gyr and $W=1$ Gyr, taking $\alpha=-1.1$ for all plateau models. Figure \ref{fig:analytical-dtd} illustrates the similarity between these models and two different treatments of the analytical double-degenerate DTD from \citet{Greggio2005-AnalyticalRates}. For more discussion on this, see Appendix \ref{app:analytical-dtds}.

\paragraph{Triple-system} A DTD based on simulations of triple-system evolution by \citet{Rajamuthukumar2022-TripleEvolution}. We approximate their DTD as a special case of the plateau model (Equation \ref{eq:plateau-dtd}) where the initial rate is quite low until an instantaneous rise to the plateau value at time $t_{\rm rise}$:
\begin{equation}
    R_{\rm Ia}(t) \propto
    \begin{cases}
        \epsilon, & t < t_{\rm rise} \\
        1, & t_{\rm rise} \leq t < W \\
        (t / W) ^ \alpha, & t \geq W
    \end{cases}
    \label{eq:triple-dtd}
\end{equation}
We take $t_{\rm rise}=0.5$ Gyr, $W=0.5$ Gyr, $\alpha=-1.1$, and $\epsilon=0.05$ so the initial rate is 5\% of the peak rate.

\paragraph{Exponential} An exponentially declining DTD with timescale $\tau$:
\begin{equation}
    R_{\rm Ia}(t) \propto \frac{1}{\tau} e^{-t/\tau}
    \label{eq:exponential-dtd}
\end{equation}
We investigate the case where $\tau=1.5$ Gyr, which has been used in previous GCE studies \citep[e.g.,][]{Schonrich2009-RadialMixing,Weinberg2017-ChemicalEquilibrium} and is similar to the analytical single-degenerate DTD from \citet{Greggio2005-AnalyticalRates}. We also investigate a longer timescale of $\tau=3$ Gyr. \citet{Matteucci1986-SupernovaEnrichment}, \citet{Stolger2020-ExponentialDTD}

\paragraph{Two-population} A DTD in which 50\% of SNe Ia belong to a ``prompt'' Gaussian component at small $t$ and the other 50\% form an exponential tail at large $t$:
\begin{equation}
    R_{\rm Ia}(t) \propto 0.5\Big[e^{-\frac{(t-t_p)^2}{2\sigma^2}} + e^{-t/\tau}\Big]
    \label{eq:prompt-dtd}
\end{equation}
Following the two-population model from \citet{Mannucci2006-TwoPopulations}, we take $t_p=50$ Myr, $\sigma=15$ Myr, and $\tau=3$ Gyr, which results in roughly 50\% of SNe Ia exploding within $t<100$ Myr. \citet{Poulhazan2018-PrecisionPollution} refer to this as the ``bimodal'' model.

% Minimum SN Ia delay time
The minimum SN Ia delay time $t_D$, i.e. the lifetime of the most massive progenitor system, is also a parameter of the DTD which can have an effect on chemical evolution. \citet{Andrews2017-ChemicalEvolution} found that a longer minimum delay time produces a sharper high-$\alpha$ ``knee'' and shifts it to higher [Fe/H], although they only explored this for an exponential DTD. They adopt a fiducial value of $t_D=150$ Myr, as do \citet{Johnson2021-Migration}. \citet{Poulhazan2018-PrecisionPollution} adopt a much smaller value of $t_D=29$ Myr as the lifetime of their most massive progenitors.

We take $t_D=40$ Myr to be our fiducial value as it is the approximate lifetime of an $8\,{\rm M}_\odot$ star [Larson 1974]. We find that adopting a longer $t_D$ has only a minor effect on the chemical evolution for most DTD models except the power-law, but in that case the effect of a longer minimum delay time can be approximated by adding an initial plateau of $\sim100$ Myr to the DTD. See Section \ref{sec:onezone-results} and Figure \ref{fig:onezone-twopanel} for details.

\subsection{Star Formation Histories}
\label{sec:sfh}

In this paper, we consider four models for the SFH: inside-out, late-burst, early-burst, and two-infall. The first two models are run in VICE's ``star formation mode,'' where the surface density of star formation rate $\dot\Sigma_*$ is prescribed along with the star formation efficiency timescale $\tau_*$, and the infall rate surface density $\dot\Sigma_{\rm in}$ and gas surface density $\Sigma_{\rm gas}$ are calculated from the specified quantities. The latter two models are run in ``infall mode,'' where we specify $\dot\Sigma_{\rm in}$  and $\tau_*$ and the other two quantities follow. 

The functions presented below are dimensionless. For a detailed look at the normalization of the star formation rate with $R_{\rm gal}$, we refer the reader to Appendix B of \citet{Johnson2021-Migration}.

\paragraph{Inside-out} Following \citet{Johnson2021-Migration}, this is our fiducial SFH:
\begin{equation}
    f_{\rm IO}(t|R_{\rm gal}) = (1 - e^{-t/\tau_{\rm rise}}) e^{-t/\tau_{\rm sfh}}
    \label{eq:insideout-sfh}
\end{equation}
where we assume $\tau_{\rm rise}=2$ Gyr for all radii. The SFH timescale $\tau_{\rm sfh}$ varies with $R_{\rm gal}$, with $\tau_{\rm sfh}(R_{\rm gal}=8\,\rm{kpc})\approx15$ Gyr at the Solar annulus and longer timescales in the outer galaxy. The $\tau_{\rm sfh} - R_{\rm gal}$ relation is determined from the data of \citet{Sanchez2020-StarFormationTimescales}; see Section 2.5 of \citet{Johnson2021-Migration} for the details.

\paragraph{Late-burst} A variation on the inside-out SFH with a Gaussian burst in the star formation rate:
\begin{equation}
    f_{\rm LB}(t|R_{\rm gal}) = f_{\rm IO}(t|R_{\rm gal}) \Big(1 + A_b e^{-(t-t_b)^2/2\sigma_b^2} \Big)
    \label{eq:lateburst-sfh}
\end{equation}
where $A_b$ is the dimensionless amplitude of the starburst, $t_b$ is the time of the peak of the burst, and $\sigma_b$ is the width of the Gaussian. Following \citet{Johnson2021-Migration}, we adopt $A_b=1.5$, $t_b=11.2$ Gyr, and $\sigma_b=1$ Gyr. The determination of $\tau_{\rm sfh}$ and the normalization of the SFR as a function of $R_{\rm gal}$ are the same as in the inside-out case.

\paragraph{Early-burst} The infall rate at a given radius declines exponentially with time:
\begin{equation}
    f_{\rm EB}(t|R_{\rm gal}) = e^{-t/\tau_{\rm sfh}}
    \label{eq:earlyburst-ifr}
\end{equation}
The time-dependence of the SFE timescale $\tau_*$ comes from \citet{Conroy2022-ThickDisk}:
\begin{equation}
    \frac{\tau_*(t|\Sigma_g)}{1\,\rm{Gyr}} =
    \begin{cases}
        50, & t < 2.5\,\rm{Gyr} \\
        \frac{50}{[1+3(t-2.5)]^2}, & 2.5\leq t \leq 3.7\,\rm{Gyr} \\
        2.36, & t > 3.7\,\rm{Gyr}
    \end{cases}
    \label{eq:earlyburst-conroy22}
\end{equation}
We combine this with the Kennicutt-Schmidt relation used by \citet{Johnson2021-Migration}, minus the dependence on the molecular timescale $\tau_{\rm mol}$:
\begin{equation}
    \dot \Sigma_* = 
    \begin{cases}
        \Sigma_g, & \Sigma_g \geq \Sigma_{g,2} \\
        \Sigma_g \Big(\frac{\Sigma_g}{\Sigma_{g,2}}\Big)^{2.6}, & \Sigma_{g,1} \leq \Sigma_g \leq \Sigma_{g,2} \\
        \Sigma_g \Big(\frac{\Sigma_{g,1}}{\Sigma_{g,2}}\Big)^{2.6} \Big(\frac{\Sigma_g}{\Sigma_{g,1}}\Big)^{0.7}, & \Sigma_g \leq \Sigma_{g,1}
    \end{cases}
    \label{eq:kennicutt-schmidt}
\end{equation}
Which leads to an overall SFE timescale of
\begin{equation}
    \tau_*(t,\Sigma_g) = \tau_*(t|\Sigma_g) \frac{\Sigma_g}{\dot \Sigma_*}
    \label{eq:earlyburst-taustar}
\end{equation}

\paragraph{Two-infall} 
We parameterize the infall rate as two successive, exponentially declining bursts as in \citet{Chiappini2001-AbundanceGradients}:
\begin{equation}
    \label{eq:twoinfall-ifr}
    f_{\rm TI}(t|R_{\rm gal}) = N_1(R_{\rm gal}) e^{-t/\tau_1} + N_2(R_{\rm gal}) e^{-(t-t_{\rm on})/\tau_2}
\end{equation}
In this model, the first infall produces the thick disk and the second infall produces the thin disk. The normalization ratio $N_2/N_1$ is calculated so that the thick-to-thin-disk surface density ratio $f_\Sigma(R)=\Sigma_2(R)/\Sigma_1(R)$ is given by
\begin{equation}
    f_\Sigma(R) = f_\Sigma(0) e^{R(1/R_2 - 1/R_1)}
\end{equation}
where we take the thick disk scale radius $R_1=2.0$ kpc, thin disk scale radius $R_2=2.5$ kpc, and $f_\Sigma(0)=0.27$ following \citet{BlandHawthornGerhard2016-MilkyWayReview}.

Figure \ref{fig:sfhs} presents an overview of these model star formation histories.

\begin{figure*}
    \centering
    \includegraphics[width=\linewidth]{figures/star_formation_histories.pdf}
    \caption{The surface densities of star formation $\dot \Sigma_*$ (first row), gas infall $\dot \Sigma_{\rm in}$ (second row), and gas mass $\Sigma_{\rm gas}$ (third row), and the star formation efficiency timescale $\tau_*$ (fourth row) as functions of simulation time for our four model SFHs: inside-out (first column; see Equation \ref{eq:insideout-sfh}), late-burst (second column; see Equation \ref{eq:lateburst-sfh}), early-burst (third column; see Equations \ref{eq:earlyburst-ifr} and \ref{eq:earlyburst-taustar}), and two-infall (fourth column; see Equation \ref{eq:twoinfall-ifr}). In each panel, we plot curves for the simulation zones which have inner radii at 4 kpc (yellow), 6 kpc (orange), 8 kpc (red), 10 kpc (violet), 12 kpc (indigo), and 14 kpc (blue).}
    \label{fig:sfhs}
    \script{star_formation_histories.py}
\end{figure*}

We adopt the same nucleosynthetic yields as \citet{Johnson2021-Migration}; the values are presented in Table \ref{tab:multizone-parameters}.

% \subsection{Nucleosynthetic Yields}
% \label{sec:yields}
% 
% Table \ref{tab:yields} summarizes the nucleosynthetic yields assumed in this paper, which we adopt from \citet{Johnson2021-Migration}.
% 
% \begin{deluxetable}{lll}
%     \tablecaption{Nucleosynthetic yields assumed in our simulations.\label{tab:yields}}
%     \tablehead{
%         \colhead{Source} & \colhead{Element} & \colhead{Yield}
%     }
%     \startdata
%     CCSN    & O     & 0.015     \\
%     CCSN    & Fe    & 0.0012    \\
%     \hline
%     SN Ia   & O     & 0         \\
%     SN Ia   & Fe    & 0.00214
%     \enddata
% \end{deluxetable}

\subsection{Observational Sample}
\label{sec:observational-sample}

We compare our simulation results to abundance measurements from the 17th data release \citep[DR17;][]{Abdurro'uf2022-SDSSIV-DR17} of the Apache Point Observatory Galactic Evolution Experiment \citep[APOGEE;][]{Majewski2017-APOGEE}. APOGEE used infrared spectrographs \citep{Wilson2019-APOGEE-Spectrographs} mounted on two telescopes: the 2.5-meter Sloan Foundation Telescope \citep{Gunn2006-SloanTelescope} at Apache Point Observatory, United States in the Northern Hemisphere, and the Ir{\'e}n{\'e}e DuPont Telescope \citep{BowenVaughan1973-DuPontTelescope} at Las Campanas Observatory, Chile in the Southern Hemisphere. After the spectra were passed through the data reduction pipeline \citep{Nidever2015-APOGEE-DataReduction}, the APOGEE Stellar Parameter and Chemical Abundance Pipeline \citep[ASPCAP;][]{Holtzmann2015-ASPCAP,GarciaPerez2016-ASPCAP} extracted chemical abundances using the model grids and interpolation method described by \citet{Jonsson2020-APOGEE-DR16}.

We restrict our sample to red giant branch and red clump stars with high-quality spectra. Table \ref{tab:sample-selection} lists our selection criteria, which largely follow from \citet{Hayden2015-ChemicalCartography}. This produces a final sample of \variable{output/sample_size.txt}stars. APOGEE stars are matched with their Bailer-Jones photo-geometric distance estimate from \textit{Gaia} Early Data Release 3 \citep{Gaia2016-Mission,Gaia2021-EDR3}, which we use to calculate galactocentric radius $R_{\rm gal}$ and midplane distance $z$.
We adopt estimated ages from \citet{Leung2023-Ages}, who use a variational encoder-decoder network to estimate the age of APOGEE giants without contamination from age-abundance correlations. Their sample does not experience the $\sim10$ Gyr plateau that affects ages from astroNN \citep{Mackereth2019-astroNN-Ages}. We use an age uncertainty cut of 40\% per the recommendations of \citet{Leung2023-Ages}, which produces a total sample of \variable{output/age_sample_size.txt}APOGEE stars with age estimates. The median uncertainty in log-age space is $\sim 0.1$.

\begin{deluxetable*}{lll}
    \tablecaption{Sample selection parameters from APOGEE DR17\label{tab:sample-selection}}
    \tablehead{
        \colhead{Parameter} & \colhead{Range} & \colhead{Notes}
    }
    \startdata
        $\log g$            & $1.0 < \log g < 3.8$          & Select giants only \\
        $T_{\rm eff}$       & $3500 < T_{\rm eff} < 5500$ K & Reliable temperature range \\
        $S/N$               & $S/N > 80$                    & Required for accurate stellar parameters \\
        ASPCAPFLAG Bits     & $\notin$ 23                   & Remove stars flagged as bad \\
        EXTRATARG Bits      & $\notin$ 0, 1, 2, 3, or 4     & Select main red star sample only \\
        Age                 & $\sigma_{\rm Age} < 40\%$     & Age uncertainty from \citet{Leung2023-Ages}
    \enddata
\end{deluxetable*}


\section{Results from One-Zone Models}
\label{sec:onezone-results}

All models are run for a duration of $t_{\rm max}=13.2$ Gyr with a time-step of $\Delta t=0.01$ Gyr. We adopt an outflow mass-loading factor $\eta\equiv \dot M_{\rm out}/\dot M_*=2.15$ \citep[see Equation 8 from][]{Johnson2021-Migration}, a star formation efficiency timescale $\tau_*\equiv M_{\rm gas}/\dot M_*=2$ Gyr, and an initial gas mass $M_{\rm gas,0}=0$. We adopt a continuous recycling prescription \citep[see Equation 2 from][]{JohnsonWeinberg2020-Starbursts}. Unless otherwise specified, we adopt the inside-out star formation rate (Equation \ref{eq:insideout-sfh}) evaluated at $R_{\rm gal}=8$ kpc (i.e., $\tau_{\rm rise}=2$ Gyr and $\tau_{\rm sfh}=15.1$ Gyr), and a minimum SN Ia delay time of 40 Myr. A summary of these one-zone fiducial parameters is presented in Table \ref{tab:onezone-parameters}.

\begin{deluxetable}{Ccl}
    \tablecaption{A summary of parameters and their fiducial values for our one-zone chemical evolution models.\label{tab:onezone-parameters}}
    \tablehead{
        \colhead{Quantity} & \colhead{Value} & \colhead{Description}
    }
    \startdata
    \Delta t        & 10 Myr    & Time-step size \\
    t_{\rm max}     & 13.2 Gyr  & Maximum simulation time \\
    \eta            & 2.15      & Outflow mass-loading factor \\
    \tau_*          & 2.0 Gyr   & Star formation efficiency timescale \\
    \dot M_r        & continuous    & Recycling rate \\
    t_D             & 40 Myr    & Minimum SN Ia delay time \\
    M_{\rm gas,0}   & 0         & Initial gas mass \\
    \dot M_*(t)     & Equation \ref{eq:insideout-sfh}  & Star formation rate \\
    \tau_{\rm rise} & 2 Gyr     & SFR rise timescale \\
    \tau_{\rm sfh}  & 15.1 Gyr  & SFR exponential timescale
    \enddata
\end{deluxetable}

The left-hand panel of Figure \ref{fig:onezone-threepanel} compares the results of three one-zone models which are identical except for the slope of the power-law DTD. A steeper slope produces a sharper ``knee'' and a faster decline in [O/Fe], resulting in a narrower distribution of [O/Fe] around the low-$\alpha$ sequence and a dearth of high-$\alpha$ stars. Conversely, a shallower slope produces a gentler knee and slower decline in [O/Fe], resulting in a broader [O/Fe] distribution and a greater number of high-$\alpha$ stars. In all cases the [O/Fe] distribution is distinctly unimodal. The distribution of [Fe/H] is not as strongly affected by the power-law slope: a shallower slope results in only a modest increase in the width of the distribution.

Similar trends can be seen when adjusting the timescale of the exponential DTD, as shown in the middle panel of Figure \ref{fig:onezone-threepanel}. Here, the knee is not a sharp feature associated with the onset of SNe Ia as in the power-law case, but rather a gentle curve in the abundance track around 1 Gyr. A short exponential timescale shifts this knee down to lower [O/Fe] values, while a longer timescale raises it. The effect on the [O/Fe] distribution is similar to the power-law slope: a short timescale produces more low-$\alpha$ stars, while a long timescale produces more high-$\alpha$ stars (but still a unimodal distribution). The effect on the [Fe/H] distribution is somewhat more pronounced, with longer timescales skewing to lower [Fe/H] values. A doubling of the timescale from 1.5 Gyr to 3 Gyr raises the [O/Fe] abundance ratio at $t=1$ Gyr by $\sim0.05$ dex and at $t=3$ Gyr by $\sim0.1$ dex, but the equilibrium abundance reached at $t=10$ Gyr remains unchanged.

\begin{figure*}
    \centering
    \includegraphics[width=0.32\linewidth]{figures/onezone_powerlaw_slope.pdf}
    \includegraphics[width=0.32\linewidth]{figures/onezone_exponential_timescale.pdf}
    \includegraphics[width=0.32\linewidth]{figures/onezone_plateau_width.pdf}
    \caption{\textit{Left:} Abundance tracks in the [Fe/H]--[O/Fe] plane for one-zone chemical evolution models models which assume a power-law DTD. The dotted, dashed, and solid curves represent models with power-law slopes $\alpha=-0.8$, $-1.1$, and $-1.4$, respectively. For reference, the solid gray curve represents an exponential DTD with $\tau=3$ Gyr. In all cases a minimum SN Ia delay time of 40 Myr is assumed. The open symbols located along each curve mark logarithmic steps in simulation time. The marginal panels above and to the right of the central panel present the distributions of [Fe/H] and [O/Fe], respectively. For display purposes, these distributions are convolved with a Gaussian with a standard deviation of 0.02 dex.
    \textit{Middle:} Similar, comparing exponential DTDs with varying timescales. The solid, dashed, and dotted curves represent timescales $\tau=6$, 3, and 1.5 Gyr, respectively.
    \textit{Right:} Similar, comparing plateau DTDs with varying plateau widths. The solid, dashed, and dotted green curves represent widths $W=1$, 0.3, and 0.1 Gyr, respectively. All plateau DTD models assume a post-plateau power-law slope of $\alpha=-1.1$. For reference, the solid gray curve represents an exponential DTD with $\tau=3$ Gyr, and the dotted purple curve represents a power-law DTD $\alpha=-1.1$ and no plateau.}
    \label{fig:onezone-threepanel}
    \script{onezone_threepanel.py}
\end{figure*}

The left-hand panel of Figure \ref{fig:onezone-twopanel} shows that the minimum SN Ia delay time has a much stronger effect on the abundance track and MDFs of models which assume a power-law DTD than models which assume a plateau or exponential DTD, due to the much higher SN Ia rate in the power-law models at early times (see Figure \ref{fig:dtds}). Moreover, a power-law DTD with a long minimum delay may be observationally hard to distinguish from a plateau model. In Figure \ref{fig:onezone-twopanel}, the abundance track for the model with a power-law DTD and $t_D=150$ Myr (red dashed line) is similar to that of the plateau DTD with $W=0.3$ Gyr and $t_D=40$ Myr (blue solid line), and the [O/Fe] distributions are virtually identical. 

A 150 Myr minimum delay time is incompatible with the two-population DTD model, which has $\sim 50$\% of SNe Ia explode in the first 100 Myr, and would have a negligible effect on the triple-system DTD as its relative SN Ia rate is so low at early times. 

The right-hand panel of Figure \ref{fig:onezone-twopanel} compares the one-zone model outputs from the full range of DTDs we investigate in this paper. In general, the choice of DTD has the greatest effect on the location of the knee in [Fe/H]-[O/Fe] space (center panel) and the high-$\alpha$ end of the distribution of [O/Fe] (right panel). The DTD also has a smaller effect on the MDF in the range $-1.5\lesssim$ [Fe/H] $\lesssim-0.5$ (top panel).

\begin{figure*}
    \centering
    \includegraphics[width=0.49\linewidth]{figures/onezone_minimum_delay.pdf}
    \includegraphics[width=0.49\linewidth]{figures/onezone_dtd.pdf}
    \caption{\textit{Left:} Comparison of one-zone models with minimum SN Ia delay times of 40 Myr (solid curves) and 150 Myr (dashed curves) for two DTDs: a power-law with $\alpha=-1.1$ (blue curves), and an exponential with $\tau=1.5$ Gyr (purple curves). The layout is similar to the plots in Figure \ref{fig:onezone-threepanel}.
    \textit{Right:} Comparison of one-zone models which assume all five functional forms for the DTD in this paper: triple-system (green), plateau with $W=1$ Gyr (cyan), exponential with $\tau=1.5$ Gyr (purple), power-law with $\alpha=-1.1$ (blue), and two-population (pink). A minimum SN Ia delay time of 40 Myr is assumed for all models.}
    \label{fig:onezone-twopanel}
    \script{onezone_twopanel.py}
\end{figure*}

% \subsection{The Star Formation History}

% Figure \ref{fig:onezone-delay-taustar} shows results from one-zone models with a power-law DTD which differ in their assumed minimum SN Ia delay time ($t_D$) or star formation efficiency timescale ($\tau_*$). In [Fe/H]-[O/Fe] space, doubling or halving the minimum delay time produces similar results to doubling or halving the SFE timescale. A shorter minimum delay time means SNe Ia will start to explode sooner after the star formation burst, giving less time for CCSNe to explode at the plateau of the abundance track. Similarly, a longer SFE timescale means fewer massive stars will be produced by the initial star formation burst, resulting in fewer CCSNe before the first SNe Ia explode. 

% The minimum delay and SFE timescale have similar effects on the abundance tracks but very different effects on the MDFs. Altering the SFE timescale changes the stellar distribution of [Fe/H], especially at the low-metallicity end, as shown in the top panel of Figure \ref{fig:onezone-delay-taustar}. However, as a change in the SFE timescale affects both the number of high-mass and low-mass stars produced, it does not affect the distribution of [O/Fe], as shown in the right-hand panel of Figure \ref{fig:onezone-delay-taustar}. Meanwhile, a change in the minimum SN Ia delay time does not impact the total number of SNe which explode, so the distribution of [Fe/H] is largely unaffected, while it does result in a change in the number of high-$\alpha$ stars formed and therefore the distribution of [O/Fe].

% \begin{figure}
%     \centering
%     \includegraphics{figures/onezone_delay_taustar.pdf}
%     \caption{Similar to Figure \ref{fig:onezone-powerlaw-slope} but with varying minimum SN Ia delay time ($t_D$) and star formation efficiency timescale ($\tau_*$). Tracks with $\tau_*=2$ Gyr are represented in black, while a track with a shorter $\tau_*=1$ Gyr is shown in red and a track with a longer $\tau_*=4$ Gyr is in yellow. The dotted, solid, and dashed black curves represent a minimum delay time of 40 Myr, 80 Myr, and 160 Myr, respectively. A power-law DTD with $\alpha=-1.1$ is assumed in all models.}
%     \label{fig:onezone-delay-taustar}
%     \script{onezone_delay_taustar.py}
% \end{figure}

\section{Results from Multi-Zone Models}
\label{sec:multizone-results}

\begin{deluxetable}{lll}
    \tablecaption{Median Parameter Uncertainties\label{tab:uncertainties}}
    \tablehead{
        \colhead{Parameter} & \colhead{Median Uncertainty} & \colhead{Source}
    }
    \startdata
        [Fe/H] & $9.2\times10^{-3}$ & APOGEE DR17 \\
        $\rm [O/Fe]$ & $1.8\times10^{-2}$ & APOGEE DR17 \\
        log(Age/Gyr) & $0.10$ & \citet{Leung2023-Ages}
        % Age/Gyr & 29\% & astroNN
    \enddata
\end{deluxetable}

\subsection{The distribution of [Fe/H]}

Figure \ref{fig:feh-df-comparison} shows MDFs across the Galaxy for a selection of models and APOGEE data. The two left-hand columns show MDFs from our simulations which assume the same DTD (exponential with $\tau=1.5$ Gyr) but different model SFHs. For comparison, the APOGEE MDFs are binned similarly and presented in the center column of Figure \ref{fig:feh-df-comparison}. 

Holding the SFH fixed, varying the DTD has a very minor effect on the MDF across the disk. The two right-hand columns of Figure \ref{fig:feh-df-comparison} plot the MDFs for two multi-zone simulations which both assume an inside-out SFH but different DTDs: a power-law with slope $\alpha=-1.4$, and an exponential with timescale $\tau=3$ Gyr. The balance between prompt and delayed SNe Ia is starkly different between the two models, with $\sim 80\%$ of SNe Ia exploding within 1 Gyr in the former model but only $\sim 30\%$ in the latter.

The choice of DTD and SFH has a larger effect on the inner regions of the galaxy than the outer. Inner regions have a higher SFR and more stars are formed at early times than late, so differences in the SN Ia rate have a more pronounced effect on the MDF. For all SFH models except two-infall, the outer galaxy has a longer SFR timescale than the inner galaxy, so the MDF in the outer galaxy is relatively unaffected by the fraction of prompt SNe Ia.

While there are some quantitative differences in how the shape of the MDF varies with Galactic region, the qualitative trends are unaffected by the choice of model SFH or DTD. These trends are primarily driven by chemical equilibrium, abundance gradients, and radial migration \citep[][see their Section 3.2 for further discussion]{Johnson2021-Migration}.

\begin{figure*}
    \centering
    \includegraphics[width=\linewidth]{figures/feh_df_comparison.pdf}
    \caption{Distributions of [Fe/H] from multi-zone simulations with various models for the SFH and DTD. Each row presents distributions of stars within a range of absolute midplane distance: $1\leq|z|<2$ kpc (\textit{top}), $0.5\leq|z|<1$ kpc (\textit{middle}), and $0\leq|z|<0.5$ kpc (\textit{bottom}). Within each panel, curves of different color represent the distributions of stars binned by Galactocentric radius $R_{\rm gal}$, from $3\leq R_{\rm gal}<5$ kpc (yellow) to $13\leq R_{\rm gal}<15$ kpc (blue). Each distribution is normalized so the area under the curve is 1, and the vertical scale is consistent across each row. All distributions are convolved with observational uncertainties in APOGEE DR17 (see Table \ref{tab:uncertainties}) and smoothed with a box-car width of 0.2 dex. 
    \textit{Left columns:} comparison between the inside-out and two-infall SFH models. In both cases an exponential DTD with timescale $\tau=1.5$ Gyr is assumed. 
    \textit{Center column:} the distributions from APOGEE DR17 for reference, binned and smoothed similarly.
    \textit{Right columns:} comparison between the power-law ($\alpha=-1.4$) and exponential ($\tau=3$ Gyr) DTD models. In both cases the inside-out SFH is assumed.}
    \label{fig:feh-df-comparison}
    \script{feh_df_comparison.py}
\end{figure*}

% \begin{figure}
%     \centering
%     \includegraphics{figures/feh_df_dtd.pdf}
%     \caption{Distributions of [Fe/H] from multi-zone simulations with different DTDs: a power-law with slope $\alpha=-1.4$ (\textit{left}), and an exponential with timescale $\tau=3$ Gyr (\textit{right}). Each row presents distributions of stars within a range of absolute midplane distance: $1\leq|z|<2$ kpc (\textit{top}), $0.5\leq|z|<1$ kpc (\textit{middle}), and $0\leq|z|<0.5$ kpc (\textit{bottom}). Within each panel, curves of different color represent the distributions of stars binned by Galactocentric radius $R_{\rm gal}$, from $3\leq R_{\rm gal}<5$ kpc (yellow) to $13\leq R_{\rm gal}<15$ kpc (blue). Each distribution is normalized so the area under the curve is 1, and the vertical scale is consistent across each row. All distributions are convolved with observational uncertainties in APOGEE DR17 (see Table \ref{tab:uncertainties}) and smoothed with a box-car width of 0.2 dex. In both cases an inside-out SFH is assumed.}
%     \label{fig:feh-df-dtd}
%     \script{feh_df_dtd.py}
% \end{figure}

% \begin{figure*}
%     \centering
%     \includegraphics[width=\linewidth]{figures/feh_df_sfh.pdf}
%     \caption{Distributions of [Fe/H] from multi-zone simulations with different SFHs. The plot format is similar to Figure \ref{fig:feh-df-dtd}. In all cases an exponential DTD with timescale $\tau=1.5$ Gyr is assumed. Distributions from APOGEE DR17, binned and smoothed similarly, are presented in the right-most column for reference.}
%     \label{fig:feh-df-sfh}
%     \script{feh_df_sfh.py}
% \end{figure*}

\subsection{The distribution of [O/Fe]}

The early-burst model produces a bimodal distribution of [O/Fe] even out to large radii and close to the midplane, whereas the APOGEE distribution does not show a prominent high-$\alpha$ peak beyond $R_{\rm{Gal}}\sim11$ kpc.

Away from the midplane, there is a clear trend of a higher average stellar [O/Fe] at small $R_{\rm gal}$ and lower [O/Fe] at large $R_{\rm gal}$. However, the APOGEE data show a clear dichotomy between inner-galaxy, high-$\alpha$ stars and outer-galaxy, low-$\alpha$ stars, with few in between. In the inside-out and late-burst SFHs, the shift from high-$\alpha$ to low-$\alpha$ is much more gradual and the trough at [O/Fe]$\approx0.2$ does not appear. The two-infall SFH appears to produce \textit{three} modes, at [O/Fe]$\approx 0.0$, $0.2$, and $0.4$, with the first and last showing the most variation with $R_{\rm gal}$ at high latitudes. The early-burst SFH produces the closest match to the data by far: the low-$\alpha$ sequence away from the midplane shows increasing contribution with larger radii, with the innermost radii containing very few low-$\alpha$ stars. However, the high-$\alpha$ sequence contains substantial contribution from stars at large radii, which is not seen in the data.

\begin{figure*}
    \centering
    \includegraphics[width=\linewidth]{figures/ofe_df_sfh.pdf}
    \caption{Distributions of [O/Fe] from multi-zone simulations with different SFHs. In all cases an exponential DTD with timescale $\tau=1.5$ Gyr is assumed. The format of each panel is the same as in Figure \ref{fig:feh-df-comparison}, except that all distributions are smoothed with a box-car width of 0.05 dex. Distributions from APOGEE DR17, binned and smoothed similarly, are presented in the right-most column for reference.}
    \label{fig:ofe-df-sfh}
    \script{ofe_df_sfh.py}
\end{figure*}

Figure \ref{fig:ofe-df-dtd} shows [O/Fe] distributions from simulations with the same SFH but a variety of different DTDs. We choose the early-burst SFH because it produces distinct high- and low-$\alpha$ sequences, which makes the effect of the DTD on the former more apparent. The most apparent effect of the DTD is to shift the mode of the high-$\alpha$ sequence. The two-population DTD, which has the most prompt SNe Ia of any model in this paper, places the high-$\alpha$ sequence at [O/Fe]$\approx 0.2$, while the triple-system DTD, which has the fewest prompt SNe Ia, places $\sim0.2$ dex higher at [O/Fe]$\approx0.4$.

\begin{figure*}
    \centering
    \includegraphics[width=\linewidth]{figures/ofe_df_dtd.pdf}
    \caption{Distributions of [O/Fe] from multi-zone simulations with different DTDs. In all cases an early-burst SFH is assumed. The plot format is similar to Figure \ref{fig:ofe-df-sfh}.}
    \label{fig:ofe-df-dtd}
    \script{ofe_df_dtd.py}
\end{figure*}

\subsection{Bimodality in [O/Fe]}
\label{sec:bimodality}

A striking feature of the [O/Fe] distribution in the Milky Way is the distinct separation into two components, the high- and low-$\alpha$ sequences. 

Reference \citet{Vincenzo2021-AlphaDistribution} for how the bimodality is made stronger by the selection function, but it's not just due to selection effects.

\begin{figure*}
    \centering
    \includegraphics[width=\linewidth]{figures/ofe_bimodality_summary.pdf}
    \caption{The distributions of [O/Fe] along two slices of [Fe/H]: $-0.6\leq$[Fe/H]$<-0.4$ (blue dashed) and $-0.4\leq$[Fe/H]$<-0.2$ (red solid). \textit{Top row:} results from five multi-zone simulations which assume the late-bust SFH but different DTD models. \textit{Bottom row}: the first four panels compare the four SFH models, all assuming an exponential DTD with $\tau=1.5$ Gyr. The bottom-right panel (highlighted) plots data from APOGEE DR17 for reference. All panels contain stars within the Galactic region defined by $7\leq R_{\rm gal}<9$ kpc and $0\leq|z|<2$ kpc. The distributions of stars in $|z|$ The maximum of each distribution is normalized to 1 and the vertical scale is consistent across all panels.}
    \label{fig:ofe-bimodality}
    \script{ofe_bimodality_summary.py}
\end{figure*}

\subsection{The [O/Fe]--[Fe/H] Plane}
\label{sec:ofe-feh}

Figure \ref{fig:ofe-feh-sfh} compares the [O/Fe]--[Fe/H] plane in the Solar neighborhood ($7\leq R_{\rm gal}<9$ kpc, $0\leq|z|<0.5$ kpc) between our four model SFHs. 

\begin{figure}
    \centering
    \includegraphics{figures/ofe_feh_sfh.pdf}
    \caption{A comparison of the [O/Fe]--[Fe/H] plane between the four SFH models in our multi-zone simulations. All assume an exponential DTD with $\tau=1.5$ Gyr. Each panel plots a random mass-weighted sample of 10,000 star particles in the Solar neighborhood ($7\leq R_{\rm gal}<9$ kpc, $0\leq|z|<0.5$ kpc) color-coded by the Galactocentric radius at birth. A Gaussian scatter has been applied to all points based on the median abundance errors in APOGEE DR17 (ses Table \ref{tab:uncertainties}). The black curves represent the ISM abundance tracks in the 8 kpc zone. The red contours represent a 2-D kernel density estimate with a bandwidth of 0.03 of the APOGEE abundance distribution in that Galactic region, with the solid and dashed contours enclosing 30\% and 80\% of stars in the sample, respectively.}
    \label{fig:ofe-feh-sfh}
    \script{ofe_feh_sfh.py}
\end{figure}

\begin{figure*}
    \centering
    \includegraphics[width=\linewidth]{figures/ofe_feh_dtd.pdf}
    \caption{The [O/Fe]--[Fe/H] plane from multi-zone simulations with different DTD models. All assume the inside-out SFH. Each panel is similar to those in Figure \ref{fig:ofe-feh-sfh}, except each row contains star particles from a different bin in $|z|$, with stars closest to the midplane in the bottom row and stars farthest from the midplane in the top row. All panels present stars within the Solar annulus ($7\leq R_{\rm gal}<9$ kpc).}
    \label{fig:ofe-feh-dtd}
    \script{ofe_feh_dtd.py}
\end{figure*}

\subsection{The Age--[O/Fe] Plane}
\label{sec:age-ofe}

\begin{figure}
    \centering
    \includegraphics{figures/age_ofe_sfh_alt.pdf}
    \caption{A comparison of the age--[O/Fe] relation between multi-zone simulations with different SFH models. All assume an exponential DTD with $\tau=1.5$ Gyr. Each panel plots a random mass-weighted sample of 10,000 star particles in the Solar neighborhood ($7\leq R_{\rm gal}<9$ kpc, $0\leq|z|<0.5$ kpc) color-coded by the Galactocentric radius at birth. A Gaussian scatter has been applied to all points based on the median [O/Fe] error from APOGEE DR17 and the median age error from \citet{Leung2023-Ages} (see Table \ref{tab:uncertainties}). Black squares represent the mass-weighted median age of star particles within bins of [O/Fe] with a width of 0.05 dex, and the horizontal black error bars encompass the 16th and 84th percentiles. Red triangles represent the median age from \citet{Leung2023-Ages} in bins of [O/Fe], and the horizontal red error bars encompass the 16th and 84th percentiles. For clarity, bins which contain less than 1\% of the total mass (in the simulations) or total number of stars (in the data) are not plotted.}
    \label{fig:age-ofe-sfh}
    \script{age_ofe_sfh_alt.py}
\end{figure}

\begin{figure*}
    \centering
    \includegraphics[width=\linewidth]{figures/age_ofe_dtd_alt.pdf}
    \caption{A comparison of the age--[O/Fe] relation between multi-zone simulations with different DTD models. All assume the early-burst SFH. Each row contains star particles from a different bin in $|z|$, with stars closest to the midplane in the bottom row and stars farthest from the midplane in the top row. In all panels stars are limited to the Solar annulus ($7\leq R_{\rm gal}<9$ kpc), and the layout of each panel is as in Figure \ref{fig:age-ofe-sfh}.}
    \label{fig:age-ofe-dtd}
    \script{age_ofe_dtd_alt.py}
\end{figure*}


\section{Discussion}
\label{sec:discussion}

We've run a lot of models and need some semi-quantitative way to say that some models do better than others. We compare our multi-zone outputs to APOGEE across five diagnostics: the [Fe/H] DF, [O/Fe] DF, [Fe/H]--[O/Fe] plane, age--[O/Fe] plane, and bimodality in the [O/Fe] DF. We perform statistical tests between APOGEE and simulation outputs in each region of the galaxy, then average the scores together (weighted by the number of APOGEE targets in each galactic region) to obtain a single numerical score for each simulation. Note that we are not trying to use these scores to fit our models to the data, nor are we trying to rule out any SFH in general. We use these scores to get a handle on which combinations of SFH and DTD are favorable or disfavorable in certain regimes. 

To quantify the difference between the abundance distributions generated by VICE and those observed in APOGEE, we compute the Kullback-Leibler (KL) divergence \citep{KullbackLeibler1951}, defined as
\begin{equation}
    D_{\rm{KL}}(P \parallel Q) \equiv \int_{-\infty}^{\infty} p(x) \log\Big(\frac{p(x)}{q(x)}\Big) dx
    \label{eq:kl-divergence}
\end{equation}
for distributions $P$ and $Q$ with probability density functions $p(x)$ and $q(x)$. A KL divergence of 0 indicates that the two distributions contain equal amounts of information. In this case, $P$ is the APOGEE DF, $Q$ is the model DF, and $x$ is either [Fe/H] or [O/Fe]. For each model SFH and DTD, we compute $D_{\rm{KL}}$ in 18 different Galactocentric regions: bins in $R_{\rm gal}$ of $3-5$ kpc, $5-7$ kpc, $7-9$ kpc, $9-11$ kpc, $11-13$ kpc, and $13-15$ kpc, and bins in $|z|$ of $0-0.5$ kpc, $0.5-1$ kpc, and $1-2$ kpc. The KL divergence for the entire model is taken to be the average of $D_{\rm{KL}}$ for each region weighted by the number of APOGEE stars in that region.

\citet{PerezCruz2008-KLTest2D} present a method for estimating the KL divergence between two continuous, multivariate samples using a $k$-nearest neighbor estimate. We implement this method to calculate the divergence between the multi-zone models and APOGEE data in [Fe/H]---[O/Fe] space.\footnote{We got the implementation from \url{https://mail.python.org/pipermail/scipy-user/2011-May/029521.html}.}

We employ a different test in the age---[O/Fe] plane. [motivated by the large age uncertainties?] As shown in Figures \ref{fig:age-ofe-sfh} and \ref{fig:age-ofe-dtd}, in each Galactic region we bin the model outputs and data into bins of [O/Fe] with a width of 0.05 dex. We define the RMS median age difference for the region as
\begin{equation}
    \Delta\tau_{\rm RMS} \equiv \sqrt{\frac{\sum_k \Delta\tau_k^2 n_{{\rm L23},k}}{n_{\rm L23,tot}}}
    \label{eq:age-ofe-score}
\end{equation}
where $\Delta\tau_k=\rm{med}(\tau_{\rm VICE})-\rm{med}(\tau_{\rm L23})$ is the difference between the mass-weighted median star particle age in VICE and the median stellar age from \citet{Leung2023-Ages} in bin $k$, $n_{{\rm L23},k}$ is the number of stars from the \citet{Leung2023-Ages} age sample in bin $k$, and $n_{\rm L23,tot}$ is the total number of stars in the sample in that Galactic region. As before, the score for the model as a whole is the average of $\Delta \tau_{\rm RMS}$ across all regions, weighted by the number of stars in the age sample in each region.

The results of these tests are presented in Table \ref{tab:results}. Instead of reporting the numerical scores for each simulation output, in the column corresponding to a particular diagnostic we simply write \yes, \meh, or \no, which corresponds to a score in the top, middle, or bottom tirtile out of all of our simulations, respectively. A \yes indicates that particular combination of model SFH and DTD performed better than most when compared to the APOGEE sample for that diagnostic.

\variable{output/summary_table.tex}

Compare to previous studies: \citet{Poulhazan2018-PrecisionPollution,Palicio2023-AnalyticDTD}.

\citet{Poulhazan2018-PrecisionPollution} found that DTD models with a significant prompt component (power-law and two-population) produced the highest average stellar abundance of [O/Fe]. They also found that these models produced the narrowest [O/Fe] distributions and suppressed the long tail to low [O/Fe] ratios. Conversely, their models with essentially 0 prompt SNe Ia produced a lower average stellar [O/Fe] ratio at late times and a broader distribution of [O/Fe]. Their isolated galaxy simulations have an SFR which features an early ($<1$ Gyr) burst followed by a roughly exponential decline, qualitatively similar to our early-burst model but with the peak of star formation occuring $\sim3$ Gyr earlier. We note that their simulations were not designed to reproduce a Milky Way-like galaxy; as such, their [O/Fe] distributions peak around $\sim0.45$ which is higher than the APOGEE high-$\alpha$ sequence. Nevertheless, our study reproduces their finding that a DTD with more prompt SNe Ia will produce a narrower distribution of [O/Fe]. One exception to this is the exponential DTD, which has a marginally narrower distribution than the power-law DTD close to the midplane [numbers here] but a more prominent hith-$\alpha$ tail. This is likely a consequence of the fact that the exponential DTD has a much lower SN Ia rate at late times ($\sim10$ Gyr) than other models [expand on this].

\citet{Palicio2023-AnalyticDTD} compare seven different DTDs, including the analytical SD, DD-WIDE, and DD-CLOSE, as well as the empirical two-population and $t^{-1}$ power-law DTDs, in a one-zone model with a two-infall SFH. [Note that they use some different parameters for the analytical models than we do). Critically for our purposes, and in contrast to previous studies of the two-infall model \citep[e.g.,][others]{Chiappini1997-TwoInfall,Spitoni2021-TwoInfall}, the authors include the effects of outflows, although they explore mass-loading values of $\eta=0.2-0.8$ which is significantly lower than in our simulations.

\subsection{Radial Migration \& Bimodality}

Something about \citet{Johnson2021-Migration} and \citet{Schonrich2009-RadialMixing}.

We need to constrain DTD or galactic SFH better. With SDSS-V, the latter seems much more achievable.

\section{Conclusions}
\label{sec:conclusions}

\begin{itemize}

    \item Our diagnostics consistently favor a DTD with a large fraction of delayed SNe Ia, such as the 3 Gyr exponential, 1 Gyr plateau, or triple system evolution models, over a DTD with a high number of prompt SNe Ia, such as the $t^{-1.4}$ power-law or two-population models. [Discuss the disconnect and/or complement with \citet{Maoz2017-CosmicDTD}, etc.; something about different timescales probed by volumetric surveys vs. us]
    
    \item While the choice of DTD cannot produce a bimodal distribution of [O/Fe], it can affect the shape of the distribution, the ratio of high-$\alpha$ to low-$\alpha$ stars, and the location of the high-$\alpha$ sequence if it exists. The SFH is the critical determining factor in bimodality.

    \item The distribution of [Fe/H] depends strongly on the chosen SFH, but there is only a weak effect from the DTD.

    \item The early-burst SFH model is the only one which produced a bimodal distribution of [O/Fe] across the disk resembling APOGEE, from the models we investigate. [Add: summary of why everything else fails.]

    \item Something about how radial migration can't produce bimodality on its own.
    
\end{itemize}

\begin{acknowledgments}
    Funding for the Sloan Digital Sky 
    Survey IV has been provided by the 
    Alfred P. Sloan Foundation, the U.S. 
    Department of Energy Office of 
    Science, and the Participating 
    Institutions. 
    
    SDSS-IV acknowledges support and 
    resources from the Center for High 
    Performance Computing  at the 
    University of Utah. The SDSS 
    website is www.sdss4.org.
    
    SDSS-IV is managed by the 
    Astrophysical Research Consortium 
    for the Participating Institutions 
    of the SDSS Collaboration including 
    the Brazilian Participation Group, 
    the Carnegie Institution for Science, 
    Carnegie Mellon University, Center for 
    Astrophysics | Harvard \& 
    Smithsonian, the Chilean Participation 
    Group, the French Participation Group, 
    Instituto de Astrof\'isica de 
    Canarias, The Johns Hopkins 
    University, Kavli Institute for the 
    Physics and Mathematics of the 
    Universe (IPMU) / University of 
    Tokyo, the Korean Participation Group, 
    Lawrence Berkeley National Laboratory, 
    Leibniz Institut f\"ur Astrophysik 
    Potsdam (AIP),  Max-Planck-Institut 
    f\"ur Astronomie (MPIA Heidelberg), 
    Max-Planck-Institut f\"ur 
    Astrophysik (MPA Garching), 
    Max-Planck-Institut f\"ur 
    Extraterrestrische Physik (MPE), 
    National Astronomical Observatories of 
    China, New Mexico State University, 
    New York University, University of 
    Notre Dame, Observat\'ario 
    Nacional / MCTI, The Ohio State 
    University, Pennsylvania State 
    University, Shanghai 
    Astronomical Observatory, United 
    Kingdom Participation Group, 
    Universidad Nacional Aut\'onoma 
    de M\'exico, University of Arizona, 
    University of Colorado Boulder, 
    University of Oxford, University of 
    Portsmouth, University of Utah, 
    University of Virginia, University 
    of Washington, University of 
    Wisconsin, Vanderbilt University, 
    and Yale University.
    
    This work has made use of data from the European Space Agency (ESA) mission
    {\it Gaia} (\url{https://www.cosmos.esa.int/gaia}), processed by the {\it Gaia}
    Data Processing and Analysis Consortium (DPAC,
    \url{https://www.cosmos.esa.int/web/gaia/dpac/consortium}). Funding for the DPAC
    has been provided by national institutions, in particular the institutions
    participating in the {\it Gaia} Multilateral Agreement.

    % This work made use of Astropy:\footnote{http://www.astropy.org} a community-developed core Python package and an ecosystem of tools and resources for astronomy \citep{astropy:2013, astropy:2018, astropy:2022}. 
\end{acknowledgments}

\software{VICE \citep{JohnsonWeinberg2020-Starbursts}, Astropy \citep{astropy2013,astropy2018,astropy2022}, scikit-learn \citep{Pedregosa2011-ScikitLearn}, SciPy \citep{2020SciPy-NMeth}, Matplotlib \citep{Hunter2007-Matplotlib}}

\appendix

\section{Reproducibility}
\label{app:reproducibility}

This study was carried out using the reproducibility software
\href{https://github.com/showyourwork/showyourwork}{\showyourwork}
\citep{Luger2021-showyourwork}, which leverages continuous integration to
programmatically download the data from
\href{https://zenodo.org/}{zenodo.org}, create the figures, and
compile the manuscript. Each figure caption contains two links: one
to the dataset stored on zenodo used in the corresponding figure,
and the other to the script used to make the figure (at the commit
corresponding to the current build of the manuscript). The git
repository associated to this study is publicly available at
\url{\GitHubURL}, and the release v.X.X allows anyone to re-build the entire 
manuscript. The datasets are stored at [URL].

\section{Analytical DTDs}
\label{app:analytical-dtds}

Discuss \citet{Greggio2005-AnalyticalRates} vs our simple function approximations here?

\citet{Greggio2005-AnalyticalRates} derives analytical DTDs for SD and DD progenitor systems from assumptions about binary stellar evolution and outcomes of mass exchange. They find that the parameters which have a large effect on the shape of the DTD are the distribution and range of stellar masses in progenitor systems; the efficiency of accretion in the SD scenario; and the distribution of separations at birth in the DD scenario. The left-hand panel of Figure \ref{fig:analytical-dtd} shows the analytical DTDs for SD progenitors and two different prescriptions for DD progenitors (``WIDE'' and ``CLOSE''). In the ``WIDE'' scheme, it is assumed that there is a wide distribution of ratios $A/A_0$ of the separation of the DD system to the initial separation of the binary, and that the distributions of $A$ and total mass of the system $m_{\rm DD}$ are independent, so one cannot necessarily predict the total merge time of a system based on its initial parameters. In the ``CLOSE'' scheme, there is assumed to be a narrow distribution of $A/A_0$ and a correlation between $A$ and $m_{\rm DD}$, so the most massive binaries tend to merge quickly and the least massive merge last.

The right-hand panel of Figure \ref{fig:analytical-dtd} shows the results of one-zone chemical evolution models with the \citet{Greggio2005-AnalyticalRates} DTDs. We assume $\eta=2.5$, $\tau_*=2$ Gyr, an inside-out SFH evaluated at a radius of 8 kpc, a continuous recycling approximation, and a minimum SN Ia delay of 40 Myr. We compare the SD, DD WIDE, and DD CLOSE schemes to standard power-law, broken power-law, and exponential DTDs. The SD and DD CLOSE DTDs follow nearly identical tracks in [O/Fe] vs [Fe/H]; however, their distributions on [O/Fe] differ at the low end. The SD DTD follows an exponential with a 1.5 Gyr timescale, whereas the DD CLOSE DTD is well-approximated by a broken power-law with an initial plateau of 300 Myr and a subsequent declining slope of -1.1. The WIDE prescription is likewise best approximated by a broken power-law, but with a longer plateau width of 1 Gyr. In all cases, the difference between the analytical DTD and the nearest broken power-law or exponential is likely too small to be observationally detectable, so in our multi-zone models we implement the more generic functions in lieu of the analytical DTDs.

A multi-zone with the \citet{Greggio2005-AnalyticalRates} SD DTD produced nearly identical results to an exponential DTD with a 1.5 Gyr timescale.

\begin{figure*}
    \centering
    \includegraphics[width=0.49\linewidth]{figures/dtd_analytical.pdf}
    \includegraphics[width=0.49\linewidth]{figures/onezone_analytical_dtd.pdf}
    \caption{\textit{Left:} Analytical DTDs from \citet[][solid curves]{Greggio2005-AnalyticalRates} and simple DTD functions (dashed curves). Some functions are presented with a constant multiplicative factor for visual clarity. \textit{Right:} Abundance tracks and distributions from one-zone models with the analytical and simple DTDs (same color scheme). For clarity, we vary the mass-loading factor to be $\eta=4$, $\eta=2$, and $\eta=1$ for the red, green, and blue curves, respectively. All simulation parameters between the similarly-colored solid and dashed curves are identical.}
    \label{fig:analytical-dtd}
    \script{analytical_dtd_twopanel.py}
\end{figure*}

\section{Stellar Migration}
\label{app:migration}

For each star particle in VICE, \citet{Johnson2021-Migration} randomly assign an analogue star particle from \texttt{h277} and adopt its radial migration distance $\Delta R$ and final midplane distance $z$. This allows VICE to adopt a realistic pattern of radial migration without needing to implement its own hydrodynamical simulation. However, in regions where the number of h277 star particles is relatively low, such as at large $R_{\rm gal}$ and small $t$, a single h277 star particle can be assigned as an analogue to multiple VICE stellar populations. These populations will have similar formation and migration histories and consequently similar abundances, which produces unphysical ``clumps'' of stars in the abundance distributions at high latitudes and large radii. 

We fit a Gaussian to the distribution of $\Delta R = R_{\rm final} - R_{\rm initial}$ from the \texttt{h277} output in bins of $R_{\rm initial}$ and age. Each Gaussian is centered at 0 and we find that the scale $\sigma_{\Delta R}$ is best described by the function
\begin{equation}
    \sigma_{\Delta R} = 1.35\,{\rm kpc} \Big(\frac{R_{\rm form}}{8\,{\rm kpc}}\Big)^{0.61} \Big(\frac{\tau}{1\,{\rm Gyr}}\Big)^{0.33}
    \label{eq:radial-migration}
\end{equation}
where $\tau$ is the age of the star particle. 

We fit a sech$^2$ function \citep{Spitzer1942} to the distribution of midplane distances $z$. Vertical migration away from the midplane does not affect the chemical evolution simulation, but we do use $z$ in our analysis. The probability density function (PDF) of $z$ given some scale height $h_z$ is
\begin{equation}
    {\rm PDF}(z) = \frac{1}{4 h_z} {\rm sech}^2\Big(\frac{z}{2 h_z}\Big)
    \label{eq:sech-pdf}
\end{equation}
and the corresponding cumulative distribution function (CDF) is
\begin{equation}
    {\rm CDF}(z) = \frac{1}{1 + e^{-z / h_z}}.
    \label{eq:sech-cdf}
\end{equation}
We fit the above function to the distributions of $z$ in h277 in varying bins of $\tau$ and $R_{\rm final}$ and found that $h_z$ is best described by the function
\begin{equation}
    h_z = (0.25\,{\rm kpc}) 
    e^{\frac{\tau-5\,{\rm Gyr}}{7.0\,{\rm Gyr}}}
    e^{\frac{R_{\rm final}-8\,{\rm kpc}}{6.0\,{\rm kpc}}}.
    \label{eq:scale-height}
\end{equation}

When a star particle is created by VICE at initial radius $R_{\rm form}$, we sample its total radial migration distance $\Delta R$ from a Gaussian with a width described by Equation \ref{eq:radial-migration}, and we sample its final midplane distance $z_{\rm final}$ from the distribution described by Equation \ref{eq:sech-pdf} with a width given by Equation \ref{eq:scale-height}. The star particle migrates to its final radius $R_{\rm final}$ in a similar manner to the ``diffusion'' case from \citet{Johnson2021-Migration}, but with a time dependence $\propto \Delta t^{1/3}$. This produces distributions of $R_{\rm final}$ and $z_{\rm final}$ which are similar to the analogue migration case for all but the oldest stars. 

\citet{Okalidis2022-AurigaMigration} studied the stellar migration in the Auriga simulations and found that the migration strength had both a dependence on age and formation radius. They also found that in strongly-barred galaxies, the migration is stronger overall but has a slower timestep evolution than diffusion ($\Delta t^{0.5}$).

Figure \ref{fig:radial-migration} compares the distributions of $R_{\rm final}$ in bins of $R_{\rm form}$ and age between the analogue and Gaussian migration schema. 

\begin{figure*}
    \centering
    \includegraphics[width=\linewidth]{figures/radial_migration.pdf}
    \caption{The distribution of final radius $R_{\rm final}$ as a function of formation radius $R_{\rm form}$ and age for the h277 analogue (top row) and Gaussian sampling scheme (bottom row). From left to right, star particles are binned by formation annulus, from the inner disk (far left column) to the Solar annulus (center column) to the outer disk (far right column). Within each panel, colored curves represent the different age bins, ranging from the youngest stars (dark blue) to the oldest (dark red). In the top row, we exclude age bins with fewer than 100 unique analogue IDs for clarity. All distributions are normalized so that the area under the curve is 1, and have been smoothed by a boxcar function with a width of 0.5 kpc. The vertical dotted black lines indicate the bounds of each bin in $R_{\rm form}$; stars within that region of the distribution have not migrated outside their birth annulus over their lifetime.}
    \label{fig:radial-migration}
    \script{radial_migration.py}
\end{figure*}

\begin{figure*}
    \centering
    \includegraphics[width=\linewidth]{figures/midplane_distance.pdf}
    \caption{Similar to Figure \ref{fig:radial-migration} but for the distribution of final midplane distance $z_{\rm final}$ as a function of final radius and age. From left to right, star particles are binned by \textit{final} annulus. In the top row, we exclude age bins with fewer than 500 unique analogue IDs for clarity. All distributions have been smoothed by a boxcar function with a width of 0.1 kpc.}
    \label{fig:midplane-distance}
    \script{midplane_distance.py}
\end{figure*}

\bibliography{bib}

\end{document}
