% Define document class
\documentclass[twocolumn,linenumbers,twocolappendix]{aastex631}

\usepackage{showyourwork}
\usepackage{amsmath}
\usepackage{amssymb}
\usepackage{layouts}
\usepackage{xcolor}
% \usepackage{siunitx}

% custom unit definitions
% \DeclareSIUnit\year{yr}

% user-defined commands
\newcommand{\yes}{\textcolor{green}{\checkmark}}
\newcommand{\meh}{\textcolor{black}{$\sim$}}
\newcommand{\no}{\textcolor{red}{$\times$}}

% Begin!
\begin{document}

% Title
\title{The Galactic Delay-Time Distribution of Type Ia Supernovae:\\
    A Chemical Evolution Perspective}

% Author list
\author[0000-0003-3781-0747]{Liam O. Dubay}
\author[0000-0001-7258-1834]{Jennifer A. Johnson}
\author[0000-0002-6534-8783]{James W. Johnson}
\author[0000-0001-7775-7261]{David H. Weinberg}

% Abstract with filler text
\begin{abstract}
    Abstract.
\end{abstract}

% Main body with filler text
\section{Introduction}

\section{Methods}
\label{sec:methods}

\subsection{DTD Models}
\label{sec:dtd-models}

\begin{figure}
    \centering
    \includegraphics{figures/delay_time_distributions.pdf}
    \caption{Delay time distribution functions}
    \label{fig:dtds}
    \script{delay_time_distributions.py}
\end{figure}

\subsubsection{From Observations}

\citet{Maoz2014-Review}

A typical form for the DTD is a power-law $t^\alpha$ where $\alpha\sim-1$. \citet{Maoz2017-CosmicDTD} compare volumetric SN Ia rates to the cosmic star formation history. They get a power law with a slope of $-1.13\pm 0.06$ for field galaxies, and a slope of $-1.39^{+0.32}_{-0.05}$ for galaxy clusters.

\citet{Heringer2019-FieldGalaxyDTD} study the distribution of SNe Ia in field galaxies and get a power-law index of $-1.34^{+0.19}_{-0.17}$ which is closer to Maoz's cluster DTD.

\citet{Wiseman2021-DESRates} measure the SN Ia rate in the redshift range $0.2 < z < 0.6$ and get a similar power law to \citet{Maoz2017-CosmicDTD} with a slope of $-1.13 \pm 0.05$. They also find that slower-declining SNe have a steeper DTD slope.

Another commonly assumed functional form is an exponential. \citet{Schonrich2009-RadialMixing} take their DTD to be an exponential with a 1.5 Gyr timescale, though this isn't strictly observationally motivated.

\citet{Stolger2020-ExponentialDTD} perform maximum likelihood estimations based on the cosmic star formation history and the star formation histories of individual galaxies. In both cases their best fit functions were approximately exponential, though they started with a skew-normal function so could not have ended up with a power law even if they wanted to. They don't actually report the timescale of this exponential, but based on the plots I'd say it's about 1.5 - 2 (1.7?) Gyr. Their best fit parameters make no sense to me as they don't report their units.

% \begin{deluxetable}{lCl}
% \tablecaption{Sources of the various forms of delay time distribution.}
% \tablehead{
% \colhead{Name} & \colhead{Form} & \colhead{Source}
% }
% \startdata
% Field Galaxy Power-Law      & t^{-1.1}              & \citet{Maoz2017-CosmicDTD} \\   
% Galaxy Cluster Power-Law    & t^{-1.4}              & \citet{Maoz2017-CosmicDTD} \\
% Short Exponential           & e^{-t/1.5~\rm{Gyr}}  & \citet{Schonrich2009-RadialMixing} \\
% \enddata
% \end{deluxetable}

\subsubsection{Theoretical Models}

\citet{Matteucci1986-SupernovaEnrichment}

\citet{Greggio2005-AnalyticalRates}

\citet{Rajamuthukumar2022-TripleEvolution}

\citet{Schonrich2009-RadialMixing,Weinberg2017-ChemicalEquilibrium} assume an exponential with $\tau=1.5$ Gyr

\begin{figure}
    \centering
    \includegraphics{figures/dtd_greggio05.pdf}
    \caption{Analytical DTDs from \citet[][solid curves]{Greggio2005-AnalyticalRates} and close approximation simple functions (dashed curves). Some functions are presented with a constant multiplicative factor for visual clarity.}
    \label{fig:dtd-greggio05}
    \script{dtd_greggio05.py}
\end{figure}

\subsubsection{Our Models}

We explore four different classes of DTD: a power-law, a broken power-law with an initial flat plateau, an exponential, and an exponential with a prompt Gaussian component, with one or two different parameterizations for each. Figure \ref{fig:dtds} presents these DTDs as a function of time since star formation.

The power-law DTD is motivated by studies of the cosmic Ia rate \citep{Maoz2017-CosmicDTD,Wiseman2021-DESRates}, and we explore slopes of $\alpha=-1.1$ and $\alpha=-1.4$ which correspond to the field galaxy and galaxy cluster DTDs of \citet{Maoz2017-CosmicDTD}, respectively. We explore the broken power-law with plateau widths of 300 Myr and 1 Gyr \textemdash the first as a proxy for both 

\begin{enumerate}
    \item A simple power law with slope $\alpha$:
    \begin{equation}
        R(t) = A_\alpha (t/1\,\rm{Gyr})^\alpha
        \label{eq:powerlaw-dtd}
    \end{equation}
    where $A_\alpha = (\alpha+1)(t_{\rm max}^{\alpha+1} - t_{\rm min}^{\alpha+1})^{-1}$ is the normalization coefficient.
    We investigate the cases $\alpha=-1.1$ and $\alpha=-1.4$, following the field galaxy and galaxy cluster DTDs of \citet{Maoz2017-CosmicDTD}, respectively.
    
    \item A ``plateau'' model consisting of a flat slope of width $W$ followed by a power law decline with normalization $B_W$:
    \begin{equation}
        R(t) = 
        \begin{cases}
            B_W, & t < W \\
            B_W (t/W)^\alpha, & t \ge W
        \end{cases}
        \label{eq:plateau-dtd}
    \end{equation}
    where $B_W=[(W-t_{\rm min}) + \frac{t_{\rm max}^{\alpha+1} - W^{\alpha+1}}{(\alpha+1)W^\alpha}]^{-1}$ is the normalization coefficient.
    We investigate the cases $W=0.3$ Gyr and $W=1$ Gyr, taking $\alpha=-1.1$ for all plateau models. Figure \ref{fig:dtd-greggio05} illustrates the similarity between these models and two different treatments of the analytical double-degenerate DTD from \citet{Greggio2005-AnalyticalRates}.

    \item An exponentially declining DTD with timescale $\tau$ and normalization $C_\tau$:
    \begin{equation}
        R(t) = 1/\tau \exp(-t/\tau)
        \label{eq:exponential-dtd}
    \end{equation}
    We investigate the case where $\tau=1.5$ Gyr, which has been used in previous studies \citep[e.g.,][]{Schonrich2009-RadialMixing,Weinberg2017-ChemicalEquilibrium} and is similar to the analytical single-degenerate DTD from \citet{Greggio2005-AnalyticalRates}. We also investigate a longer timescale of $\tau=3$ Gyr.

    \item A DTD in which 50\% of SNe Ia belong to a ``prompt'' Gaussian component at small $t$ and the other 50\% form an exponential tail at large $t$, normalized by $D_{t_p,\sigma,\tau}$:
    \begin{equation}
        R(t) = D_{t_p,\sigma,\tau} \Big[\exp\Big(-\frac{(t-t_p)^2}{2\sigma^2}\Big) + \exp(-t/\tau)\Big]
        \label{eq:prompt-dtd}
    \end{equation}
    Following the two-population model from \citet{Mannucci2006-TwoPopulations}, we take $t_p=50$ Myr, $\sigma=15$ Myr, and $\tau=3$ Gyr, which results in roughly 50\% of SNe Ia exploding within $t<100$ Myr.

    \item A DTD based on simulations of triple-system evolution by \citet{Rajamuthukumar2022-TripleEvolution}. We approximate their DTD as a special case of the plateau model (Equation \ref{eq:plateau-dtd}) where the initial rate is quite low until an instantaneous rise to the plateau value at time $t_{\rm rise}$:
    \begin{equation}
        R(t) = 
        \begin{cases}
            \epsilon B_W, & t < t_{\rm rise} \\
            B_W, & t_{\rm rise} \leq t < W \\
            B_W (t / W) ^ \alpha, & t \geq W
        \end{cases}
        \label{eq:triple-dtd}
    \end{equation}
    We take $t_{\rm rise}=0.5$ Gyr, $W=0.5$ Gyr, $\alpha=-1.1$, and $\epsilon=0.05$ so the initial rate is 5\% of the peak rate.
\end{enumerate}

\subsection{APOGEE Data}
\label{sec:apogee-data}

We use APOGEE DR17 \citep{Abdurro'uf2022-SDSSIV-DR17}. Our selection criteria are presented in Table \ref{tab:sample-selection}, producing a final sample of 192,926 stars.

\begin{deluxetable*}{lll}
    \tablecaption{Sample Selection\label{tab:sample-selection}}
    \tablehead{
        \colhead{Parameter} & \colhead{Range} & \colhead{Notes}
    }
    \startdata
        $\log g$            & $1.0 < \log g < 3.8$          & Select giants only \\
        $T_{\rm eff}$       & $3500 < T_{\rm eff} < 5500$ K & Reliable temperature range \\
        $S/N$               & $S/N > 80$                    & Required for accurate stellar parameters \\
        ASPCAPFLAG Bits     & $\notin$ 23                   & Remove stars flagged as bad \\
        EXTRATARG Bits      & $\notin$ 0, 1, 2, 3, or 4     & Select main red star sample only \\
    \enddata
\end{deluxetable*}

\subsubsection{Stellar Ages}
\label{sec:stellar-ages}

191,179 stars in our sample have astroNN ages and 71,684 have ages from \citet{Leung2023-Ages}.

% For $19784$ stars, or $\sim3.5\%$ of the sample, the astroNN [Fe/H] abundance estimate is $\gtrsim0.5$ dex lower than the value provided by ASPCAP. The vast majority of these stars are on the lower-main sequence according to \textit{Gaia} photometry, so we exclude stars with $\log(g) > 4$ in ASPCAP. In addition, 197 or $\sim3\%$ of stars with APOKASC-2 asteroseismic ages are reported to be $>5$ Gyr younger by astroNN. The APOKASC-2 age is unphysically high for many of these stars, often exceeding 15 Gyr. We identify prior mass loss as the cause of this discrepancy, so the astroNN ages are likely closer to the truth.

\subsection{Chemical Evolution Models}
\label{sec:vice-models}

\subsubsection{Stellar Migration}
\label{sec:migration}

\citet{Johnson2021-Migration}

We fit a Gaussian to the distribution of $\Delta R = R_{\rm final} - R_{\rm initial}$ from the \texttt{h277} output in bins of $R_{\rm initial}$ and age. Each Gaussian is centered at 0 and we find that the scale $\sigma_{\Delta R}$ is best described by the function

\begin{equation}
    \sigma_{\Delta R} = 1.35\,{\rm kpc}(R_{\rm form}/8\,{\rm kpc})^{0.61} (\tau/1\,{\rm Gyr})^{0.33}
    \label{eq:radial-migration}
\end{equation}

\noindent where $\tau$ is the age of the star particle. 

We fit a sech$^2$ function \citep{Spitzer1942} to the distribution of midplane distances $z_{\rm final}$. Vertical migration away from the midplane does not affect the chemical evolution simulation, but we do use $z_{\rm final}$ in our analysis. The probability density function (PDF) of $z_{\rm final}$ given some scale height $h_z$ is

\begin{equation}
    {\rm PDF}(z_{\rm final}) = \frac{1}{4 h_z} {\rm sech}^2\Big(\frac{z_{\rm final}}{2 h_z}\Big)
    \label{eq:sech-pdf}
\end{equation}

\noindent and the corresponding cumulative distribution function (CDF) is

\begin{equation}
    {\rm CDF}(z_{\rm final}) = \frac{1}{1 + \exp(-z_{\rm final} / h_z)}.
    \label{eq:sech-cdf}
\end{equation}

\noindent We fit the above function to the distributions of $z_{\rm final}$ in h277 in varying bins of $\tau$ and $R_{\rm final}$ and found that $h_z$ is best described by the function

\begin{equation}
    h_z = (0.25\,{\rm kpc}) 
    \exp\Big(\frac{\tau-5\,{\rm Gyr}}{7.0\,{\rm Gyr}}\Big)
    \exp\Big(\frac{R_{\rm final}-8\,{\rm kpc}}{6.0\,{\rm kpc}}\Big)
    \label{eq:scale-height}
\end{equation}

\section{Results from Single-Zone Models}
\label{sec:onezone-results}

All models are run for a duration of 13.2 Gyr with a simulation timestep of 0.01 Gyr.

\subsection{The SN Ia Delay-Time Distribution}

Adjusting the parameters of the DTD affects how steeply the abundance track in [Fe/H]-[O/Fe] space falls from its initial ``plateau'' and the location of the ``knee.'' Figure \ref{fig:onezone-slope-timescale} shows abundance tracks from one-zone models which assume different slopes of a power-law DTD or different timescales of an exponential DTD. A steeper power-law slope results in a steeper decline in [O/Fe] and a lower knee, as does a shorter exponential timescale. The consequence is that a very shallow power-law produces a similar abundance track to a very steep exponential, which suggests these models may be difficult to distinguish in this parameter space. However, these cases can be distinguished by their MDFs and [O/Fe] distributions, where the power-law DTD produces a narrower, less skewed distribution while the exponential DTD produces a broader, more skewed distribution. In the [O/Fe] distribution especially, an exponential DTD with a 6 Gyr timescale produces a large number of high-$\alpha$ stars.

\begin{figure}
    \centering
    \includegraphics{figures/onezone_slope_timescale.pdf}
    \caption{\textit{Center:} Abundance tracks in the [Fe/H]-[O/Fe] plane for one-zone models with various DTD parameters. Tracks from models which assume a power-law DTD are in black, with dotted, dashed, and solid curves representing power-law slopes ($\alpha$) of $-0.8$, $-1.1$, and $-1.4$, respectively. Tracks from models which assume an exponential DTD are in blue, with dotted, dashed, and solid curves representing exponential decay timescales ($\tau$) of 1.5 Gyr, 3 Gyr, and 6 Gyr, respectively. In all cases a minimum delay time of 40 Myr is assumed. \textit{Top:} MDFs generated by the one-zone models. \textit{Right:} Distribution of [O/Fe] generated by the one-zone models.}
    \label{fig:onezone-slope-timescale}
    \script{onezone_slope_timescale.py}
\end{figure}

Figure \ref{fig:onezone-dtd} compares the one-zone model outputs from the full range of DTDs we investigate in this paper. In general, the choice of DTD has the greatest effect on the location of the knee in [Fe/H]-[O/Fe] space (center panel) and the high-$\alpha$ end of the distribution of [O/Fe] (right panel). The DTD also has a smaller effect on the MDF in the range $-15.5\lesssim$ [Fe/H] $\lesssim-0.5$ (top panel).

\begin{figure}
    \centering
    \includegraphics{figures/onezone_dtd.pdf}
    \caption{Similar to Figure \ref{fig:onezone-slope-timescale} but with a different selection of DTDs. In all cases a minimum delay time of 40 Myr is assumed.}
    \label{fig:onezone-dtd}
    \script{onezone_dtd.py}
\end{figure}

\subsection{The Minimum SN Ia Delay Time}

Figure \ref{fig:onezone-dtd-delay} shows that the minimum SN Ia delay time has a much stronger effect on the abundance track and MDFs of models which assume a power-law DTD than models which assume an exponential, due to the much higher SN Ia rate in the power-law models at early times (see Figure \ref{fig:dtds}). Moreover, a power-law DTD with a long minimum delay may be observationally hard to distinguish from an exponential. 

\begin{figure}
    \centering
    \includegraphics{figures/onezone_dtd_delay.pdf}
    \caption{Similar to Figure \ref{fig:onezone-slope-timescale} but with varying minimum SN Ia delay time ($t_D$). Dashed curves represent models with a minimum delay time of 40 Myr, while solid curves represent models with a minimum delay time of 150 Myr. The black curves represent a power-law DTD with a slope of $-1.1$, while the blue curves represent an exponential DTD with a timescale of 1.5 Gyr.}
    \label{fig:onezone-dtd-delay}
    \script{onezone_dtd_delay.py}
\end{figure}

A 150 Myr minimum delay time is incompatible with the prompt DTD model, which has $\sim 50$ of SNe Ia explode in the first 100 Myr.

\subsection{The Star Formation History}

Figure \ref{fig:onezone-delay-taustar} shows results from one-zone models with a power-law DTD which differ in their assumed minimum SN Ia delay time ($t_D$) or star formation efficiency timescale ($\tau_*$). In [Fe/H]-[O/Fe] space, doubling or halving the minimum delay time produces similar results to doubling or halving the SFE timescale. A shorter minimum delay time means SNe Ia will start to explode sooner after the star formation burst, giving less time for CCSNe to explode at the plateau of the abundance track. Similarly, a longer SFE timescale means fewer massive stars will be produced by the initial star formation burst, resulting in fewer CCSNe before the first SNe Ia explode. 

The minimum delay and SFE timescale have similar effects on the abundance tracks but very different effects on the MDFs. Altering the SFE timescale changes the stellar distribution of [Fe/H], especially at the low-metallicity end, as shown in the top panel of Figure \ref{fig:onezone-delay-taustar}. However, as a change in the SFE timescale affects both the number of high-mass and low-mass stars produced, it does not affect the distribution of [O/Fe], as shown in the right-hand panel of Figure \ref{fig:onezone-delay-taustar}. Meanwhile, a change in the minimum SN Ia delay time does not impact the total number of SNe which explode, so the distribution of [Fe/H] is largely unaffected, while it does result in a change in the number of high-$\alpha$ stars formed and therefore the distribution of [O/Fe].

\begin{figure}
    \centering
    \includegraphics{figures/onezone_delay_taustar.pdf}
    \caption{Similar to Figure \ref{fig:onezone-slope-timescale} but with varying minimum SN Ia delay time ($t_D$) and star formation efficiency timescale ($\tau_*$). Tracks with $\tau_*=2$ Gyr are represented in black, while a track with a shorter $\tau_*=1$ Gyr is shown in red and a track with a longer $\tau_*=4$ Gyr is in yellow. The dotted, solid, and dashed black curves represent a minimum delay time of 40 Myr, 80 Myr, and 160 Myr, respectively. A power-law DTD is assumed in all models.}
    \label{fig:onezone-delay-taustar}
    \script{onezone_delay_taustar.py}
\end{figure}

\subsection{Analytical Formulations}

\citet{Greggio2005-AnalyticalRates} derives analytical DTDs for SD and DD progenitor systems from assumptions about binary stellar evolution and outcomes of mass exchange. They find that the parameters which have a large effect on the shape of the DTD are the distribution and range of stellar masses in progenitor systems; the efficiency of accretion in the SD scenario; and the distribution of separations at birth in the DD scenario. Figure \ref{fig:greggio05-dtd} shows the analytical DTDs for SD progenitors and two different prescriptions for DD progenitors (``WIDE'' and ``CLOSE''). In the ``WIDE'' scheme, it is assumed that there is a wide distribution of ratios $A/A_0$ of the separation of the DD system to the initial separation of the binary, and that the distributions of $A$ and total mass of the system $m_{\rm{DD}}$ are independent, so one cannot necessarily predict the total merge time of a system based on its initial parameters. In the ``CLOSE'' scheme, there is assumed to be a narrow distribution of $A/A_0$ and a correlation between $A$ and $m_{\rm{DD}}$, so the most massive binaries tend to merge quickly and the least massive merge last.

An implementation of these DTDs in a chemical evolution model requires numerical integration (see their Equations 14, 33, and 34, respectively, for the exact functional forms), so to save on computation time we approximate the DTD for the DD prescriptions as a broken power-law with an exponential rise and tail:

\begin{equation}
    f_{\rm{Ia}}^{\rm{DD}}(t) \propto
    \begin{cases}
        t^{\alpha_1} (1 - A_{\rm{rise}} e^{-t/\tau_{\rm{rise}}}) & t \le 1~\rm{Gyr} \\
        B t^{\alpha_2} (1 + A_{\rm{tail}} e^{-(t - 1~\rm{Gyr})/\tau_{\rm{tail}}}) & t > 1~\rm{Gyr}
    \end{cases}
    \label{eq:greggio05-approx}
\end{equation}

\noindent with the normalization $B$ of the second part set such that

\begin{equation}
    B = \frac{1}{1 + A_{\rm{tail}}} t^{(\alpha_1-\alpha_2)} (1 - A_{\rm{rise}} e^{-(1~\rm{Gyr})/\tau_{\rm{rise}}}).
\end{equation}

\noindent We use a non-linear least squares method to fit Equation \ref{eq:greggio05-approx} to the analytical DTDs for DD progenitors. We obtain best-fit parameters $\alpha_1=-0.19$, $A_{\rm{rise}}=1.7$, $\tau_{\rm{rise}}=0.084$ Gyr, $\alpha_2=-0.91$, $A_{\rm{tail}}=0.85$, and $\tau_{\rm{tail}}=0.088$ Gyr for the ``WIDE'' scheme, and $\alpha_1=-0.53$, $A_{\rm{rise}}=3.0$, $\tau_{\rm{rise}}=0.042$ Gyr, $\alpha_2=-1.1$, $A_{\rm{tail}}=0.62$, and $\tau_{\rm{tail}}=0.11$ Gyr for the ``CLOSE'' scheme.\footnote{
Our implementation of these best-fit approximations, as well as the original SD and DD formulations, is available in the GitHub repository for this paper.
} These best-fit approximations are represented by the dashed curves in Figure \ref{fig:greggio05-dtd} \citep[see also Figure 8 in][]{Greggio2005-AnalyticalRates}. The difference between the true analytical DTDs and our best-fit approximations is small enough as to be irrelevant for our purposes, so for the rest of this section we rely on the best-fit approximations for faster simulations.

% \begin{figure}
%     \centering
%     \includegraphics{figures/greggio05_dtd.pdf}
%     \caption{Analytical DTDs computed by \citet{Greggio2005-AnalyticalRates}. The solid red curve represents the SD DTD for Chandrasekhar explosions, while the green and blue solid curves represent, respectively, the CLOSE and WIDE formulations of a DD DTD. The best-fit approximations for the DD DTDs (see Equation \ref{eq:greggio05-approx}) are represented by the dashed olive and cyan curves, respectively. All DTDs are normalized to unity.}
%     \label{fig:greggio05-dtd}
%     \script{greggio05_dtd.py}
% \end{figure}

Figure \ref{fig:onezone-greggio05} shows the results of one-zone chemical evolution models with the \citet{Greggio2005-AnalyticalRates} DTDs. We assume $\eta=2.5$, $\tau_*=2$ Gyr, an inside-out SFH evaluated at a radius of 8 kpc, a continuous recycling approximation, and a minimum SN Ia delay of 40 Myr. We compare the SD, DD WIDE, and DD CLOSE schemes to standard power-law, broken power-law, and exponential DTDs. The SD and DD CLOSE DTDs follow nearly identical tracks in [O/Fe] vs [Fe/H]; however, their distributions on [O/Fe] differ at the low end. The SD DTD follows an exponential with a 1.5 Gyr timescale, whereas the DD CLOSE DTD is well-approximated by a broken power-law with an initial plateau of 350 Myr and a subsequent declining slope of -1.1. The WIDE prescription is likewise best approximated by a broken power-law, but with a longer plateau width of 1 Gyr. In all cases, the difference between the analytical DTD and the nearest broken power-law or exponential is likely too small to be observationally detectable, so in our multi-zone models we implement the more generic functions in lieu of the analytical DTDs.

A multi-zone with the \citet{Greggio2005-AnalyticalRates} SD DTD produced nearly identical results to an exponential DTD with a 1.5 Gyr timescale.

\begin{figure*}
    \centering
    \includegraphics{figures/onezone_greggio05_single.pdf}
    \includegraphics{figures/onezone_greggio05_double.pdf}
    \caption{Abundance tracks and distributions from one-zone models with the analytical DTDs from \citet{Greggio2005-AnalyticalRates}. \textit{Left:} a comparison between the analytical SD DTD (solid curve), an exponential DTD with a 1.5 Gyr timescale (dashed curve), and a power-law DTD with a slope of -1.1 (dotted curve). \textit{Right:} a comparison between two different prescriptions for an analytical DD DTD (solid curves), power-law DTDs with an initial plateau of 1 Gyr and 350 Myr (dashed curves), and a power-law DTD with no plateau and a slope of -1.1 (dotted curve).}
    \label{fig:onezone-greggio05}
    \script{onezone_greggio05.py}
\end{figure*}

\begin{figure}
    \centering
    \includegraphics{figures/onezone_prompt.pdf}
    \caption{Caption}
    \label{fig:onezone-prompt}
    \script{onezone_prompt.py}
\end{figure}

\section{Results from Multi-Zone Models}
\label{sec:multizone-results}

\begin{deluxetable}{lll}
    \tablecaption{Median Parameter Uncertainties\label{tab:uncertainties}}
    \tablehead{
        \colhead{Parameter} & \colhead{Median Uncertainty} & \colhead{Source}
    }
    \startdata
        [Fe/H] & $9.2\times10^{-3}$ & APOGEE \\
        $\rm [O/Fe]$ & $1.8\times10^{-2}$ & APOGEE \\
        log(Age/Gyr) & $0.12$ & \citet{Leung2023-Ages}
        % Age/Gyr & 29\% & astroNN
    \enddata
\end{deluxetable}

\subsection{Comparison to APOGEE MDFs}

To quantify the similarity between the MDFs generated by VICE and the observed APOGEE MDF, we compute the Kullback-Leibler (KL) divergence \citep{KullbackLeibler1951}, defined as

\begin{equation}
\label{eq:kl-divergence}
D_{\rm{KL}}(P \parallel Q) = \int_{-\infty}^{\infty} p(x) \log\Big(\frac{p(x)}{q(x)}\Big) dx
\end{equation}

\noindent for distributions $P$ and $Q$ with probability density functions $p(x)$ and $q(x)$. In this case, $P$ is the APOGEE MDF, $Q$ is the model MDF, and $x$ is [Fe/H]. For each model SFH and DTD, we compute $D_{\rm{KL}}$ in 18 different galactocentric zones defined by radial bins $3-5$ kpc, $5-7$ kpc, $7-9$ kpc, $9-11$ kpc, $11-13$ kpc, and $13-15$ kpc and absolute $z$-height bins $0-0.5$ kpc, $0.5-1$ kpc, and $1-2$ kpc. The KL divergence for the entire model is taken to be the average of $D_{\rm{KL}}$ for each zone, weighted by the number of APOGEE stars in each zone.

\subsection{Comparison to APOGEE ODFs}


\section{Discussion}
\label{sec:discussion}

Picking the team.

\begin{deluxetable*}{ll|cccc}
\label{tab:results}
\tablehead{
\colhead{DTD} & \colhead{SFH} & \colhead{MDF} & \colhead{ODF} & \colhead{[Fe/H]--[O/Fe]} & \colhead{Age--[O/Fe]}
}
\startdata
Power law               & Inside-out    & \meh  & \no   & \meh  & \no     \\
($\alpha=-1.1$)         & Late burst    & \meh  & \no   & \meh  & \no     \\
                        & Early burst   & \no   & \yes  & \meh  & \meh    \\
                        & Two-infall    & \no   & \no   & \meh  & \no     \\
\hline
Power law               & Inside-out    & \meh  & \no   & \no   & \no     \\
($\alpha=-1.4$)         & Late burst    & \meh  & \no   & \no   & \no     \\
                        & Early burst   & \no   & \yes  & \no   & \meh    \\
                        & Two-infall    & \no   & \no   & \no   & \no     \\
\hline
Exponential             & Inside-out    & \yes  & \no   & \yes  & \no     \\
($\tau=1.5$ Gyr)        & Late burst    & \yes  & \no   & \yes  & \no     \\
                        & Early burst   & \no   & \yes  & \meh  & \yes    \\
                        & Two-infall    & \yes  & \meh  & \meh  & \meh    \\
\hline
Exponential             & Inside-out    & \yes  & \meh  & \yes  & \meh    \\
($\tau=3$ Gyr)          & Late burst    & \yes  & \meh  & \yes  & \meh    \\
                        & Early burst   & \no   & \yes  & \no   & \yes    \\
                        & Two-infall    & \meh  & \meh  & \meh  & \meh    \\
\hline
Plateau                 & Inside-out    & \yes  & \meh  & \yes  & \no     \\
($W=0.3$ Gyr)           & Late burst    & \yes  & \no   & \yes  & \no     \\
                        & Early burst   & \no   & \yes  & \meh  & \yes    \\
                        & Two-infall    & \yes  & \meh  & \yes  & \meh    \\
\hline
Plateau                 & Inside-out    & \meh  & \meh  & \yes  & \meh    \\
($W=1.0$ Gyr)           & Late burst    & \meh  & \meh  & \yes  & \meh    \\
                        & Early burst   & \no   & \yes  & \no   & \yes    \\
                        & Two-infall    & \meh  & \meh  & \meh  & \meh    \\
\hline
Prompt                  & Inside-out    & \meh  & \no   & \no   & \no     \\
($t_{\rm max}=0.05$ Gyr)& Late burst    & \meh  & \no   & \no   & \no     \\
                        & Early burst   & \no   & \yes  & \no   & \yes    \\
                        & Two-infall    & \yes  & \no   & \no   & \no     \\
\hline
Triple system           & Inside-out    & \yes  & \meh  & \meh  & \yes    \\
($t_{\rm{max}}=1$ Gyr)  & Late burst    & \yes  & \yes  & \meh  & \meh    \\
                        & Early burst   & \no   & \meh  & \no   & \yes    \\
                        & Two-infall    & \meh  & \meh  & \meh  & \meh    \\
\enddata
\end{deluxetable*}

\section{Conclusions}
\label{sec:conclusions}

Blah

\begin{acknowledgments}
Funding for the Sloan Digital Sky 
Survey IV has been provided by the 
Alfred P. Sloan Foundation, the U.S. 
Department of Energy Office of 
Science, and the Participating 
Institutions. 

SDSS-IV acknowledges support and 
resources from the Center for High 
Performance Computing  at the 
University of Utah. The SDSS 
website is www.sdss4.org.

SDSS-IV is managed by the 
Astrophysical Research Consortium 
for the Participating Institutions 
of the SDSS Collaboration including 
the Brazilian Participation Group, 
the Carnegie Institution for Science, 
Carnegie Mellon University, Center for 
Astrophysics | Harvard \& 
Smithsonian, the Chilean Participation 
Group, the French Participation Group, 
Instituto de Astrof\'isica de 
Canarias, The Johns Hopkins 
University, Kavli Institute for the 
Physics and Mathematics of the 
Universe (IPMU) / University of 
Tokyo, the Korean Participation Group, 
Lawrence Berkeley National Laboratory, 
Leibniz Institut f\"ur Astrophysik 
Potsdam (AIP),  Max-Planck-Institut 
f\"ur Astronomie (MPIA Heidelberg), 
Max-Planck-Institut f\"ur 
Astrophysik (MPA Garching), 
Max-Planck-Institut f\"ur 
Extraterrestrische Physik (MPE), 
National Astronomical Observatories of 
China, New Mexico State University, 
New York University, University of 
Notre Dame, Observat\'ario 
Nacional / MCTI, The Ohio State 
University, Pennsylvania State 
University, Shanghai 
Astronomical Observatory, United 
Kingdom Participation Group, 
Universidad Nacional Aut\'onoma 
de M\'exico, University of Arizona, 
University of Colorado Boulder, 
University of Oxford, University of 
Portsmouth, University of Utah, 
University of Virginia, University 
of Washington, University of 
Wisconsin, Vanderbilt University, 
and Yale University.
\end{acknowledgments}

\appendix

\bibliography{bib}

\end{document}
