% Define document class
% \documentclass[modern,linenumbers]{aastex631}
\documentclass[twocolumn,twocolappendix,linenumbers,trackchanges]{aastex631}

\usepackage{showyourwork}
\usepackage{amsmath}
\usepackage{amssymb}
\usepackage{graphicx}
% \usepackage{layouts}
\usepackage{xcolor}
\usepackage{upgreek}
\let\tablenum\relax
\usepackage{siunitx}
\usepackage{xspace}

% user-defined commands
\newcommand{\yes}{\textcolor{green}{\checkmark}\xspace}
\newcommand{\meh}{\textcolor{black}{$\sim$}\xspace}
\newcommand{\no}{\textcolor{red}{$\times$}\xspace}
\newcommand{\osuaffil}{Department of Astronomy, The Ohio State University, 140 W. 18th Ave, Columbus OH 43210, USA}
\newcommand{\ccappaffil}{Center for Cosmology and AstroParticle Physics, The Ohio State University, 191 W. Woodruff Ave., Columbus OH 43210, USA}
\newcommand{\aFe}{[$\alpha$/Fe]\xspace}
\newcommand{\vice}{{\tt VICE}\xspace}

% Citation aliases
\defcitealias{Johnson2021-Migration}{J21}
\defcitealias{Leung2023-Ages}{L23}

\shorttitle{SN Ia DTD in GCE Models}
\shortauthors{Dubay, Johnson, \& Johnson}
% \linespread{1.8}

\begin{document}

% Title
% \title{The Galactic Delay-Time Distribution of Type Ia Supernovae:\\
%        A Chemical Evolution Perspective}
\title{Galactic Chemical Evolution Models Favor an Extended Type Ia Supernova Delay-Time Distribution}

% Author list
\author[0000-0003-3781-0747]{Liam O. Dubay}
\affiliation{\osuaffil}
\affiliation{\ccappaffil}
\author[0000-0001-7258-1834]{Jennifer A. Johnson}
\affiliation{\osuaffil}
\affiliation{\ccappaffil}
\author[0000-0002-6534-8783]{James W. Johnson}
\affiliation{Observatories of the Carnegie Institution for Science, 813 Santa Barbara St., Pasadena CA 91101, USA}
\affiliation{\osuaffil}
\affiliation{\ccappaffil}
% \author[0000-0001-7775-7261]{David H. Weinberg}

% Abstract from AAS 241
\begin{abstract}
    Type Ia supernovae (SNe Ia) produce most of the Fe-peak elements in the Universe and therefore are a crucial ingredient in galactic chemical evolution models. SNe Ia do not explode immediately after star formation and the delay-time distribution (DTD) has not been definitively determined by supernova surveys or theoretical models. Because the DTD also affects the relationship among age, [Fe/H], and \aFe in chemical evolution models, comparison with observations of stars in the Milky Way is an important consistency check for any proposed DTD. We implement several popular forms of the DTD in combination with multiple star formation histories for the Milky Way in %one-zone and 
    multi-zone chemical evolution models including a prescription for radial stellar migration.
    % using the Versatile Integrator for Chemical Evolution (\vice). \vice includes a prescription for radial stellar migration in multi-zone models. 
    We compare our predicted interstellar medium abundance tracks, stellar abundance distributions, and stellar age distributions to the final data release of the Apache Point Observatory Galactic Evolution Experiment (APOGEE). We find that the SN Ia DTD has the largest effect on the \aFe distribution: a DTD with more prompt SNe Ia produces a stellar abundance distribution that is skewed toward a lower \aFe ratio, whereas a DTD with more delayed SNe Ia produces a distribution that is skewed toward higher \aFe. While the DTD alone cannot explain the observed bimodality in the \aFe distribution, in combination with an appropriate star formation history it affects the goodness of fit between the simulated and observed high-$\alpha$ sequence. Our model results favor a DTD with longer delay times and fewer prompt SNe Ia than the fiducial $t^{-1}$ power law.
\end{abstract}

\section{Introduction}

% [Hook and intro paragraph]
Galactic chemical evolution (GCE) studies seek to explain the observed distribution of metals throughout the Milky Way Galaxy. \citet{Tinsley1979-StellarLifetimes} made a compelling case that the non-solar \aFe\footnote{
    In standard bracket notation, $[X/Y]\equiv \log_{10}(X/Y) - \log_{10}(X/Y)_{\odot}$. In this work we will use \aFe and [O/Fe] interchangeably, although observational studies will often use a combination of $\alpha$-elements to calculate \aFe.
} ratios seen by, e.g., \citet{Wallerstein1962-GDwarfAbundances} were caused by different stellar lifetimes for the contributors of the Fe-peak elements than the $\alpha$-elements. 
Type Ia supernovae (SNe Ia), the thermonuclear explosions of carbon-oxygen white dwarfs (WDs), are responsible for a majority of the Fe produced in the Galaxy \citep{Matteucci1986-SupernovaEnrichment}; meanwhile, core collapse supernovae (CCSNe), the explosions of massive stars, produce the $\alpha$-elements (e.g., O and Mg) in addition to a smaller fraction of Fe.
SNe Ia are delayed by $\sim0.1-10$ Gyr after star formation events, as evidenced by observations at large distances from star forming regions and in elliptical galaxies \citep[e.g.,][]{Maza1976-SNStatistics}.
This delayed enrichment leads to a decrease in \aFe with increasing [Fe/H] \citep{Matteucci1986-SupernovaEnrichment}.
% The decrease in \aFe with [Fe/H] has long been understood to be a consequence of the delayed nature of SNe Ia, which do not contribute substantially to chemical enrichment for $\sim 10^8$ years after the commencement of star formation \citep{Tinsley1979-StellarLifetimes,Matteucci1986-SupernovaEnrichment}. 
% As SNe Ia are a product of intermediate-mass stars which undergo binary evolution, they do not contribute substantially to chemical enrichment for $\sim 10^8$ years after the commencement of star formation \citep{Matteucci1986-SupernovaEnrichment}.
% The timescale for Fe enrichment in the Galaxy, a key component of chemical evolution models, is directly linked to the progenitor lifetimes of SNe Ia. 
Therefore, the relative abundances of the $\alpha$-elements and Fe as a function of stellar age trace the balance of SN rates over time.

% [What is the delay time distribution]
The {\it delay time distribution} (DTD) refers to the rate of SN Ia events per unit mass of star formation as a function of stellar population age
% The rate of SNe Ia which explode as a function of time after the formation of a hypothetical unit-mass single stellar population is known as the delay-time distribution 
\citep[for a review, see Section 3.5 of][]{Maoz2014-Review}.
When the DTD is convolved with the Galactic star formation rate (SFR), it yields the overall SN Ia rate. 
The quantitative details of the relationship between \aFe and [Fe/H] are set by the DTD, and as such it is a key parameter in GCE models.
However, the DTD remains poorly constrained because it reflects the detailed evolution of the SN Ia progenitor systems, so different models for the progenitors of SNe Ia will naturally predict different forms for the DTD.

% [Introduction to SNe Ia]
The explosion mechanism(s) of SNe Ia are not fully understood \citep[for reviews, see][]{Maoz2014-Review,Livio2018-ProgenitorReview,Ruiter2020-ProgenitorReview}. Two general production channels have been proposed. In the single-degenerate (SD) case, the WD accretes mass from a close non-degenerate companion until it surpasses $\sim1.4$ M$_\odot$ and explodes \citep{Whelan1973-SDModel,Nomoto1982-SDModel,Yoon2003-SDModel}. In the double-degenerate (DD) case, two WDs merge after a gravitational-wave inspiral \citep{Iben1984-IaBinary,Webbink1984-DDModel,Pakmor2012-WDMerger} or head-on collision \citep{Benz1989-CollisionalDD,Thompson2011-CollisionalDD}. Searches for signs of interaction between the SN ejecta and a non-degenerate companion \citep[e.g.,][]{Panagia2006-RadioEmission,Chomiuk2016-RadioEmission,Fausnaugh2019-EarlyIaLightCurves,Tucker2020-SNeIaSpectra,Dubay2022-SNeIaCSM} or for a surviving companion \citep[e.g.,][]{Schaefer2012-ExCompanionSNR,Do2021-Progenitor1972E,Tucker2023-SN2011fe} have placed tight constraints on the SD channel, heavily disfavoring it as the main pathway for producing ``normal'' SNe Ia. The DD channel, now the preferred model, faces issues of its own with matching observed SN Ia rates because not all WD mergers may lead to a thermonuclear explosion \citep[e.g.,][]{NomotoIben1985-DDMergers,SaioNomoto1998-DDMergers,Shen2012-DDMergers}, and the progenitor systems are difficult to detect even within our own Galaxy \citep{RebassaMansergas2019-WhereAreDDProgenitors}.

% [Theoretical DTDs]
% Uncertainties in the progenitor channels of SNe Ia translate into uncertainties in the shape of the DTD. 
% By formulating analytical DTDs for several progenitor models, \citet{Greggio2005-AnalyticalRates} found that the SD channel tends to produce a steeper DTD function than the DD channel, and the median delay time (by which 50\% of SNe Ia have exploded) can vary by a factor of 10 depending on the chosen progenitor scenario and its particular parameters.
Theoretical models have yet to converge on a single prediction for the DTD.
For the DD channel, assumptions about the distribution of post-common envelope separations and the rate of gravitational wave inspiral suggest a broad $\sim t^{-1}$ DTD at long delay times ($\gtrsim 1$ Gyr), but at short delays ($\lesssim 1$ Gyr) the rate is limited by the need to produce two WDs \citep[see Section 3.5 from][]{Maoz2014-Review}.
Higher-order progenitor systems could also produce a $t^{-1}$ DTD \citep{Fang2018-QuadrupleSystems,Rajamuthukumar2023-TripleEvolution}.
The DTD which would result from the SD channel depends greatly on the assumptions of binary population synthesis, but in general is expected to cover a narrower range of delay times and may feature a steep exponential cutoff at the long end \citep[e.g.,][]{Greggio2005-AnalyticalRates}.
% More recently, \citet{Rajamuthukumar2023-TripleEvolution} computed a DTD from the evolution of triple system progenitors. [Expand, connect binary evolution to delay times, look for other theory papers]

% [Observational DTDs]
Surveys of SNe Ia 
% have provided the primary observational constraints on the DTD. This can be done 
can constrain the DTD
by comparing the observed rate of SNe Ia to their host galaxy parameters \citep[e.g.,][]{Mannucci2005-SNRate,Heringer2019-FieldGalaxyDTD} or inferred star formation histories \citep[SFHs; e.g.,][]{Maoz2012-SloanIIDTD}, measuring SN Ia rates in galaxy clusters \citep[e.g.,][]{Maoz2010-ClusterDTD}, or comparing the volumetric SN Ia rate to the cosmic SFH as a whole \citep[e.g.,][]{Graur2014-VolumetricSNIaRates,Strolger2020-ExponentialDTD}. 
Early studies, which had limited sample sizes, produced unimodal \citep{Strolger2004-SNIaProgenitors} or bimodal \citep{Mannucci2006-TwoPopulations} DTDs where the majority of SNe Ia explode within a relatively narrow range of delay times.
More recent studies have recovered broader DTD functions, with many converging on a declining power-law of $\sim t^{-1}$ \citep[e.g.,][]{Maoz2017-CosmicDTD,Castrillo2021-DTD,Wiseman2021-DESRates}, though there is some evidence for a steeper slope in galaxy clusters \citep{Maoz2017-CosmicDTD,FriedmannMaoz2018-ClusterDTD}.
It is especially difficult to constrain the DTD for short delay times \citep{MaozMannucci2012-SNeIaReview} because of the need for SN Ia rates at long lookback times and uncertainties in the age estimates of stellar populations.

% Notes on observational DTDs:
% \citet{Strolger2004-SNIaProgenitors} analyzed 42 SNe Ia with redshifts $0.2<z<1.6$ and favored a delayed ($\gtrsim2$ Gyr) narrow Gaussian DTD while disfavoring exponentially declining models with a large fraction of prompt SNe Ia.
% \citet{Mannucci2005-SNRate} found that the SN Ia rate per unit mass has a strong dependence on host galaxy morphology and color. Using these results, \citet{Mannucci2006-TwoPopulations} obtained a best-fit model for the DTD which consists of both a prompt component ($<100$ Myr) and a long exponential tail. 
% The Maoz studies all assume a particular form for the DTD and fit for the specific parameters. They do not quantitatively favor one form over another (e.g., power-law over exponential). 
% \citet{Castrillo2021-DTD} find a best fit power law slope of $\alpha\approx-1.1$ with a minimum delay time of $\sim50$ Myr, they also find a fairly high correlation between SN Ia rate and H$\alpha$ (a proxy for young stars) which is surprising.
% \citet{Stolger2020-ExponentialDTD} perform maximum likelihood estimations based on the cosmic star formation history and the star formation histories of individual galaxies. In both cases their best fit functions were approximately exponential, though they started with a skew-normal function so could not have ended up with a power law even if they wanted to. They don't actually report the timescale of this exponential, but based on the plots I'd say it's about 1.5 - 2 (1.7?) Gyr. Their best fit parameters make no sense to me as they don't report their units.

The uncertainties in the SN Ia DTD propagate into GCE models. In principle, the observed chemical abundance patterns should therefore contain information about the SN Ia DTD, and by extension their progenitors.
A few studies have compared different DTDs in one-zone chemical evolution models, but their comparisons to abundance data are limited to the local solar neighborhood \citep[e.g.,][]{Andrews2017-ChemicalEvolution,Palicio2023-AnalyticDTD}. 
[Talk about \citep{Matteucci2009-DTDModels} multi-zone models].
\citet{Poulhazan2018-PrecisionPollution} compared six DTDs in a cosmological smoothed-particle hydrodynamics simulation. They found that models with a prompt SN Ia component produced the highest abundance ratios and the narrowest \aFe distributions at late times, although their simulation was not tuned to reproduce the parameters of the Milky Way, and they did not compare to observed chemical abundances. 
The current era of large spectroscopic surveys such as the Apache Point Observatory Galactic Evolution Experiment \citep[APOGEE;][]{Majewski2017-APOGEE} and the forthcoming Milky Way Mapper \citep{Kollmeier2017-SDSS-V} has made abundances across the Milky Way disk available for comparison to more sophisticated GCE models.

% Notes on GCE studies
% \citet{Andrews2017-ChemicalEvolution} predict abundance tracks and distributions using one-zone models for four different DTDs; they find that most are consistent with their abundance data.
% \citet{Palicio2023-AnalyticDTD} derive analytic solutions to one-zone chemical evolution equations for seven DTDs, but their comparisons to data are limited to the solar neighborhood. 
% \citet{Poulhazan2018-PrecisionPollution} investigate six different DTDs in a smoothed-particle hydrodynamics simulation, but they only present global results for the entire galaxy, they don't compare abundances throughout the disk, and they don't compare to observations.

This paper presents the first comprehensive look at the DTD in a multi-zone GCE model that can qualitatively reproduce the observed abundance structure of the Milky Way disk. A multi-zone approach allows for a radially-dependent parameterization of the SFH, outflows, stellar migration, and abundance gradient which can better match observations across the Galactic disk. We evaluate a selection of DTDs from the literature with multiple SFHs and a prescription for radial stellar migration in the Versatile Integrator for Chemical Evolution \citep[\vice;][]{JohnsonWeinberg2020-Starbursts}. In Section \ref{sec:methods}, we present our models for the DTD and SFH. In Section \ref{sec:onezone-results}, we detail our one-zone chemical evolution models and present results. In Section \ref{sec:multizone-results}, we present the results of our multi-zone models and compare to observations. In Section \ref{sec:discussion}, we discuss the implications for the DTD and future surveys. In Section \ref{sec:conclusions}, we summarize our conclusions.

\section{Methods}
\label{sec:methods}

We use \vice to run chemical evolution models which closely follow those of \citet{JohnsonWeinberg2020-Starbursts} and \citet[][hereafter \citetalias{Johnson2021-Migration}]{Johnson2021-Migration}. We refer the interested reader to the former for details about the \vice package and to the latter for details about the model Milky Way disk, including the star formation law, radial density gradient, and outflows. Similar to \citetalias{Johnson2021-Migration}, we adopt a prescription for radial migration based on the {\tt h277} hydrodynamical simulation \citep{Christensen2012-h277}. Appendix \ref{app:migration} describes our method for determining the migration distance $\Delta R_{\rm gal}$ for each model stellar population, which produces smoother distributions in chemical abundance space than the simulation-based approach.

\subsection{Nucleosynthetic Yields}
\label{sec:yields}

For simplicity and easier comparison to the results of \citetalias{Johnson2021-Migration}, we focus our analysis on O and Fe, representing the $\alpha$ and Fe-peak elements, respectively. Both elements are produced by CCSNe. \vice adopts the instantaneous recycling approximation for CCSNe, so the equation which governs CCSN enrichment as a function of star formation for some element $x$ is simply
\begin{equation}
    \dot M_x^{\rm CC} = y_x^{\rm CC} \dot M_*
    \label{eq:ccsn-enrichment}
\end{equation}
where $y_x^{\rm CC}$ is the CCSN yield of element $x$ per unit mass of star formation, and $\dot M_*$ is the SFR. 
Following \citetalias{Johnson2021-Migration}, who in turn adopt their CCSN yields from \citet{ChieffiLimongi2004-CCSNYields} and \citet{LimongiChieffi2006-CCSNYields}, we adopt $y_{\rm O}^{\rm CC}=0.015$ and $y_{\rm Fe}^{\rm CC}=0.0012$. The primary effect of these yields is to set the low-[Fe/H] ``plateau'' in [O/Fe] which represents pure CCSN enrichment. The chosen yields for this paper produce a plateau at ${\rm [O/Fe]}=0.45$; see \citet{Weinberg2023-CCSNYield} for more discussion on the effect of the CCSN yields on chemical evolution.

Following the formalism of \citet{Weinberg2017-ChemicalEquilibrium}, the rate of Fe contribution to the ISM from SNe Ia is 
\begin{equation}
    \dot M_{\rm Fe}^{\rm Ia} = y_{\rm Fe}^{\rm Ia} \langle \dot M_*\rangle_{\rm Ia} \\
\end{equation}
where $\langle \dot M_*\rangle_{\rm Ia}$ is the time-averaged SFR weighted by the DTD and $y_{\rm Fe}^{\rm Ia}$ is the Fe yield of SNe Ia. \citet{Weinberg2017-ChemicalEquilibrium} show in their Appendix A that
\begin{equation}
    \langle \dot M_*\rangle_{\rm Ia} \equiv \frac{\int_0^t \dot M_*(t')R_{\rm Ia}(t-t')dt'}{\int_{t_D}^{t_{\rm max}} R_{\rm Ia}(t')dt'}
    \label{eq:weighted-sfr}
\end{equation}
where %$\dot M_*(t)$ is the SFR at time $t$, 
%$R_{\rm Ia}(t) dt$ is the number of SNe Ia which explode in the time interval $(t, t+dt)$ per unit mass of stars formed at $t=0$, 
$R_{\rm Ia}$ is the DTD in units of ${\rm M}_\odot^{-1}\,{\rm yr}^{-1}$,
$t_D$ is the minimum SN Ia delay time, and $t_{\rm max}$ is the maximum simulation time. The denominator of Equation \ref{eq:weighted-sfr} is therefore the total number of SNe Ia per M$_{\odot}$ of stars formed $N_{\rm Ia}/M_*$.

The yield $y_{\rm Fe}^{\rm Ia}$ measures the mass of Fe produced by SNe Ia over the full duration of the DTD, which can be expressed as:
\begin{equation}
    y_{\rm Fe}^{\rm Ia} = m_{\rm Fe}^{\rm Ia} \int_{t_D}^{t_{\rm max}} R_{\rm Ia}(t')dt' = m_{\rm Fe}^{\rm Ia} \frac{N_{\rm Ia}}{M_*},
    \label{eq:dtd-integral}
\end{equation}
where $m_{\rm Fe}^{\rm Ia}$ is the average mass of Fe produced by a single SN Ia, and $N_{\rm Ia}/M_*=2.2\pm1\times10^{-3}\,{\rm M}_\odot^{-1}$ is the average number of SNe Ia per mass of stars formed \citep{MaozMannucci2012-SNeIaReview}. 
Adjusting the value of $y_{\rm Fe}^{\rm Ia}$ primarily affects the end point of chemical evolution tracks in [O/Fe]--[Fe/H] space. 
Following \citetalias{Johnson2021-Migration}, we adopt $y_{\rm Fe}^{\rm Ia}=0.00214$. This yield is originally adapted from the W70 model of \citet{Iwamoto1999-SNIaYields}, but it is increased slightly so that the inside-out SFH produces stars with ${\rm [O/Fe]}\approx 0.0$ by the end of the model.
\citet{Palla2021-SNIaYield} studied the effect of different SN Ia yields on GCE models in detail.

\subsection{Delay Time Distributions}
\label{sec:dtd-models}

We explore five different functional forms for the DTD: a single power-law, a broken power-law with an initially flat plateau, a simple exponential, a two-population model, and a model based on triple-system dynamics. We also investigate one or two useful variations of the input parameters for each functional form. Figure \ref{fig:dtds} presents a selection of these DTDs, and Table \ref{tab:dtds} summarizes all of our DTDs and their parameters. We use simple forms rather than simulated physical or analytical models of SNe Ia for the sake of decreased computational time and easier interpretation of the model predictions. Physically-motivated models of the DTD must contend with many unknown or poorly-constrained parameters, so our simplified forms have the advantage of reducing the number of free parameters. In Appendix \ref{app:analytical-dtds}, we show that a few of our simple forms adequately approximate the more complete analytic models of \citet{Greggio2005-AnalyticalRates}. 

In this subsection we present functional forms $f_{\rm Ia}$ describing the shape of each DTD. $f_{\rm Ia}$ does not include the normalization, so it has units of Gyr$^{-1}$ as opposed to M$_\odot^{-1}$ Gyr$^{-1}$ and ist related to $R_{\rm Ia}$ according to
% The relationship between $f_{\rm Ia}$, which has units of ${\rm Gyr}^{-1}$, and $R_{\rm Ia}$, which has units of ${\rm M}_\odot^{-1}\,{\rm Gyr}^{-1}$, is: 
\begin{equation}
    R_{\rm Ia}(t) = 
    \begin{cases}
        \frac{N_{\rm Ia}}{M_*}
        \frac{f_{\rm Ia}(t)}{\int_{t_D}^{t_{\rm max}} f_{\rm Ia}(t') dt'}, & t \ge t_D \\
        0 & t < t_D.
    \end{cases}
    \label{eq:dtd-function}
\end{equation}
\vice automatically normalizes the provided function for the DTD according to this equation.
In the remainder of this subsection, we discuss each form of $f_{\rm Ia}$ individually.

\begin{figure}
    \centering
    \includegraphics{figures/delay_time_distributions.pdf}
    \caption{Selection of models for the SN Ia delay time distribution (DTD) used in this paper. All functions are normalized such that $f_{\rm Ia}(t=1\,{\rm Gyr})=1$. The black squares represent the DTD recovered for the SDSS-II sample of SNe Ia by \citet{Maoz2012-SloanIIDTD} at the same scale as the DTDs. The horizontal and vertical error bars indicate the time range and 1$\sigma$ uncertainties, respectively, of each DTD measurement. The colored circles along the horizontal axis indicate the median delay time for each model.}
    \label{fig:dtds}
    \script{delay_time_distributions.py}
\end{figure}

% \begin{deluxetable*}{lcll}
% \tablecaption{Summary of SN Ia DTDs explored in this paper.\label{tab:dtds}}
% \tablehead{
% \colhead{Model} & \colhead{Eq.} & \colhead{Parameters} & \colhead{Similar to}
% }
% \startdata
\begin{table*}
    \centering
    \caption{Summary of SN Ia delay time distributions (DTDs) explored in this paper.\label{tab:dtds}}
    \begin{tabular}{lcll}
        \hline\hline
        Model & Eq. & Parameters & Similar to \\
        \hline
        Two-population  & \ref{eq:prompt-dtd}   & $t_{\rm max}=0.05$ Gyr, $\sigma=0.015$ Gyr, & \citet{Mannucci2006-TwoPopulations} \\
                        &                       & $\tau=3.0$ Gyr & \\
        Power-law   & \ref{eq:powerlaw-dtd} & $\alpha=-1.4$                 & \citet[][cluster]{Maoz2017-CosmicDTD}; 
                                                      \citet{Heringer2019-FieldGalaxyDTD}       \\
        Power-law   & \ref{eq:powerlaw-dtd} & $\alpha=-1.1$                 & \citet[][field]{Maoz2017-CosmicDTD}; 
                                                      \citet{Wiseman2021-DESRates}              \\
        Exponential & \ref{eq:exponential-dtd}  & $\tau=1.5$ Gyr    & \citet[][SD]{Greggio2005-AnalyticalRates};
                                                                      \citet{Schonrich2009-RadialMixing};       \\
                    &                           &                   & \citet{Weinberg2017-ChemicalEquilibrium}  \\
        Exponential & \ref{eq:exponential-dtd}   & $\tau=3.0$ Gyr    & --- \\
        Plateau     & \ref{eq:plateau-dtd}  & $W=0.3$ Gyr, $\alpha=-1.1$    & \citet[][CLOSE DD]{Greggio2005-AnalyticalRates} \\
        Plateau     & \ref{eq:plateau-dtd}  & $W=1.0$ Gyr, $\alpha=-1.1$    & \citet[][WIDE DD]{Greggio2005-AnalyticalRates} \\
        Triple-system   & \ref{eq:triple-dtd}   & $f_{\rm init}=0.05f_{\rm peak}$, $t_{\rm rise}=0.5$ Gyr, & \citet{Rajamuthukumar2023-TripleEvolution} \\
                        &                       & $W=0.5$ Gyr, $\alpha=-1.1$ & \\
        \hline
    \end{tabular}
\end{table*}
% \enddata
% \end{deluxetable*}

\paragraph{Two-population} A DTD in which $\sim50\%$ of SNe Ia belong to a ``prompt'' Gaussian component at small $t$ and the remainder form an exponential tail at large $t$:
\begin{equation}
    f_{\rm Ia}^{\rm twopop}(t) = 0.5\Big[e^{-\frac{(t-t_p)^2}{2\sigma^2}} + e^{-t/\tau}\Big]
    \label{eq:prompt-dtd}
\end{equation}
To approximate the DTD from \citet{Mannucci2006-TwoPopulations}, we take $t_p=50$ Myr, $\sigma=15$ Myr, and $\tau=3$ Gyr, which results in $\sim 50\%$ of SNe Ia exploding within $t<100$ Myr. As Figure \ref{fig:dtds} illustrates, the two-population DTD has a shorter median delay time than most other models (except the power-law with $\alpha=-1.4$, not shown).
This formulation is slightly different than the approximation used in other GCE studies \citep[e.g.,][]{Matteucci2006-BimodalDTDConsequences,Poulhazan2018-PrecisionPollution}, where it has a more distinctly bimodal shape. We have compared the two approximations to this DTD in a one-zone model and found that they produce very similar abundance tracks and MDFs. This DTD is adopted by the Feedback In Realistic Environments \citep[FIRE;][]{Hopkins2014-FIRE-1} and FIRE-2 \citep{Hopkins2018-FIRE-2} simulations.

\paragraph{Power-law} A single power law with slope $\alpha$:
\begin{equation}
    f_{\rm Ia}^{\rm plaw}(t) = (t/1\,\rm{Gyr})^\alpha
    \label{eq:powerlaw-dtd}
\end{equation}
% where $A_\alpha = (\alpha+1)(t_{\rm max}^{\alpha+1} - t_{\rm min}^{\alpha+1})^{-1}$ is the normalization coefficient.
A declining power-law with $\alpha\sim-1$ is expected to arise from typical assumptions about the distribution of post-common envelope separations and the rate of gravitational wave inspiral \citep[see Section 3.5 from][]{Maoz2014-Review}. It is therefore a commonly assumed DTD in GCE studies (e.g., \citealt{Rybizki2017-Chempy}; \citetalias{Johnson2021-Migration}; \citealt{Weinberg2023-CCSNYield}). Additionally, the observational evidence for a power-law DTD is strong. \citet{Maoz2017-CosmicDTD} obtain a DTD with $\alpha=-1.07\pm0.09$ based on volumetric rates and an assumed cosmic SFH for field galaxies in redshift range $0\leq z\leq 2.25$. \citet{Wiseman2021-DESRates} obtain a similar slope of $\alpha=-1.13\pm0.05$ for field galaxies in the redshift range $0.2<z<0.6$. For galaxy clusters, \citet{Maoz2017-CosmicDTD} find a steeper DTD slope of $\alpha=-1.39^{+0.32}_{-0.05}$. \citet{Heringer2019-FieldGalaxyDTD} use a SFH-independent method to constrain the DTD for field galaxies within $0.01<z<0.2$ and find a steeper slope of $\alpha=-1.34^{+0.19}_{-0.17}$.
In this paper, we investigate the cases $\alpha=-1.1$ and $\alpha=-1.4$.

\paragraph{Exponential} An exponentially declining DTD with timescale $\tau$:
\begin{equation}
    f_{\rm Ia}^{\rm exp}(t) = e^{-t/\tau}
    \label{eq:exponential-dtd}
\end{equation}
This model allows analytic solutions to the abundances as a function of time for some SFHs, making it a popular choice. \citet{Schonrich2009-RadialMixing} and \citet{Weinberg2017-ChemicalEquilibrium} both assume an exponential DTD with a timescale $\tau=1.5$ Gyr. We show in Appendix \ref{app:analytical-dtds} that a $\tau=1.5$ Gyr exponential is an adequate approximation for the analytical SD DTD from \citet{Greggio2005-AnalyticalRates}. 
There is less observational support for an exponential DTD. \citet{Strolger2020-ExponentialDTD}, fitting to the cosmic SFH and SFHs from field galaxies, find a range of exponential-like solutions with timescales $\sim 1.5 - 6$ Gyr.
In this paper, we investigate timescales $\tau=1.5$ and 3 Gyr.

\paragraph{Plateau} A modification of the power-law in which the DTD ``plateaus'' for a duration $W$ before declining:
\begin{equation}
    f_{\rm Ia}^{\rm plat}(t) =
    \begin{cases}
        1, & t < W \\
        (t/W)^\alpha, & t \ge W
    \end{cases}
    \label{eq:plateau-dtd}
\end{equation}
% where $B_W=[(W-t_{\rm min}) + \frac{t_{\rm max}^{\alpha+1} - W^{\alpha+1}}{(\alpha+1)W^\alpha}]^{-1}$ is the normalization coefficient.
Our primary motivation is to consider a model which matches observations at delay times beyond a few Gyr, where the DTD is best constrained, but with a smaller fraction of prompt ($\lesssim 100$ Myr) SNe Ia than the single power law.
To our knowledge, this form of the DTD has not been considered in previous GCE models. We show in Appendix \ref{app:analytical-dtds} that this form can approximate the more complicated analytical DD DTDs from \citet{Greggio2005-AnalyticalRates}. We investigate the cases $W=0.3$ Gyr and $W=1$ Gyr, taking $\alpha=-1.1$ for all plateau models.

\paragraph{Triple-system} A DTD based on simulations of triple-system evolution by \citet{Rajamuthukumar2023-TripleEvolution}. We approximate their numerically-generated DTD as a special case of the plateau model (Equation \ref{eq:plateau-dtd}) where the initial rate is quite low until an instantaneous rise to the plateau value at time $t_{\rm rise}$:
\begin{equation}
    f_{\rm Ia}^{\rm triple}(t) =
    \begin{cases}
        \epsilon, & t < t_{\rm rise} \\
        1, & t_{\rm rise} \leq t < W \\
        (t / W) ^ \alpha, & t \geq W
    \end{cases}
    \label{eq:triple-dtd}
\end{equation}
with $t_{\rm rise}=0.5$ Gyr, $W=0.5$ Gyr, $\alpha=-1.1$, and $\epsilon=0.05$ (i.e., the initial rate is 5\% of the peak rate). As illustrated in Figure \ref{fig:dtds}, the triple-system DTD has the longest median delay time out of all the models we investigate.

\subsubsection{The Minimum SN Ia Delay Time}

In addition to the DTD shape, the minimum SN Ia delay time $t_D$ is another parameter that can have an effect on chemical evolution observables, such as the location of the high-$\alpha$ knee and the [O/Fe] distribution function \citep[DF;][]{Andrews2017-ChemicalEvolution}. The value of $t_D$ is set by the lifetime of the most massive SN Ia progenitor system. Previous GCE studies have adopted values ranging from $t_D\approx30$ Myr \citep[e.g.,][]{Poulhazan2018-PrecisionPollution} to $t_D=150$ Myr \citepalias[e.g.,][]{Johnson2021-Migration}. We take $t_D=40$ Myr as our fiducial value as it is the approximate lifetime of an $8\,{\rm M}_\odot$ star. In Section \ref{sec:onezone-results} we find that adopting a longer $t_D$ has only a minor effect on the chemical evolution for most DTDs except the power-law, but in that case the effect of a longer $t_D$ can be approximated by adding an initial plateau of width $W=0.3$ Gyr to the DTD (see Figure \ref{fig:onezone-twopanel}).

\subsection{Star Formation Histories}
\label{sec:sfh}

We consider four models for the SFH, which we refer to as inside-out, late-burst, early-burst, and two-infall. 
The former two models, which feature a smooth SFH, were investigated by \citetalias{Johnson2021-Migration} using a similar methodology to this paper. The inside-out model produced a better agreement to the age--[O/Fe] relation observed by \citet{Feuillet2019-MilkyWayAges}, while the late-burst model better matched their observed age--metallicity relation. 
% The latter two models (early-burst and two-infall) feature discontinuous or ``bursty'' SFHs, and were proposed by \citet{Conroy2022-ThickDisk} and \citet{Chiappini1997-TwoInfall}, respectively, to explain the distinct evolution of the high- and low-$\alpha$ sequences. 
The latter two models feature discontinuous or ``bursty'' SFHs. The early-burst model, proposed by \citet{Conroy2022-ThickDisk}, uses an efficiency-driven starburst to explain the break in the \aFe trend observed in the H3 survey \citep{Conroy2019-H3Survey}.
% , and [cite] found that it can also produce a bimodal \aFe distribution.
The two-infall model was proposed by \citet{Chiappini1997-TwoInfall} and features two distinct episodes of gas infall which produce the thick and thin disks.
Together, these four models cover a range of behavior including a smooth SFH and SFR-, SFE-, and infall-driven starbursts.

The inside-out and late-burst models
% The first two models are taken from \citetalias{Johnson2021-Migration} and 
are run in \vice's ``star formation mode,'' where the SFR surface density $\dot\Sigma_*$ is prescribed along with the star formation efficiency (SFE) timescale $\uptau_*\equiv \Sigma_g/\dot\Sigma_*$. The remaining quantities, infall rate (IFR) surface density $\dot\Sigma_{\rm in}$ and gas surface density $\Sigma_g$, are calculated from the specified quantities. The latter two models are run in ``infall mode,'' where we specify $\dot\Sigma_{\rm in}$, $\uptau_*$, and an initial mass of the ISM at the onset of star formation, which we assume to be zero for all models (including those run in SFR mode). The mode in which \vice models are run makes no difference as a unique solution can always be obtained if two of the four parametric forms are specified. 
% See Appendix B of \citetalias{Johnson2021-Migration} for details on the normalization of the SFH. 
The SFH is normalized such that the model predicts an accurate total stellar mass and surface density gradient (see Appendix B of \citetalias{Johnson2021-Migration}).
We present an overview of the four SFHs in Figure \ref{fig:sfhs}, and we discuss them individually here.

\begin{figure*}
    \centering
    \includegraphics[width=\linewidth]{figures/star_formation_histories.pdf}
    \caption{The surface densities of star formation $\dot \Sigma_*$ (first row), gas infall $\dot \Sigma_{\rm in}$ (second row), and gas mass $\Sigma_g$ (third row), and the SFE timescale $\uptau_*$ (fourth row) as functions of simulation time for our four model SFHs: inside-out (first column; see Equation \ref{eq:insideout-sfh}), late-burst (second column; see Equation \ref{eq:lateburst-sfh}), early-burst (third column; see Equations \ref{eq:earlyburst-ifr} and \ref{eq:earlyburst-taustar}), and two-infall (fourth column; see Equation \ref{eq:twoinfall-ifr}). In each panel, we plot curves for the model zones which have inner radii at 4 kpc (yellow), 6 kpc (orange), 8 kpc (red), 10 kpc (violet), 12 kpc (indigo), and 14 kpc (blue).}
    \label{fig:sfhs}
    \script{star_formation_histories.py}
\end{figure*}

\paragraph{Inside-out} As in \citetalias{Johnson2021-Migration}, this is our fiducial SFH. The dimensionless time-dependence of the SFR is given by
\begin{equation}
    f_{\rm IO}(t|R_{\rm gal}) = (1 - e^{-t/\tau_{\rm rise}}) e^{-t/\tau_{\rm sfh}},
    \label{eq:insideout-sfh}
\end{equation}
where we assume $\tau_{\rm rise}=2$ Gyr for all radii. The SFH timescale $\tau_{\rm sfh}$ varies with $R_{\rm gal}$, with $\tau_{\rm sfh}(R_{\rm gal}=8\,\rm{kpc})\approx15$ Gyr at the solar annulus and longer timescales in the outer Galaxy. The $\tau_{\rm sfh} - R_{\rm gal}$ relation is based on the radial gradients in stellar age in Milky Way-like spirals measured by \citet[][see Section 2.5 of \citetalias{Johnson2021-Migration} for details]{Sanchez2020-StarFormationTimescales}.

\paragraph{Late-burst} A variation on the inside-out SFH with a burst in the SFR at late times which is described by a Gaussian according to
\begin{equation}
    f_{\rm LB}(t|R_{\rm gal}) = f_{\rm IO}(t|R_{\rm gal}) \Big(1 + A_b e^{-(t-t_b)^2/2\sigma_b^2} \Big),
    \label{eq:lateburst-sfh}
\end{equation}
where $A_b$ is the dimensionless amplitude of the starburst, $t_b$ is the time of the peak of the burst, and $\sigma_b$ is the width of the Gaussian. 
Evidence for a recent star formation burst $\sim 2-3$ Gyr ago has been found in {\it Gaia} \citep{Mor2019-Starburst} and in massive WDs in the solar neighborhood \citep{Isern2019-Starburst}.
Following \citetalias{Johnson2021-Migration}, we adopt $A_b=1.5$, $t_b=11.2$ Gyr, and $\sigma_b=1$ Gyr. The determination of $\tau_{\rm sfh}$ 
% and the normalization of the SFR as a function of $R_{\rm gal}$ 
is the same as in the inside-out case.

\paragraph{Early-burst} An extension of the model proposed by \citet{Conroy2022-ThickDisk} to explain the non-monotonic behavior of the high-$\alpha$ sequence down to ${\rm [Fe/H]}\approx-2.5$. This model features an abrupt factor $\sim20$ rise in the SFE at early times, driving an increase in the [O/Fe] abundance at the transition between the epochs of halo and thick disk formation. \citet{Stahlholdt2022-StarFormationEpochs} find evidence for a burst $\sim10$ Gyr ago which marks the beginning of a second phase of star formation. [Something about $\alpha$-bimodality from another paper]. We adopt the following formula for the time-dependence of the SFE timescale from \citet{Conroy2022-ThickDisk}:
\begin{equation}
    \frac{\uptau_{\rm EB}}{1\,\rm{Gyr}} =
    \begin{cases}
        50, & t < 2.5\,\rm{Gyr} \\
        \frac{50}{[1+3(t-2.5)]^2}, & 2.5\leq t \leq 3.7\,\rm{Gyr} \\
        2.36, & t > 3.7\,\rm{Gyr}.
    \end{cases}
    \label{eq:earlyburst-taustar}
\end{equation}
While \citet{Conroy2022-ThickDisk} use a constant infall rate in their one-zone model, we adopt a radially-dependent infall rate which declines exponentially with time:
\begin{equation}
    f_{\rm EB}(t|R_{\rm gal}) = e^{-t/\tau_{\rm sfh}},
    \label{eq:earlyburst-ifr}
\end{equation}
where $\tau_{\rm sfh}$ is the same as in the inside-out case.
To calculate $\dot \Sigma_*$ from the above quantities, we modify the fiducial star formation law adopted from \citetalias{Johnson2021-Migration}, substituting $\uptau_{\rm EB}$ for the SFE timescale of molecular gas:
\begin{equation}
    \dot \Sigma_* = 
    \begin{cases}
        \Sigma_g \uptau_{\rm EB}^{-1}, & \Sigma_g \geq \Sigma_{g,2} \\
        \Sigma_g \uptau_{\rm EB}^{-1} \Big(\frac{\Sigma_g}{\Sigma_{g,2}}\Big)^{2.6}, & \Sigma_{g,1} \leq \Sigma_g \leq \Sigma_{g,2} \\
        \Sigma_g \uptau_{\rm EB}^{-1} \Big(\frac{\Sigma_{g,1}}{\Sigma_{g,2}}\Big)^{2.6} \Big(\frac{\Sigma_g}{\Sigma_{g,1}}\Big)^{0.7}, & \Sigma_g \leq \Sigma_{g,1},
    \end{cases}
    \label{eq:earlyburst-sfr}
\end{equation}
with $\Sigma_{g,1}=5\times 10^6\,{\rm M}_\odot\,{\rm kpc}^{-2}$ and $\Sigma_{g,2}=2\times 10^7\,{\rm M}_\odot\,{\rm kpc}^{-2}$.

\paragraph{Two-infall} First proposed by \citet{Chiappini1997-TwoInfall}, this model parameterizes the infall rate as two successive, exponentially declining bursts to explain the origin of the high- and low-$\alpha$ disk populations:
\begin{equation}
    \label{eq:twoinfall-ifr}
    f_{\rm TI}(t|R_{\rm gal}) = N_1(R_{\rm gal}) e^{-t/\tau_1} + N_2(R_{\rm gal}) e^{-(t-t_{\rm on})/\tau_2},
\end{equation}
% In this model, the first infall produces the thick disk and the second infall produces the thin disk. 
where $N_1$ and $N_2$ are the normalizations of the first and second infall, respectively, and $t_{\rm on}$ is the onset time of the second infall.
The normalization ratio $N_2/N_1$ is calculated so that the thick-to-thin-disk surface density ratio $f_\Sigma(R)=\Sigma_2(R)/\Sigma_1(R)$ is given by
\begin{equation}
    f_\Sigma(R) = f_\Sigma(0) e^{R(1/R_2 - 1/R_1)}.
\end{equation}
Following \citet{BlandHawthornGerhard2016-MilkyWayReview}, we adopt values for the thick disk scale radius $R_1=2.0$ kpc, thin disk scale radius $R_2=2.5$ kpc, and $f_\Sigma(0)=0.27$.
We note that most previous studies which use the two-infall model \citep[e.g.,][]{Spitoni2021-TwoInfall} do not consider gas outflows and instead adjust the nucleosynthetic yields to reproduce the solar abundance. We adopt radially-dependent outflows as in \citetalias{Johnson2021-Migration} (see their Section 2.4 for details) for all our SFHs, including two-infall. We discuss the implications of this difference in Section \ref{sec:two-infall-discussion}.

\subsection{The Multi-Zone Disk Model}
\label{sec:multizone-methods}

\begin{deluxetable*}{Cccl}
    \tablecaption{A summary of parameters and their fiducial values for our chemical evolution models. We omit some parameters that are unchanged from \citetalias{Johnson2021-Migration}; see their Table 1 for details.\label{tab:multizone-parameters}}
    \tablehead{
        \colhead{Quantity} & \colhead{Fiducial Value(s)} & \colhead{Section} & \colhead{Description}
    }
    \startdata
        % Multi-zone model parameters
        R_{\rm gal}     & [0, 20] kpc   & \ref{sec:multizone-methods} & Galactocentric radius \\
        \delta R_{\rm gal}  & 100 pc    & \ref{sec:multizone-methods} & Width of each concentric ring \\
        \Delta R_{\rm gal}  & N/A       & \ref{app:migration} & Change in orbital radius due to stellar migration \\
        p(\Delta R_{\rm gal}|\tau,R_{\rm form}) & Equation \ref{eq:radial-migration}    & \ref{app:migration} & Probability density function of radial migration distance \\
        z                   & [-3, 3] kpc                & \ref{app:migration} & Distance from Galactic midplane at present day \\
        p(z|\tau,R_{\rm final}) & Equation \ref{eq:sech-pdf}            & \ref{app:migration} & Probability density function of Galactic midplane distance\\
        \Delta t        & 10 Myr    & \ref{sec:multizone-methods} & Time-step size \\
        t_{\rm max}     & 13.2 Gyr  & \ref{sec:multizone-methods} & Maximum simulation time \\
        n               & 8         & \ref{sec:multizone-methods} & Number of stellar populations formed per ring per time-step \\
        R_{\rm SF}      & 15.5 kpc  & \ref{sec:multizone-methods} & Maximum radius of star formation \\
        M_{g,0}   & 0         & \ref{sec:sfh}     & Initial gas mass \\
        \dot M_r    & continuous    & \ref{sec:multizone-methods} & Recycling rate \citep[][Equation 2]{JohnsonWeinberg2020-Starbursts} \\
        % DTD
        R_{\rm Ia}(t)   & Equation \ref{eq:dtd-function}    & \ref{sec:dtd-models}  & Delay time distribution of Type Ia supernovae \\
        t_D             & 40 Myr    & \ref{sec:dtd-models}  & Minimum SN Ia delay time \\
        N_{\rm Ia}/M_*  & $2.2\times10^{-3}$ M$_\odot^{-1}$ & \ref{sec:yields}  & SNe Ia per unit mass of stars formed \citep{MaozMannucci2012-SNeIaReview} \\
        % Nucleosynthetic yields
        y_{\rm O}^{\rm CC}  & 0.015     & \ref{sec:yields}  & CCSN yield of O    \\
        y_{\rm Fe}^{\rm CC} & 0.0012    & \ref{sec:yields}  & CCSN yield of Fe   \\
        y_{\rm O}^{\rm Ia}  & 0         & \ref{sec:yields}  & SN Ia yield of O       \\
        y_{\rm Fe}^{\rm Ia} & 0.00214   & \ref{sec:yields}  & SN Ia yield of Fe \\
        % Star formation histories
        f_{\rm IO}(t|R_{\rm gal})   & Equation \ref{eq:insideout-sfh}   & \ref{sec:sfh} & Time-dependence of the inside-out SFR \\
        f_{\rm LB}(t|R_{\rm gal})   & Equation \ref{eq:lateburst-sfh}   & \ref{sec:sfh} & Time-dependence of the late-burst SFR \\
        \tau_{\rm rise}             & 2 Gyr     & \ref{sec:sfh} & SFR rise timescale for inside-out and early-burst models \\
        \uptau_{\rm EB}(t)          & Equation \ref{eq:earlyburst-taustar}  & \ref{sec:sfh}   & Time-dependence of the early-burst SFE timescale \\
        f_{\rm EB}(t|R_{\rm gal})   & Equation \ref{eq:earlyburst-ifr}  & \ref{sec:sfh} & Time-dependence of the early-burst IFR \\
        f_{\rm TI}(t|R_{\rm gal})   & Equation \ref{eq:twoinfall-ifr}   & \ref{sec:sfh} & Time-dependence of the two-infall IFR \\
        \hline
        % One-zone parameters
        \uptau_*                        & 2 Gyr & \ref{sec:onezone-results} & SFE timescale in one-zone models \\
        \eta(R_{\rm gal}=8\,{\rm kpc})  & 2.15  & \ref{sec:onezone-results} & Outflow mass-loading factor at the solar annulus \\
        \tau_{\rm sfh}(R_{\rm gal}=8\,{\rm kpc})    & 15.1 Gyr  & \ref{sec:sfh} & SFH timescale at the solar annulus \\
    \enddata
\end{deluxetable*}
\vspace{-24pt}

We make use of the multi-zone GCE model tools in \vice developed by \citetalias{Johnson2021-Migration}. The basic setup of our models follows theirs: the model Galaxy is a disk with a radius of 20 kpc and is divided into concentric rings of width $\delta R_{\rm gal}=100$ pc. The gas in each ring is assumed to be instantaneously mixed; while star particles are allowed to migrate radially (see Appendix \ref{app:migration}), there is no exchange of gas between rings. We run our models with a time-step size of $\Delta t=10$ Myr up to a maximum time of $t_{\rm max}=13.2$ Gyr. Following \citetalias{Johnson2021-Migration}, we set \vice to form $n=8$ stellar populations per ring per time-step, and we set a maximum star-formation radius of $R_{\rm SF} = 15.5$ kpc, such that $\dot\Sigma_*=0$ for $R_{\rm gal}>R_{\rm SF}$ (stars are allowed to migrate past $R_{\rm SF}$, they are just not born there). We adopt continuous recycling, which accounts for the time-dependent return of mass from all previous generations of stars \citep[see Equation 2 from][]{JohnsonWeinberg2020-Starbursts}. We summarize these parameters, as well as those discussed in previous sub-sections, in Table \ref{tab:multizone-parameters}.

\subsection{Observational Sample}
\label{sec:observational-sample}

\begin{table*}
    \centering
    \caption{Sample selection parameters and median uncertainties from APOGEE DR17.}
    \label{tab:sample}
    \begin{tabular}{lll}
        \hline\hline
        \multicolumn{1}{c}{Parameter} & \multicolumn{1}{c}{Range or Value} & \multicolumn{1}{c}{Notes} \\
        \hline
        $\log g$            & $1.0 < \log g < 3.8$          & Select giants only \\
        $T_{\rm eff}$       & $3500 < T_{\rm eff} < 5500$ K & Reliable temperature range \\
        $S/N$               & $S/N > 80$                    & Required for accurate stellar parameters \\
        ASPCAPFLAG Bits     & $\notin$ 23                   & Remove stars flagged as bad \\
        EXTRATARG Bits      & $\notin$ 0, 1, 2, 3, or 4     & Select main red star sample only \\
        Age                 & $\sigma_{\rm Age} < 40\%$     & Age uncertainty from \citetalias{Leung2023-Ages} \\
        \hline
        $\sigma({\rm [Fe/H]})$ & $9.2\times10^{-3}$ & Median uncertainty in [Fe/H] \\
        $\sigma({\rm [O/Fe]})$ & $1.8\times10^{-2}$ & Median uncertainty in [O/Fe] \\
        $\sigma($log(Age/Gyr)) & $0.10$ & Median age uncertainty \citepalias{Leung2023-Ages} \\
        \hline
    \end{tabular}
\end{table*}

% \begin{table}
%     \centering
%     \caption{Median parameter uncertainties}
%     \label{tab:uncertainties}
%     \begin{tabular}{ccc}
%         \hline\hline
%         Parameter & Median Uncertainty & Source \\
%         \hline
%         ${\rm [Fe/H]}$ & $9.2\times10^{-3}$ & APOGEE DR17 \\
%         ${\rm [O/Fe]}$ & $1.8\times10^{-2}$ & APOGEE DR17 \\
%         log(Age/Gyr) & $0.10$ & \citetalias{Leung2023-Ages} \\
%         \hline
%     \end{tabular}
% \end{table}

We compare our model results to abundance measurements from the final data release \citep[DR17;][]{Abdurro'uf2022-SDSSIV-DR17} of the Apache Point Observatory Galactic Evolution Experiment \citep[APOGEE;][]{Majewski2017-APOGEE}. APOGEE used infrared spectrographs \citep{Wilson2019-APOGEE-Spectrographs} mounted on two telescopes: the 2.5-meter Sloan Foundation Telescope \citep{Gunn2006-SloanTelescope} at Apache Point Observatory in the Northern Hemisphere, and the Ir{\'e}n{\'e}e DuPont Telescope \citep{BowenVaughan1973-DuPontTelescope} at Las Campanas Observatory in the Southern Hemisphere. After the spectra were passed through the data reduction pipeline \citep{Nidever2015-APOGEE-DataReduction}, the APOGEE Stellar Parameter and Chemical Abundance Pipeline \citep[ASPCAP;][]{Holtzmann2015-ASPCAP,GarciaPerez2016-ASPCAP} extracted chemical abundances using the model grids and interpolation method described by \citet{Jonsson2020-APOGEE-DR16}.

We restrict our sample to red giant branch and red clump stars with high-quality spectra. Table \ref{tab:sample} lists our selection criteria, which largely follow from \citet{Hayden2015-ChemicalCartography}. This produces a final sample of \variable{output/sample_size.txt}stars with [O/Fe] and [Fe/H] abundance measurements. APOGEE stars are matched with their Bailer-Jones photo-geometric distance estimate from \textit{Gaia} Early Data Release 3 \citep{Gaia2016-Mission,Gaia2021-EDR3}, which we use to calculate galactocentric radius $R_{\rm gal}$ and midplane distance $z$.
We use estimated ages from \citet[][hereafter \citetalias{Leung2023-Ages}]{Leung2023-Ages}, who use a variational encoder-decoder network to retrieve age estimates for APOGEE giants without contamination from age-abundance correlations. Importantly, the \citetalias{Leung2023-Ages} ages do not plateau beyond $\sim10$ Gyr as they do in astroNN \citep{Mackereth2019-astroNN-Ages}. We use an age uncertainty cut of 40\% per the recommendations of \citetalias{Leung2023-Ages}, which produces a total sample of \variable{output/age_sample_size.txt}APOGEE stars with age estimates. We note that we use the full sample of \variable{output/sample_size.txt}APOGEE stars unless we explicitly compare to age estimates. Table \ref{tab:sample} also presents the median uncertainties in the data for [Fe/H], [O/Fe], and age.

\section{Results from One-Zone Models}
\label{sec:onezone-results}

% \begin{table}
%     \centering
%     \caption{A summary of parameters and their fiducial values for our one-zone chemical evolution models.\label{tab:onezone-parameters}}
%     \begin{tabular}{ccl}
%         \hline\hline
%         Quantity & Value & \multicolumn{1}{c}{Description} \\
%         \hline
%         % $\Delta t$        & 10 Myr    & Time-step size \\
%         % $t_{\rm max}$     & 13.2 Gyr  & Maximum simulation time \\
%         % $\eta$            & 2.15      & Outflow mass-loading factor \\
%         % $\uptau_*$        & 2.0 Gyr   & SFE timescale \\
%         $\dot M_r$        & continuous    & Recycling rate \\
%         % $t_D$             & 40 Myr    & Minimum SN Ia delay time \\
%         % $M_{g,0}$         & 0         & Initial gas mass \\
%         % $\dot M_*(t)$     & Equation \ref{eq:insideout-sfh}  & Star formation rate \\
%         % $\tau_{\rm rise}$ & 2.0 Gyr     & SFR rise timescale \\
%         % $\tau_{\rm sfh}$  & 15.1 Gyr  & SFR exponential \\
%         \hline
%     \end{tabular}
% \end{table}

% [Intro paragraph, motivation for running one-zone models]
Before running the full multi-zone models, it is useful to observe the effects of the DTD in more idealized conditions. A one-zone model assumes the entire gas reservoir is instantaneously mixed, removing all spatial dependence. This limits the ability to compare to observations across the disk, but it allows the models to be run in a fraction of the time and eliminates the complicating factor of stellar migration. In this section, we compare the results from one-zone models which examine various parameters of the DTD while keeping other parameters fixed.

% All models are run for a duration of $t_{\rm max}=13.2$ Gyr with a time-step of $\Delta t=10$ Myr. 
For consistency, we adopt most of the parameter values from Table \ref{tab:multizone-parameters} for our one-zone models.
We adopt the inside-out SFR (Equation \ref{eq:insideout-sfh}) evaluated at $R_{\rm gal}=8$ kpc (i.e., $\tau_{\rm rise}=2$ Gyr and $\tau_{\rm sfh}=15.1$ Gyr) and an SFE timescale $\uptau_*\equiv M_g/\dot M_*=2$ Gyr. Unless otherwise specified, we adopt an outflow mass-loading factor $\eta\equiv \dot M_{\rm out}/\dot M_*=2.15$ \citepalias[see Equation 8 from][]{Johnson2021-Migration} and a minimum SN Ia delay time $t_D=40$ Myr. 
% A summary of these fiducial parameters is presented in Table \ref{tab:onezone-parameters}.

\begin{figure*}
    \centering
    \includegraphics[width=\linewidth]{figures/onezone_threepanel.pdf}
    \caption{{\it Each panel:} Abundance tracks in the [Fe/H]--[O/Fe] plane for one-zone chemical evolution models which assume the same DTD. The open symbols along each curve mark logarithmic steps in time. The top and right-hand marginal panels present the distribution functions (DFs) of [Fe/H] and [O/Fe], respectively. For display purposes, these distributions are convolved with a Gaussian kernel with a standard deviation of 0.02 dex. 
    \textit{Left:} Three models which assume a power-law DTD with varying slope $\alpha$. 
    % The dotted, dashed, and solid curves represent models with $\alpha=-0.8$, $-1.1$, and $-1.4$, respectively. 
    For reference, the solid gray curve represents an exponential DTD with $\tau=3$ Gyr. 
    \textit{Center:} Three models which assume an exponential DTD with varying timescale $\tau$. 
    % The solid, dashed, and dotted curves represent $\tau=6$, 3, and 1.5 Gyr, respectively.
    \textit{Right:} Three models which assume a plateau DTD with varying plateau width $W$. 
    % The solid, dashed, and dotted green curves represent $W=1$, 0.3, and 0.1 Gyr, respectively. 
    All assume a post-plateau slope of $\alpha=-1.1$. For reference, the solid gray curve represents an exponential DTD with $\tau=3$ Gyr, and the dotted purple curve represents a power-law DTD with $\alpha=-1.1$ and no plateau.}
    \label{fig:onezone-threepanel}
    \script{onezone_threepanel.py}
\end{figure*}

The left-hand panel of Figure \ref{fig:onezone-threepanel} compares the results of three one-zone models that are identical except for the slope of the power-law DTD. A steeper slope produces a sharper ``knee'' and a faster decline in [O/Fe], resulting in a narrower distribution of [O/Fe] around the low-$\alpha$ sequence and a dearth of high-$\alpha$ stars. Conversely, a shallower slope produces a gentler knee and slower decline in [O/Fe], resulting in a broader [O/Fe] DF and a greater number of high-$\alpha$ stars. In all cases the [O/Fe] DF is distinctly unimodal. The [Fe/H] DF is not as strongly affected by the power-law slope: a shallower slope results in only a modest increase in the width of the distribution.

Similar trends can be seen when adjusting the timescale of the exponential DTD, as shown in the middle panel of Figure \ref{fig:onezone-threepanel}. Here, the knee is not a sharp feature associated with the onset of SNe Ia as in the power-law case, but rather a gentle curve in the abundance track around a simulation time of 1 Gyr. A short exponential timescale shifts this knee down to lower [O/Fe] values, while a longer timescale raises it. The effect on the [O/Fe] DF is similar to the power-law slope: a short timescale produces more low-$\alpha$ stars, while a long timescale produces more high-$\alpha$ stars (but still a unimodal distribution). The effect on the [Fe/H] DF is somewhat more pronounced, with longer timescales skewing to lower [Fe/H] values. A doubling of the timescale from 1.5 Gyr to 3 Gyr raises the [O/Fe] abundance ratio at $t=1$ Gyr by $\sim0.05$ dex and at $t=3$ Gyr by $\sim0.1$ dex, but the equilibrium abundance reached at $t=10$ Gyr remains unchanged. For all exponential models, the low-$\alpha$ peak sits approximately 0.02 dex lower than for the power-law models.

Finally, the right-hand panel of Figure \ref{fig:onezone-threepanel} shows the effect of the plateau DTD width on abundance tracks and distributions. The effect on the [O/Fe] DF is similar to the previous two models: a longer plateau raises the median delay time, resulting in a smaller low-$\alpha$ peak and a more prominent high-$\alpha$ tail. The abundance tracks from several different plateau widths fill the space in between the exponential ($\tau=3$ Gyr) and power-law ($\alpha=-1.1$ with no plateau) models, which are both included in the panel for reference. The abundance track from the plateau ($W=1$ Gyr) model is nearly identical to the exponential ($\tau=3$ Gyr), although the [O/Fe] DFs are somewhat more distinct, with the plateau model producing a peak closer to the power-law case.

\begin{figure*}
    \centering
    \includegraphics[width=\linewidth]{figures/onezone_twopanel.pdf}
    \caption{\textit{Left:} Comparison of one-zone models with different minimum delay times $t_D$ and DTDs.
    % minimum SN Ia delay times $t_D=40$ Myr (solid) and 150 Myr (dashed curves) for three DTDs: power-law with $\alpha=-1.1$ (blue), plateau with $W=0.3$ Gyr (green), and exponential with $\tau=1.5$ Gyr (purple curves). 
    The layout is similar to Figure \ref{fig:onezone-threepanel}. For clarity, we assume a mass-loading factor $\eta=1$ for the exponential DTD curves, which places the end-point at higher [Fe/H].
    \textit{Right:} Comparison of one-zone models with five different forms for the DTD.
    % : triple-system (yellow), plateau with $W=1$ Gyr (cyan), exponential with $\tau=1.5$ Gyr (purple), power-law with $\alpha=-1.1$ (blue), and two-population (pink).
    }
    \label{fig:onezone-twopanel}
    \script{onezone_twopanel.py}
\end{figure*}

One parameter of the DTD which we hold fixed in the multi-zone models is the minimum SN Ia delay time $t_D$. Typical choices for $t_D$ in the literature range from $\sim 30$ Myr \citep[e.g.,][]{Poulhazan2018-PrecisionPollution} to 150 Myr \citepalias[e.g.,][]{Johnson2021-Migration}. The left-hand panel of Figure \ref{fig:onezone-twopanel} shows that $t_D$ has a much stronger effect on the abundance tracks and distributions of models which assume a power-law DTD than models which assume a plateau or exponential DTD. This is a consequence of the much higher number of prompt SNe Ia in the power-law models compared to the others (see Figure \ref{fig:dtds}). Moreover, a power-law DTD with a long $t_D$ may be observationally hard to distinguish from a plateau model. In Figure \ref{fig:onezone-twopanel}, the abundance track for the model with a power-law DTD and $t_D=150$ Myr (dashed purple line) is similar to that of the plateau DTD with $W=0.3$ Gyr and $t_D=40$ Myr (solid green line), and their [O/Fe] DFs are virtually identical. The difference between the two values of $t_D$ for the exponential ($\tau=3$ Gyr) DTD in both the abundance tracks and distributions is nearly indistinguishable. We do not consider the effect on the other DTDs because a 150 Myr minimum delay time is incompatible with the two-population model, which has $\sim 50$\% of SNe Ia explode in the first 100 Myr, and would have a negligible effect on the triple-system DTD due to its low SN Ia rate at short delay times.

The right-hand panel of Figure \ref{fig:onezone-twopanel} compares the one-zone model outputs from the full range of DTDs we investigate in this work. As with the individual DTD parameters, the choice of DTD primarily affects the location of the high-$\alpha$ ``knee'' in the [Fe/H]--[O/Fe] abundance tracks. At one extreme is the triple-system model, which sees the CC SN plateau extend up to ${\rm [Fe/H]}\approx0.8$ followed by a sharp downward turn as the SN Ia rate suddenly increases at a delay time of 500 Myr. Star formation proceeds for such a long time before the knee that the [O/Fe] DF shows a slight second peak around the CC SN value; this model is the only one which produces any amount of bimodality in the one-zone models. At the other extreme are the two-population and power-law ($\alpha=-1.1$) models which peak immediately after the minimum delay time of 40 Myr, placing the knee at ${\rm [Fe/H]}\approx-1.8$. The two-population model has a unique second knee at ${\rm [Fe/H]}\approx-0.2$ and ${\rm [O/Fe]}\approx0.1$, which is a result of the delayed exponential component. The abundance tracks from the plateau ($W=1$ Gyr) and exponential ($\tau=1.5$ Gyr) models lie in the intermediate space between these extremes. On the other hand, the [O/Fe] DF from the exponential model has the strongest peak, while the plateau ($W=1$ Gyr) and triple-system models produce nearly identical distributions in the low-$\alpha$ regime. The power-law ($\alpha=-1.1$) and two-population DTDs produce similar [O/Fe] DFs despite notably different abundance tracks. The exponential ($\tau=3$ Gyr) and plateau ($W=0.3$ Gyr) models, while not shown, produce similar abundance tracks to the plateau ($W=1$ Gyr) and exponential ($\tau=1.5$ Gyr) models, respectively.

% [Summary and conclusions from one-zone models]
The results presented in this section indicate that the observables which should show the greatest effect from the choice of DTD are the [O/Fe]--[Fe/H] abundance tracks and the [O/Fe] DF. Degeneracies between models in one regime, such as the abundance tracks from the exponential ($\tau=3$ Gyr) and plateau ($W=1$ Gyr) DTDs, can be resolved in the other. Of course, both of these observables are also greatly affected by the parameters of the SFH. In this section we focused on the fiducial inside-out SFH; for a similar comparison of DTDs with a two-infall SFH, we direct the reader to \citet{Palicio2023-AnalyticDTD}.

\section{Results from Multi-Zone Models}
\label{sec:multizone-results}

\subsection{The distribution of [Fe/H]}
\label{sec:feh-df}

\begin{figure*}
    \centering
    \includegraphics[width=\linewidth]{figures/feh_df_comparison.pdf}
    \caption{Distributions of [Fe/H] from multi-zone models with various models for the SFH and DTD. Each row presents distributions of stars within a range of absolute midplane distance: $1\leq|z|<2$ kpc (\textit{top}), $0.5\leq|z|<1$ kpc (\textit{middle}), and $0\leq|z|<0.5$ kpc (\textit{bottom}). Within each panel, curves of different color represent the distributions of stars binned by Galactocentric radius $R_{\rm gal}$, from $3\leq R_{\rm gal}<5$ kpc (yellow) to $13\leq R_{\rm gal}<15$ kpc (blue). Each distribution is normalized so the area under the curve is 1, and the vertical scale is consistent across each row. All distributions are convolved with observational uncertainties in APOGEE DR17 (see Table \ref{tab:uncertainties}) and smoothed with a box-car width of 0.2 dex. 
    \textit{Left columns:} comparison between the inside-out and two-infall SFHs; both assume the exponential ($\tau=1.5$ Gyr) DTD. 
    \textit{Center column:} the distributions from APOGEE DR17 for reference, binned and smoothed similarly.
    \textit{Right columns:} comparison between the power-law ($\alpha=-1.4$) and exponential ($\tau=3$ Gyr) DTDs with the inside-out SFH.}
    \label{fig:feh-df-comparison}
    \script{feh_df_comparison.py}
\end{figure*}

Figure \ref{fig:feh-df-comparison} shows MDFs across the Galaxy for a selection of models and APOGEE data. The two left-hand columns show MDFs from our models which assume the same DTD (exponential with $\tau=1.5$ Gyr) but different SFHs. For comparison, the APOGEE MDFs are binned similarly and presented in the center column of Figure \ref{fig:feh-df-comparison}.  The two right-hand columns

Holding the SFH fixed, varying the DTD has a very minor effect on the MDF across the disk. The two right-hand columns of Figure \ref{fig:feh-df-comparison} plot the MDFs for two multi-zone models which both assume an inside-out SFH but different DTDs: a power-law with slope $\alpha=-1.4$, and an exponential with timescale $\tau=3$ Gyr. The balance between prompt and delayed SNe Ia is starkly different between the two models, with $\sim 80\%$ of SNe Ia exploding within 1 Gyr in the former model but only $\sim 30\%$ in the latter. Nevertheless, while the former DTD produces a slightly narrower MDF in the inner Galaxy than the latter, the magnitude of this difference is much smaller than other factors.

The choice of DTD and SFH has a larger effect on the inner regions of the galaxy than the outer. Inner regions have a higher SFR and more stars are formed at early times than late, so differences in the SN Ia rate have a more pronounced effect on the MDF. For all SFHs except two-infall, the outer galaxy has a longer SFR timescale than the inner galaxy, so the MDF in the outer galaxy is relatively unaffected by the fraction of prompt SNe Ia.

While there are some quantitative differences in how the shape of the MDF varies with Galactic region, the qualitative trends are unaffected by the choice of model SFH or DTD. These trends are primarily driven by chemical equilibrium, abundance gradients, and radial migration \citepalias[][see their Section 3.2 for further discussion]{Johnson2021-Migration}.

To quantify the agreement between the [Fe/H] DFs generated by \vice and those observed in APOGEE, we compute the Kullback-Leibler (KL) divergence \citep{KullbackLeibler1951}, defined as
\begin{equation}
    D_{\rm{KL}}(P \parallel Q) \equiv \int_{-\infty}^{\infty} p(x) \log\Big(\frac{p(x)}{q(x)}\Big) dx
    \label{eq:kl-divergence}
\end{equation}
for distributions $P$ and $Q$ with probability density functions $p(x)$ and $q(x)$. If $D_{\rm KL}=0$, the two distributions contain equal information. In this case, $P$ is the APOGEE DF, $Q$ is the model DF, and $x={\rm [Fe/H]}$. We bin the distributions with a width of 0.01 dex and apply observational uncertainties to our modeled DFs according to Table \ref{tab:uncertainties}. For each model SFH and DTD, we compute $D_{\rm{KL}}$ in the 18 different Galactocentric regions shown in Figure \ref{fig:feh-df-comparison}: bins in $R_{\rm gal}$ with a width of 2 kpc between 3 and 15 kpc, and bins in $|z|$ of $0-0.5$ kpc, $0.5-1$ kpc, and $1-2$ kpc. The score $S$ for the entire model is taken to be the average of $D_{\rm{KL}}$ for each region $R$ weighted by the number of APOGEE stars in that region $N_{*,R}$:
\begin{equation}
    S = \frac{\sum_R D_{{\rm KL},R} N_{*,R}}{\sum_R N_{*,R}}.
    \label{eq:feh-df-score}
\end{equation}

The model combination with the best (lowest) score for the [Fe/H] DF is the two-infall SFH with the exponential ($\tau=3$ Gyr) DTD. The choice of SFH has a larger effect on the overall score than the DTD, and the best-performing SFH is the two-infall model, followed closely by the early-burst. However, the difference between the best-scoring model and the worst (late-burst SFH with triple-evolution DTD) is fairly small. The quantitative agreement between the predicted and observed [Fe/H] DFs is primarily affected by the assumption of chemical equilibrium, the abundance gradient, and radial migration. For further discussion, see Section 3.2 of \citetalias{Johnson2021-Migration}.

\subsection{The distribution of [O/Fe]}
\label{sec:ofe-df}

\begin{figure*}
    \centering
    \includegraphics[width=\linewidth]{figures/ofe_df_sfh.pdf}
    \caption{Distributions of [O/Fe] from multi-zone models with different SFHs. All assume the exponential ($\tau=1.5$ Gyr) DTD. The format of each panel is the same as in Figure \ref{fig:feh-df-comparison}, except that all distributions are smoothed with a box-car width of 0.05 dex. Distributions from APOGEE DR17, binned and smoothed similarly, are presented in the right-most column for reference.}
    \label{fig:ofe-df-sfh}
    \script{ofe_df_sfh.py}
\end{figure*}

The distribution of [O/Fe] serves as a record of the relative rates of SNe Ia and CCSNe. As such, the shape of the distribution is greatly affected by both the SFH and DTD. Figure \ref{fig:ofe-df-sfh} shows the distribution of [O/Fe] across the disk for the four model SFHs, all assuming an exponential DTD with $\tau=1.5$ Gyr, compared to the distributions measured by APOGEE. We see similar trends with Galactic region across all four models. Near the midplane, the distributions at the innermost and outermost annuli are similar, but away from the midplane, there is a clear trend of a higher average stellar [O/Fe] at small $R_{\rm gal}$ and lower [O/Fe] at large $R_{\rm gal}$.

While trends with $R_{\rm gal}$ and $|z|$ are similar, the shape of the distribution varies greatly with the chosen SFH. The inside-out and late-burst models produce similar distributions because of the similarity of their underlying SFHs \textemdash both skew heavily toward near-solar [O/Fe], although the late-burst model produces a slightly broader peak and a less-prominent high-[O/Fe] tail. The only region which shows any significant skew toward high [O/Fe] is $R_{\rm gal}=3-5$ kpc and $|z|=1-2$ kpc, but the shift to higher [O/Fe] at high latitudes is gradual and does not produce the notable trough at ${\rm [O/Fe]}\approx0.2$ which is seen in the APOGEE data. 

On the other hand, the early-burst model produces a bimodal distribution of [O/Fe] even out to large radii and close to the midplane, whereas the APOGEE distribution does not show a prominent high-$\alpha$ peak beyond $R_{\rm{Gal}}\sim11$ kpc. 
The early-burst SFH produces the closest match to the data by far: the low-$\alpha$ sequence away from the midplane shows increasing contribution with larger radii, with the innermost radii containing very few low-$\alpha$ stars. However, the high-$\alpha$ sequence contains substantial contribution from stars at large radii, which is not seen in the data.

The two-infall SFH produces \textit{three} distinct modes at ${\rm [O/Fe]}\approx 0.0$, $0.2$, and $0.4$. At small $R_{\rm gal}$ and with increasing $|z|$, the low-$\alpha$ peak decreases in prominence and the high-$\alpha$ peak increases, but the intermediate peak is a striking feature at all latitudes which is at odds with the APOGEE data.

\begin{figure*}
    \centering
    \includegraphics[width=\linewidth]{figures/ofe_df_dtd.pdf}
    \caption{Distributions of [O/Fe] from multi-zone models with different DTDs. In all cases an early-burst SFH is assumed. The plot format is similar to Figure \ref{fig:ofe-df-sfh}.}
    \label{fig:ofe-df-dtd}
    \script{ofe_df_dtd.py}
\end{figure*}

Figure \ref{fig:ofe-df-dtd} shows [O/Fe] distributions produced by models with the same SFH but a range of different DTDs. We show the early-burst SFH because it produces distinct low- and high-$\alpha$ sequences. The most obvious effect of the DTD is to shift the mode of the high-$\alpha$ sequence. The two-population DTD, which has the most prompt SNe Ia, places the high-$\alpha$ sequence at ${\rm [O/Fe]}\approx 0.2$, while the triple-system DTD, which has the fewest prompt SNe Ia, places it $\sim0.2$ dex higher at ${\rm [O/Fe]}\approx0.4$. In general, models fewer prompt SNe Ia also populate the high-$\alpha$ sequence with more stars, as the chemical evolution track spends more time in the high-$\alpha$ regime. The exception to this is the two-population DTD, which produces a very strong high-$\alpha$ peak at $|z|>0.5$ kpc, likely a result of its long exponential tail.

% [Talk about scores]
We again compute the KL divergence (Equation \ref{eq:kl-divergence}) to quantify the agreement between the [O/Fe] DFs of our models and APOGEE. We calculate a score for each model as described in Section \ref{sec:feh-df}. The best-scoring model uses the two-infall SFH with the plateau ($W=0.3$ Gyr) DTD, although both plateau and exponential DTDs as well as the triple-system DTD score well when combined with certain SFHs. The late-burst SFH in general produces [O/Fe] DFs which score poorly compared to other models.

\subsection{Bimodality in [O/Fe]}
\label{sec:bimodality}

\begin{figure*}
    \centering
    \includegraphics[width=\linewidth]{figures/ofe_bimodality_summary.pdf}
    \caption{The distributions of [O/Fe] along two different slices of [Fe/H]. 
    % $-0.6\leq$[Fe/H]$<-0.4$ (blue dashed) and $-0.4\leq$[Fe/H]$<-0.2$ (red solid). 
    All panels contain stars within the Galactic region defined by $7\leq R_{\rm gal}<9$ kpc and $0\leq|z|<2$ kpc. The $|z|$-distributions of the model stellar populations are re-sampled to match the vertical distribution of the APOGEE sample, and \num{20000} stellar populations are drawn from the adjusted distribution in each panel. The maximum of each distribution is normalized to 1 and the vertical scale is consistent across all panels.
    \textit{Top row:} results from five multi-zone models which assume the late-bust SFH but different DTDs. \textit{Bottom row}: the first four panels compare the four SFHs, all assuming an exponential DTD with $\tau=1.5$ Gyr. The bottom-right panel (highlighted) plots data from APOGEE DR17 for reference.}
    \label{fig:ofe-bimodality}
    \script{ofe_bimodality_summary.py}
\end{figure*}

A striking feature of the [O/Fe] distribution in the Milky Way is the distinct separation into two components, the high- and low-$\alpha$ sequences. The relative prominence of the two peaks varies with $R_{\rm gal}$ and $|z|$, as shown by the right-hand panels of Figures \ref{fig:ofe-df-sfh} and \ref{fig:ofe-df-dtd}. A crucial feature of this bimodality which is not apparent in the previous figures is the coincidence of both sequences at the same [Fe/H]. Figure \ref{fig:ofe-bimodality} compares the [O/Fe] distributions in the solar annulus ($7\leq R_{\rm gal}<9$ kpc and $0\leq|z|<2$ kpc) in two bins of [Fe/H] for select model outputs and APOGEE data. The APOGEE distributions in the bottom-right panel (j) show that the high-$\alpha$ mode is more prominent at lower [Fe/H] and \vice versa, but the distributions in both bins are clearly bimodal. Moreover, the ``trough'' at ${\rm [O/Fe]}\approx0.2$ is consistent in each bin.

[Reference \citet{Vincenzo2021-AlphaDistribution} for how the bimodality is made stronger by the selection function, but it's not just due to selection effects. Describe the re-sampling we do to correct for this.] For most of our models, the low-$\alpha$ peak is the more prominent of the two, so by performing this resampling we enhance strength of the bimodality. However, for some models the high-$\alpha$ peak is the strongest, and the re-sampling process de-emphasizes the low-$\alpha$ peak below the prominence threshold (e.g., panel (f) of Figure \ref{fig:ofe-bimodality}). [Explain why this is ok]

The top row of panels (a--e) in Figure \ref{fig:ofe-bimodality} shows the [O/Fe] bimodality (or lack thereof) across five different DTDs, all of which assume the late-burst SFH. To quantify the bimodality, we use the {\tt scipy}'s \citep{2020SciPy-NMeth} peak-finding algorithm with an arbitrary prominence threshold of 0.1. Five of the eight DTDs (four of the five shown, plus the exponential ($\tau=3$ Gyr) model) exceed this threshold, while both power-law DTDs and the plateau ($W=0.3$ Gyr) model do not. Panel (b) illustrates that the power-law ($\alpha=-1.1$) DTD produces a marginal high-$\alpha$ peak, although it doesn't meet the prominence threshold because it is too close to the low-$\alpha$ peak. In general, as was the case with the [O/Fe] distributions in Section \ref{sec:ofe-df}, DTDs with fewer prompt SNe Ia produce a high-$\alpha$ peak which is more prominent and at a higher [O/Fe]. One major discrepancy is that the APOGEE data show bimodality in both bins of [Fe/H], while the models run with the late-burst SFH only show a clear high-$\alpha$ peak at the low-metallicity ($-0.6<{\rm [Fe/H]}<-0.4$) bin. [Talk about why / if this matters]

Panels (f)--(i) in the bottom row of Figure \ref{fig:ofe-bimodality} illustrate the effect of the SFH on the [O/Fe] bimodality with the exponential ($\tau=1.5$ Gyr) DTD. The inside-out SFH does not produce a bimodal distribution for any of our DTDs. On the other hand, the early-burst SFH {\it always} produces a bimodal distribution in the $-0.4\leq{\rm [Fe/H]}<-0.2$ bin regardless of the assumed DTD, although the lower-metallicity bin does not (the lower peak falls short of our prominence threshold of 0.1). For models with the late-burst and two-infall SFHs, bimodality is variable depending on the DTD: those with longer median delay times (e.g., exponential, plateau, or triple-system) generally produce a bimodal distribution in the $-0.6\leq{\rm [Fe/H]}<-0.4$ bin, while the power-law DTDs do not. None of our SFH+DTD combinations can produce an [O/Fe] distribution which is bimodal in {\it both} bins of [Fe/H], as we observe in APOGEE (j).

\subsection{The [O/Fe]--[Fe/H] Plane}
\label{sec:ofe-feh}

\begin{figure}
    \centering
    \includegraphics{figures/ofe_feh_sfh.pdf}
    \caption{A comparison of the [O/Fe]--[Fe/H] plane between the four SFHs in our multi-zone models. All assume the exponential ($\tau=1.5$ Gyr) DTD. Each panel plots a random mass-weighted sample of 10,000 star particles in the solar neighborhood ($7\leq R_{\rm gal}<9$ kpc, $0\leq|z|<0.5$ kpc) color-coded by $R_{\rm gal}$ at birth. A Gaussian scatter has been applied to all points based on the median abundance errors in APOGEE DR17 (see Table \ref{tab:uncertainties}). The black curves represent the ISM abundance tracks in the 8 kpc zone. The red contours represent a 2-D kernel density estimate with a bandwidth of 0.03 of the APOGEE abundance distribution in that Galactic region, with the solid and dashed contours enclosing 30\% and 80\% of stars in the sample, respectively.}
    \label{fig:ofe-feh-sfh}
    \script{ofe_feh_sfh.py}
\end{figure}

[Subsection introduction]
Section \ref{sec:onezone-results} and specifically Figures \ref{fig:onezone-threepanel} and \ref{fig:onezone-twopanel} illustrated the effect of the DTDs and parameters on [O/Fe]--[Fe/H] tracks in idealized one-zone models.

Figure \ref{fig:ofe-feh-sfh} compares the [O/Fe]--[Fe/H] plane in the solar neighborhood ($7\leq R_{\rm gal}<9$ kpc, $0\leq|z|<0.5$ kpc) between our four model SFHs. All models presented in this figure assume an exponential DTD with $\tau=1.5$ Gyr, as it lies in the middle in terms of balance between prompt and delayed SNe Ia. The black curves represent the ISM abundance as a function of time in the central zone; in the absence of radial migration, all simulated stellar populations would lie along these lines. Stars to the left of the abundance tracks were born in the outer disk, while stars to the right of the tracks were born in the inner disk, as evidenced by the color-coding in the figure.

The inside-out and late-burst models produce similar results, the main difference being a slightly broader core of the abundance distribution around ${\rm [O/Fe]}\approx0.05$ dex. The Gaussian burst in SFH in the late-burst model introduces a loop in the ISM abundance track, where an uptick in star formation at $t\approx11$ Gyr raises the CCSN rate, leading to a slight increase in [O/Fe] before the subsequent increase in the SN Ia rate brings it back down. This same pattern is seen much more strongly in the abundance tracks for the two-infall model: here, the significant infall of pristine gas at $t=4$ Gyr leads to rapid dilution of the ISM, followed by a large burst in the SFR which raises [O/Fe] by $\sim 0.2$ dex. In this case the second infall produces a ``ridge line'' in the abundance distribution at ${\rm [O/Fe]}\approx 0.2$. Finally, the early-burst model shows the clearest separation between a high-[O/Fe] track which follows the ISM abundance from low metallicities, and a low-[O/Fe] sequence composed of stars with a wide range of birth radii. The early-burst model sees a higher density of points which closely follow the ISM track at ${\rm [Fe/H]}\lesssim 0.5$ which is likely the result of an increased SFR at early times relative to the other models.

\begin{figure*}
    \centering
    \includegraphics[width=\linewidth]{figures/ofe_feh_dtd.pdf}
    \caption{The [O/Fe]--[Fe/H] plane from multi-zone models with different DTDs. All assume the inside-out SFH. Each panel is similar to those in Figure \ref{fig:ofe-feh-sfh}, except each row contains star particles from a different bin in $|z|$, with stars closest to the midplane in the bottom row and stars farthest from the midplane in the top row. All panels contain stars within the solar annulus ($7\leq R_{\rm gal}<9$ kpc).}
    \label{fig:ofe-feh-dtd}
    \script{ofe_feh_dtd.py}
\end{figure*}

Figure \ref{fig:ofe-feh-dtd} compares [O/Fe]--[Fe/H] distributions for five models with the same SFH (inside-out) but different DTDs. We choose the inside-out SFH for this figure because it produces a monotonically-decreasing abundance track, making comparisons between the different DTDs relatively straightforward. We note that in this model, both the low- and high-$\alpha$ sequences are composed of stars with a wide range of birth $R_{\rm gal}$. The model results are arranged by increasing median delay time from left to right. Results for the two-population and power-law ($\alpha=-1.1$) DTDs, with a large fraction of prompt SNe Ia, are presented in the first and second columns, respectively. For these models, the simulated stellar abundances are reasonably well-aligned with the APOGEE contours at low $|z|$ (bottom panels), but they entirely miss the observed high-$\alpha$ sequence at high $|z|$ (top panels). In the center column, the exponential ($\tau=1.5$ Gyr) DTD produces a distribution which aligns quite well with the APOGEE contours in all $|z|$-bins, and even produces a ``ridge'' which extends to large [O/Fe] at low- and mid-latitudes (bottom and center panels, respectively). While it does better at populating the high-$\alpha$ sequence than the previous DTDs, the bulk of the simulated populations at large $|z|$ still fall below the APOGEE 30\% contour (top panel). The two right-hand columns present model results for the plateau ($W=1$ Gyr) and triple-system DTDs, which have the longest median delay times. The high-$\alpha$ ``knee'' occurs at a much higher metallicity in these models and is visible in the gas abundance tracks in the upper-left corner of the panels. At high $|z|$, the simulated abundance align quite well with the APOGEE high-$\alpha$ sequence (top panels), but there is a significant ridge of high-$\alpha$ stars from the inner Galaxy at low $|z|$ (bottom panels).

[Discuss effect of radial migration]

\citet{PerezCruz2008-KLTest2D} present a method for estimating the KL divergence between two continuous, multivariate samples using a $k$-nearest neighbor estimate. We implement this method to calculate the divergence between the multi-zone models and APOGEE data in [Fe/H]--[O/Fe] space.
% \footnote{We got the implementation from \url{https://mail.python.org/pipermail/scipy-user/2011-May/029521.html}.}

\begin{figure}
    \centering
    \includegraphics{figures/ofe_feh_twoinfall.pdf}
    \caption{The [O/Fe]--[Fe/H] plane for multiple Galactocentric regions from the model with the inside-out SFH and plateau ($W=0.3$ Gyr) DTD. The two columns of panels contain stars in different bins of $R_{\rm gal}$, and each row contains stars from a different bin of $|z|$. The contents of each panel is as described in Figure \ref{fig:ofe-feh-sfh}.}
    \label{fig:ofe-feh-twoinfall}
    \script{ofe_feh_twoinfall.py}
\end{figure}

The best-scoring model uses the two-infall SFH with the plateau ($W=0.3$ Gyr) DTD. Figure \ref{fig:ofe-feh-twoinfall} plots the stellar [O/Fe]--[Fe/H] abundances from this model in the inner and outer Galaxy. The model distributions in the outer Galaxy closely match the APOGEE distributions. Stellar populations at the high-[Fe/H] end of the distribution have migrated from inner regions.

\subsection{The Age--[O/Fe] Plane}
\label{sec:age-ofe}

\begin{figure}
    \centering
    \includegraphics{figures/age_ofe_sfh.pdf}
    \caption{A comparison of the age--[O/Fe] relation between multi-zone models with different SFHs. All assume the exponential ($\tau=1.5$ Gyr) DTD. Each panel plots a random mass-weighted sample of 10,000 star particles in the solar neighborhood ($7\leq R_{\rm gal}<9$ kpc, $0\leq|z|<0.5$ kpc) color-coded by the Galactocentric radius at birth. A Gaussian scatter has been applied to all points based on the median [O/Fe] error from APOGEE DR17 and the median age error from \citetalias{Leung2023-Ages} (see Table \ref{tab:uncertainties}). Black squares represent the mass-weighted median age of star particles within bins of [O/Fe] with a width of 0.05 dex, and the horizontal black error bars encompass the 16th and 84th percentiles. Red triangles and horizontal error bars represent the median, 16th, and 84th percentiles of age from \citetalias{Leung2023-Ages}, respectively. For clarity, bins which contain less than 1\% of the total mass (in the models) or total number of stars (in the data) are not plotted.}
    \label{fig:age-ofe-sfh}
    \script{age_ofe_sfh.py}
\end{figure}

[Subsection introduction]

Figure \ref{fig:age-ofe-sfh} compares the stellar age and [O/Fe] distributions in the solar neighborhood between our four SFHs. Similar to Figure \ref{fig:ofe-feh-sfh}, all four panels assume an exponential DTD with $\tau=1.5$ Gyr. We include age data from \citetalias{Leung2023-Ages} for qualitative comparison, though we caution against drawing strong conclusions because we don't correct for selection effects or systematics in the neural net ages. The inside-out and late-burst models show fair agreement with the data at high [O/Fe], although both show a $\sim2$ Gyr offset to older ages. One could shift the ramp-up in star formation to slightly later times or simply run the model for a shorter amount of time to close this gap. Like the data, the trend in the median age with decreasing [O/Fe] decreases monotonically in the inside-out model. The late-burst model, however, shows a bump in the distribution around a lookback time of $\sim2$ Gyr driven by the Gaussian SFR burst, which is not seen in the data, as noted by \citetalias{Johnson2021-Migration}. Neither model produces a significant amount of stars below solar [O/Fe], in contrast to the data.

For the early-burst SFH, the simulated stellar ages are almost perfectly aligned with the data for ${\rm [O/Fe]}\gtrsim 0.2$. The rapid rise in the SFE at early times delays the descent to lower [O/Fe] values and produces a clump of low-metallicity, high-[O/Fe] stars at an age of $\sim10$ Gyr. The early-burst model produces a sharp transition between a linear relationship between [O/Fe] and age for stars older than $\sim5$ Gyr and a flat relationship for the younger stars, and it too misses the lowest [O/Fe] bin in the data. Finally, the two-infall SFH produces the closest overall agreement between the simulated and observed median ages. Stars with ${\rm [O/Fe]}\gtrsim 0.25$ were produced in the first infall, while the second infall produces a clump of stars with similar metallicity, ages of $\sim8$ Gyr, and ${\rm [O/Fe]}\approx0.2$. There is an interesting small population of very old, low-$\alpha$ stars which is not seen in any of the other models. The two-infall model is the only one which plateaus at sub-solar [O/Fe] for the youngest stars, leading to a very good match to the observed data at low [O/Fe].

\begin{figure*}
    \centering
    \includegraphics[width=\linewidth]{figures/age_ofe_dtd.pdf}
    \caption{A comparison of the age--[O/Fe] relation between multi-zone models with different DTDs. All assume the early-burst SFH. Each row contains star particles from a different bin in $|z|$, with stars closest to the midplane in the bottom row and stars farthest from the midplane in the top row. In all panels stars are limited to the solar annulus ($7\leq R_{\rm gal}<9$ kpc), and the layout of each panel is as in Figure \ref{fig:age-ofe-sfh}.}
    \label{fig:age-ofe-dtd}
    \script{age_ofe_dtd.py}
\end{figure*}

Figure \ref{fig:age-ofe-dtd} similarly plots the simulated distributions in age--[O/Fe] space for five of our DTDs. All models were run with the early-burst SFH because the high-$\alpha$ sequence was distinct in metallicity space in Figure \ref{fig:age-ofe-sfh}. Similar to Figure \ref{fig:ofe-feh-dtd}, models are arranged from left to right by increasing median SN Ia delay time. Moving from left to right, the high-$\alpha$ sequence moves to higher [O/Fe] with increasing median delay time, from $\sim0.4$ for the two-population model to $\sim0.4$ for the triple-system DTD. As we have seen in previous figures, the range in [O/Fe] produced by the models on the left is much smaller than for the models on the right. At high $|z|$ (top row), the observed range of [O/Fe] is larger than produced by any of our models, although the plateau ($W=1$ Gyr) and triple-system models come close. While the other three fall short of the observed range in [O/Fe], they more closely match the median age--[O/Fe] relation. There is a slight reversal in the observed trend for the stars with the highest [O/Fe]: the $0.45\leq{\rm [O/Fe]}<0.5$ bin has a slightly \textit{lower} median age than the $0.3\leq{\rm [O/Fe]}<0.35$ bin at high $|z|$ in the \citetalias{Leung2023-Ages} sample, a small effect but one which is not observed in any of our model results.

Moving to stars at low $|z|$, our plateau ($W=1$ Gyr) and triple-system DTDs over-produce stars at the old, high-$\alpha$ end of the distribution, while also diverging significantly from the observed sequence near solar [O/Fe]. The exponential ($\tau=1.5$ Gyr) DTD comes closest to reproducing the observed range in [O/Fe], while the two DTDs with the shortest median delay time once again produce a smaller range of [O/Fe] than observed. We note that the break between the linear and flat parts of the relation is sharpest for the exponential DTD, and a more gradual transition is observed for the other four DTDs. This could be because the exponential DTD is most dominant at intermediate delay times ($\tau\sim 1-3$ Gyr) but falls off much faster than the other models at long delay times, so the evolution of [O/Fe] is quite slow for lookback times $\lesssim 5$ Gyr. Overall, the exponential ($\tau=1.5$ Gyr) DTD most closely matches the data for stars with $0\leq|z|<1$ kpc.

We use a different scoring system from previous sub-sections due to the much larger uncertainties in age than [O/Fe]. As shown in Figures \ref{fig:age-ofe-sfh} and \ref{fig:age-ofe-dtd}, in each Galactic region we bin the model outputs and data into bins of [O/Fe] with a width of 0.05 dex. We define the RMS median age difference for the region as
\begin{equation}
    \Delta\tau_{\rm RMS} \equiv \sqrt{\frac{\sum_k \Delta\tau_k^2 n_{{\rm L23},k}}{n_{\rm L23,tot}}}
    \label{eq:age-ofe-score}
\end{equation}
where $\Delta\tau_k=\rm{med}(\tau_{\rm \vice})-\rm{med}(\tau_{\rm L23})$ is the difference between the mass-weighted median star particle age in \vice and the median stellar age from \citetalias{Leung2023-Ages} in bin $k$, $n_{{\rm L23},k}$ is the number of stars from the \citetalias{Leung2023-Ages} age sample in bin $k$, and $n_{\rm L23,tot}$ is the total number of stars in the sample in that Galactic region. As before, the score for the model as a whole is the average of $\Delta \tau_{\rm RMS}$ across all regions, weighted by the number of stars in the age sample in each region.

The best, lowest-scoring model in the age--[O/Fe] plane 

\section{Discussion}
\label{sec:discussion}

\variable{output/summary_table.tex}

The results presented in Section \ref{sec:multizone-results} represent a fraction of the 32 total multi-zone model outputs (four SFHs and eight DTDs) and focus on the even smaller fraction of outputs which lie in the solar annulus. 
% While we focused our attention on results in the solar annulus (for which we still have the best data), we have model outputs which span the entire disk out to 20 kpc. In order to identify trends among our many models, it will be useful to have some semi-quantitative way to compare their outputs to APOGEE data across the disk.
In this Section, we compare all our model outputs to APOGEE across five diagnostics: the [Fe/H] DF, [O/Fe] DF, [Fe/H]--[O/Fe] plane, age--[O/Fe] plane, and [O/Fe] bimodality. We perform statistical tests between APOGEE and the model outputs in each region of the Galaxy (as described in corresponding subsections of Section \ref{sec:multizone-results}), then compute the average weighted by the size of the APOGEE sample in each region to obtain a single numerical score.

The diagnostic scores for each model are presented in Table \ref{tab:results}. These scores are useful to indicate combinations of SFH and DTD which are favorable or disfavorable in certain regimes, but we emphasize that we do not use these scores to fit the models to the data.
To avoid drawing strong conclusions from small numerical differences in scores, 
% Instead of reporting the numerical scores for each model output, 
in the column corresponding to a particular diagnostic we simply write \yes, \meh, or \no, which corresponds to a score in the top, middle, or bottom tirtile out of all of our models, respectively. A \yes indicates that particular combination of model SFH and DTD performed better than most when compared to the APOGEE sample for that diagnostic.

Table \ref{tab:results} indicates combinations of SFH and DTD which are more or less favorable in certain regimes. However, we emphasize that we do not use these scores to fit the models to the data. Much of the variation between models can be explained by the choice of SFH: in general, the two-infall models rank the highest across all diagnostics, while the late-burst models generally compare poorly to the rest. This is different for the bimodality test, where the early-burst SFH is able to produce a bimodal [O/Fe] distribution for all DTDs, as discussed in Section \ref{sec:bimodality}. The choice of DTD has a clear effect on the model scores, and this effect is similar across the diagnostics. The models which perform the best (most \yes's and fewest \no's) are those with the exponential and plateau DTDs. It is interesting that the best-performing DTD overall in our scoring system, the exponential DTD with $\tau=1.5$ Gyr, is intermediate among all our DTDs in terms of the number of prompt SNe Ia and median delay time, but has the {\it lowest} rate at very long delay times ($\sim10$ Gyr). Models with a large fraction of prompt SNe Ia, such as the power-law and two-population DTDs, fare quite poorly, with the steepest power-law ($\alpha=-1.4$) ending up in the bottom tirtile across the board for most of our SFHs. If we remove the two-infall models altogether due to their dominance in the scoring system, our results do not qualitatively change.

% The best DTDs for each SFH
The main takeaway from Table \ref{tab:results} is that while certain DTDs tend to score better than others, there is no single DTD which always produces optimal agreement with the data across all diagnostics. Depending on the choice of SFH, different DTDs will succeed at reproducing different observables. If an inside-out or late-burst SFH is chosen, the exponential DTDs with either timescale will produce optimal agreement with all observables. However, the picture gets less clear-cut when looking at the other SFHs. If the early-burst SFH is chosen, the DTDs with many prompt SNe Ia (power-law and two-population) produce the best agreement with the APOGEE [Fe/H] DFs but the {\it worst} agreement with the other observables. Finally, for the two-infall SFH, any DTD other than the power-law and two-population scores well, but the best-scoring DTD varies between diagnostics. In general, too many prompt SNe Ia worsens agreement with observations across all our SFHs, while too few prompt SNe Ia can also worsen agreement for some SFHs and observables.

% [Discuss \citet{Palicio2024-CosmicSNIaRate}]
\citet{Palicio2024-CosmicSNIaRate} predict the cosmic SN Ia rate from a variety of DTDs and cosmic star formation rates (CSFRs) and fit to observations. They find that, depending on the choice of CSFR, the DTDs of \citet{MatteucciRecchi2001-SNIaTimescale}, \citet{Mannucci2006-TwoPopulations}, \citet{Totani2008-DTD}, and the WIDE DD model of \citet{Greggio2005-AnalyticalRates} can all produce good fits to the observed cosmic SN Ia rate. This study provides a useful basis for comparison because the first and last of these DTDs are similar to our exponential ($\tau=1.5$ Gyr) and plateau ($W=1$ Gyr) DTDs, respectively (see Appendix \ref{app:analytical-dtds} for more discussion); the \citet{Mannucci2006-TwoPopulations} is our two-population DTD; and the \citet{Totani2008-DTD} is a $t^{-1}$ power-law with a minimum delay $t_D=100$ Myr. It is noteworthy that the single-degenerate DTD of \citet{MatteucciRecchi2001-SNIaTimescale} best fits the majority of CSFRs in \citet{Palicio2024-CosmicSNIaRate}, and our similar exponential ($\tau=1.5$ Gyr) DTD {\it also} performs best across all our SFHs.

\citet{Poulhazan2018-PrecisionPollution} found that DTDs with a significant prompt component (power-law and two-population) produced the highest average stellar abundance of [O/Fe]. They also found that these models produced the narrowest [O/Fe] distributions and suppressed the long tail to low [O/Fe] ratios. Conversely, their models with essentially 0 prompt SNe Ia produced a lower average stellar [O/Fe] ratio at late times and a broader distribution of [O/Fe]. Their isolated galaxy simulations have an SFR which features an early ($<1$ Gyr) burst followed by a roughly exponential decline, qualitatively similar to our early-burst model but with the peak of star formation occuring $\sim3$ Gyr earlier. We note that their simulations were not designed to reproduce a Milky Way-like galaxy; as such, their [O/Fe] distributions peak around $\sim0.45$ which is higher than the APOGEE high-$\alpha$ sequence. Nevertheless, our study reproduces their finding that a DTD with more prompt SNe Ia will produce a narrower distribution of [O/Fe]. One exception to this is the exponential DTD, which has a marginally narrower distribution than the power-law DTD close to the midplane [numbers here] but a more prominent hith-$\alpha$ tail. This is likely a consequence of the fact that the exponential DTD has a much lower SN Ia rate at late times ($\sim10$ Gyr) than other models [expand on this].

Across most of the observables, the power-law ($\alpha=-1.1$) DTD, which has the strongest motivation from extra-galactic observations, poorly reproduces the observed Galactic chemical abundance distributions. This can be mitigated somewhat with a longer minimum delay time, which we show in Section \ref{sec:onezone-results} has a similar effect on chemical evolution tracks as the plateau DTD. Even so, it is clear that the high fraction of prompt SNe Ia in the DTD of, e.g., \citet{Maoz2017-CosmicDTD} is at odds with Galactic chemical abundance measurements. 
Measurements of the cosmic SN Ia rate become much more uncertain above $z\approx1$, and it is difficult to constrain the SFH of individual galaxies below [number] [cite], so constraints on the DTD from external galaxies should be more sensitive to the rates at long delay times. On the other hand, our results demonstrate that the high-$\alpha$ sequence in GCE models is highly sensitive to the DTD at short delay times.

[Metallicity dependent DTD] \citet{Gandhi2022-MetallicityDependentRates,Johnson2023-Binaries}

[Is the MW DTD the same as elsewhere?]
\citet{Walcher2016-SelfSimilarity} argue that the similarity of the age--[$\alpha$/Fe] relation between solar neighborhood stars and nearby elliptical galaxies is evidence for a universal DTD; between a bimodal, narrow Gaussian, and power-law DTD, they find the latter best explains their data.

\subsection{The Two-Infall SFH}
\label{sec:two-infall-discussion}

[Why does two-infall do so well across the board, even in the [Fe/H] and [O/Fe] DFs where it seems worse by eye?]
Across every diagnostic in Table \ref{tab:results}, the two-infall models outperform every other SFH.
It is somewhat surprising that the two-infall models score higher than the early-burst even for the [O/Fe] DFs, when the double-peaked distributions produced by the latter are qualitatively closer to the APOGEE data than the former's triple-peaked distributions. This is because the KL divergence test heavily penalizes models with a high density in a region where the observations have little, as is the case for the high-$\alpha$ sequence in the outer Galaxy and close to the midplane. 

[Compare to previous studies]
Discuss \citet{Matteucci2006-BimodalDTDConsequences} findings of the \citet{Mannucci2006-TwoPopulations} DTD in the two-infall model.

\citet{Palicio2023-AnalyticDTD} compare seven different DTDs, including the analytical SD, DD-WIDE, and DD-CLOSE, as well as the empirical two-population and $t^{-1}$ power-law DTDs, in a one-zone model with a two-infall SFH. [Note that they use some different parameters for the analytical models than we do). Critically for our purposes, and in contrast to previous studies of the two-infall model \citep[e.g.,][others]{Chiappini1997-TwoInfall,Spitoni2021-TwoInfall}, the authors include the effects of outflows, although they explore mass-loading values of $\eta=0.2-0.8$ which is significantly lower than in our models.

[First exploration of two-infall with mass loading, radial migration. Further investigation coming in future paper.]

\subsection{Radial Migration \& Bimodality}

One curious chemical feature of the Milky Way disk is the existence of two populations separated by their \aFe abundance ratios at similar metallicity \citep[e.g.,][]{Bensby2014-solarNeighborhoodAbundances}. 

% Taken from introduction
The bimodal distribution of \aFe in stars with similar metallicities was first noted in the solar neighborhood \citep[e.g.,][]{Furhmann1998-NearbyStars}, later extended to the whole of the Milky Way disk \citep{Nidever2014-ChemicalEvolutionAPOGEE,Hayden2015-ChemicalCartography}. The ``high-$\alpha$'' population contains stars with low enrichment from SNe Ia, so their abundance ratios reflect that of almost pure CCSN enrichment; it is associated with the kinematically hot thick disk \citep{Bensby2003-AbundanceTrends}. The ``low-$\alpha$'' population is associated with the thin disk and contains stars with roughly solar $\alpha$-abundances.
Various explanations of the \aFe bimodality have been proposed, including radial stellar mixing \citep{Schonrich2009-RadialMixing} and a bursty gas infall history \citep[e.g.,][]{Spitoni2021-TwoInfall}.
% The origin of the \aFe bimodality is debated. \citet{Schonrich2009-RadialMixing} argue that it arises due to radial migration (churning) even with a smooth SFR. \citetalias{Johnson2021-Migration} use a smooth SFR and a prescription for radial migration but are unable to reproduce the bimodality. \citet{Spitoni2021-TwoInfall} (definitely earlier papers too) argue that two distinct infall episodes can explain the bimodality. 
What separates the two sequences is the ratio of CCSN to SN Ia enrichment, so the DTD ought to have some impact on the \aFe bimodality, perhaps even determining whether or not it exists.

% Taken from introduction
A complicating factor in observing and modeling the effects of the DTD is the migration of stars throughout the disk. \citet{SellwoodBinney2002-RadialMixing} first demonstrated that a transient spiral perturbation (or other potential feature; see [citations]) can cause resonant interactions which adjust stars' guiding center radii without leading to kinematic heating of the disk. Core-collapse SNe explode on timescales of a few Myr, before the progenitor has a chance to migrate far from its birth location. On the other hand, SNe Ia have a broad distribution of delay times and their progenitors can migrate great distances between birth and explosion. The consequence is that SNe Ia can enrich regions of the Galaxy far removed from regions which have experienced a high rate of star formation. Talk about \citet{Scannapieco2005-ChemicalEnrichment}. The first GCE model to incorporate radial mixing was developed by \citet{Schonrich2009-RadialMixing}, and subsequent studies have incorporated more comprehensive numerical treatments of radial migration and the effect of a central bar \citep[e.g.,][]{Minchev2013-ChemodynamicalEvolution}. In this paper, we follow the treatment by \citetalias{Johnson2021-Migration}, who [describe treatment]. 

\citet{Karapetyan2022-SNIaDistances} find a correlation between the SN Ia light curve decline rate, $\Delta m_{15}$, and the distance between the SN and the star-forming shock front of spiral arms. As this distance is expected to be an indicator of the age of the SN progenitor, such a correlation would suggest that the less-luminous, faster-declining SNe Ia come from an older progenitor population, which would be expected from progenitor models with sub-Chandrasekhar mass explosions. A similar study on the {\it radial} migration distance is probably impossible due to the challenges in estimating birth radii even in the Milky Way. It could be interesting to incorporate a SN Ia yield which varies with progenitor age into a GCE model.
[Are there analytic DTDs for a DD progenitor with sub-Chandra explosions?]

Something about \citetalias{Johnson2021-Migration} and \citet{Schonrich2009-RadialMixing}.

\section{Conclusions}
\label{sec:conclusions}

\begin{itemize}

    \item Our diagnostics consistently favor a DTD with a large fraction of delayed SNe Ia, such as the 3 Gyr exponential, 1 Gyr plateau, or triple system evolution models, over a DTD with a high number of prompt SNe Ia, such as the $t^{-1.4}$ power-law or two-population models. [Discuss the disconnect and/or complement with \citet{Maoz2017-CosmicDTD}, etc.; something about different timescales probed by volumetric surveys vs. us]
    
    \item While the choice of DTD cannot produce a bimodal distribution of [O/Fe], it can affect the shape of the distribution, the ratio of high-$\alpha$ to low-$\alpha$ stars, and the location of the high-$\alpha$ sequence if it exists. The SFH is the critical determining factor in bimodality.

    \item The distribution of [Fe/H] depends strongly on the chosen SFH, but there is only a weak effect from the DTD.

    \item The early-burst SFH is the only one which produced a bimodal distribution of [O/Fe] across the disk resembling APOGEE, from the models we investigate. [Add: summary of why everything else fails.]

    \item Something about how radial migration can't produce bimodality on its own.
    
\end{itemize}

Future work
\begin{itemize}
    \item Metallicity-dependent DTD normalization and/or shape.
\end{itemize}

We need to constrain DTD or galactic SFH better. With SDSS-V, the latter seems much more achievable.

\begin{acknowledgments}
    LOD thanks Dr. David Weinberg and attendees of OSU's Galaxy Hour for many useful discussions over the course of this project.
    [Personal acknowledgements]
    
    Funding for the Sloan Digital Sky 
    Survey IV has been provided by the 
    Alfred P. Sloan Foundation, the U.S. 
    Department of Energy Office of 
    Science, and the Participating 
    Institutions. 
    
    SDSS-IV acknowledges support and 
    resources from the Center for High 
    Performance Computing  at the 
    University of Utah. The SDSS 
    website is \url{www.sdss4.org}.
    
    SDSS-IV is managed by the 
    Astrophysical Research Consortium 
    for the Participating Institutions 
    of the SDSS Collaboration including 
    the Brazilian Participation Group, 
    the Carnegie Institution for Science, 
    Carnegie Mellon University, Center for 
    Astrophysics | Harvard \& 
    Smithsonian, the Chilean Participation 
    Group, the French Participation Group, 
    Instituto de Astrof\'isica de 
    Canarias, The Johns Hopkins 
    University, Kavli Institute for the 
    Physics and Mathematics of the 
    Universe (IPMU) / University of 
    Tokyo, the Korean Participation Group, 
    Lawrence Berkeley National Laboratory, 
    Leibniz Institut f\"ur Astrophysik 
    Potsdam (AIP),  Max-Planck-Institut 
    f\"ur Astronomie (MPIA Heidelberg), 
    Max-Planck-Institut f\"ur 
    Astrophysik (MPA Garching), 
    Max-Planck-Institut f\"ur 
    Extraterrestrische Physik (MPE), 
    National Astronomical Observatories of 
    China, New Mexico State University, 
    New York University, University of 
    Notre Dame, Observat\'ario 
    Nacional / MCTI, The Ohio State 
    University, Pennsylvania State 
    University, Shanghai 
    Astronomical Observatory, United 
    Kingdom Participation Group, 
    Universidad Nacional Aut\'onoma 
    de M\'exico, University of Arizona, 
    University of Colorado Boulder, 
    University of Oxford, University of 
    Portsmouth, University of Utah, 
    University of Virginia, University 
    of Washington, University of 
    Wisconsin, Vanderbilt University, 
    and Yale University.
    
    This work has made use of data from the European Space Agency (ESA) mission
    {\it Gaia} (\url{https://www.cosmos.esa.int/gaia}), processed by the {\it Gaia}
    Data Processing and Analysis Consortium (DPAC,
    \url{https://www.cosmos.esa.int/web/gaia/dpac/consortium}). Funding for the DPAC
    has been provided by national institutions, in particular the institutions
    participating in the {\it Gaia} Multilateral Agreement.

    [Ohio State \& Carnegie land acknowledgements]
    % From the Center for Belonging and Social Change, https://cbsc.osu.edu/about-us/land-acknowledgement
    We would like to acknowledge the land that The Ohio State University occupies is the ancestral and contemporary territory of the Shawnee, Potawatomi, Delaware, Miami, Peoria, Seneca, Wyandotte, Ojibwe and many other Indigenous peoples. Specifically, the university resides on land ceded in the 1795 Treaty of Greeneville and the forced removal of tribes through the Indian Removal Act of 1830. As a land grant institution, we want to honor the resiliency of these tribal nations and recognize the historical contexts that has and continues to affect the Indigenous peoples of this land.

    [David and Jennifer NSF grant]
    % Software acknowledgements if not a AAS journal
    % This work made use of Astropy:\footnote{http://www.astropy.org} a community-developed core Python package and an ecosystem of tools and resources for astronomy \citep{astropy:2013, astropy:2018, astropy:2022}. 
\end{acknowledgments}

\software{\vice \citep{JohnsonWeinberg2020-Starbursts}, Astropy \citep{astropy2013,astropy2018,astropy2022}, scikit-learn \citep{Pedregosa2011-ScikitLearn}, SciPy \citep{2020SciPy-NMeth}, Matplotlib \citep{Hunter2007-Matplotlib}}

\appendix

\section{Reproducibility}
\label{app:reproducibility}

This study was carried out using the reproducibility software
\href{https://github.com/showyourwork/showyourwork}{\showyourwork}
\citep{Luger2021-showyourwork}, which leverages continuous integration to
programmatically download the data from
\href{https://zenodo.org/}{zenodo.org}, create the figures, and
compile the manuscript. Each figure caption contains two links: one
to the dataset stored on zenodo used in the corresponding figure,
and the other to the script used to make the figure (at the commit
corresponding to the current build of the manuscript). The git
repository associated to this study is publicly available at
\url{\GitHubURL}, and the release v.X.X allows anyone to re-build the entire 
manuscript. The datasets are stored at [URL].

\section{Analytical DTDs}
\label{app:analytical-dtds}

Discuss \citet{Greggio2005-AnalyticalRates} vs our simple function approximations here?

\citet{Greggio2005-AnalyticalRates} derives analytical DTDs for SD and DD progenitor systems from assumptions about binary stellar evolution and outcomes of mass exchange. They find that the parameters which have a large effect on the shape of the DTD are the distribution and range of stellar masses in progenitor systems; the efficiency of accretion in the SD scenario; and the distribution of separations at birth in the DD scenario. The left-hand panel of Figure \ref{fig:analytical-dtd} shows the analytical DTDs for SD progenitors and two different prescriptions for DD progenitors (``WIDE'' and ``CLOSE''). In the ``WIDE'' scheme, it is assumed that there is a wide distribution of ratios $A/A_0$ of the separation of the DD system to the initial separation of the binary, and that the distributions of $A$ and total mass of the system $m_{\rm DD}$ are independent, so one cannot necessarily predict the total merge time of a system based on its initial parameters. In the ``CLOSE'' scheme, there is assumed to be a narrow distribution of $A/A_0$ and a correlation between $A$ and $m_{\rm DD}$, so the most massive binaries tend to merge quickly and the least massive merge last.

The right-hand panel of Figure \ref{fig:analytical-dtd} shows the results of one-zone chemical evolution models with the \citet{Greggio2005-AnalyticalRates} DTDs. We assume $\eta=2.5$, $\uptau_*=2$ Gyr, an inside-out SFH evaluated at a radius of 8 kpc, a continuous recycling approximation, and a minimum SN Ia delay of 40 Myr. We compare the SD, DD WIDE, and DD CLOSE schemes to standard power-law, broken power-law, and exponential DTDs. The SD and DD CLOSE DTDs follow nearly identical tracks in [O/Fe] vs [Fe/H]; however, their distributions on [O/Fe] differ at the low end. The SD DTD follows an exponential with a 1.5 Gyr timescale, whereas the DD CLOSE DTD is well-approximated by a broken power-law with an initial plateau of 300 Myr and a subsequent declining slope of -1.1. The WIDE prescription is likewise best approximated by a broken power-law, but with a longer plateau width of 1 Gyr. In all cases, the difference between the analytical DTD and the nearest broken power-law or exponential is likely too small to be observationally detectable, so in our multi-zone models we implement the more generic functions in lieu of the analytical DTDs.

A multi-zone with the \citet{Greggio2005-AnalyticalRates} SD DTD produced nearly identical results to an exponential DTD with a 1.5 Gyr timescale.

\begin{figure*}
    \centering
    \includegraphics[width=0.49\linewidth]{figures/dtd_analytical.pdf}
    \includegraphics[width=0.49\linewidth]{figures/onezone_analytical_dtd.pdf}
    \caption{\textit{Left:} Analytical DTDs from \citet[][solid curves]{Greggio2005-AnalyticalRates} and simple DTD functions (dashed curves). Some functions are presented with a constant multiplicative factor for visual clarity. \textit{Right:} Abundance tracks and distributions from one-zone models with the analytical and simple DTDs (same color scheme). For clarity, we vary the mass-loading factor to be $\eta=4$, $\eta=2$, and $\eta=1$ for the red, green, and blue curves, respectively. All model parameters between the similarly-colored solid and dashed curves are identical.}
    \label{fig:analytical-dtd}
    \script{analytical_dtd_twopanel.py}
\end{figure*}

\section{Stellar Migration}
\label{app:migration}

For each star particle in \vice, \citetalias{Johnson2021-Migration} randomly assign an analogue star particle from \texttt{h277} and adopt its radial migration distance $\Delta R$ and final midplane distance $z$. This allows \vice to adopt a realistic pattern of radial migration without needing to implement its own hydrodynamical simulation. However, in regions where the number of h277 star particles is relatively low, such as at large $R_{\rm gal}$ and small $t$, a single h277 star particle can be assigned as an analogue to multiple \vice stellar populations. These populations will have similar formation and migration histories and consequently similar abundances, which produces unphysical ``clumps'' of stars in the abundance distributions at high latitudes and large radii. 

We fit a Gaussian to the distribution of $\Delta R = R_{\rm final} - R_{\rm initial}$ from the \texttt{h277} output in bins of $R_{\rm initial}$ and age. Each Gaussian is centered at 0 and we find that the scale $\sigma_{\Delta R}$ is best described by the function
\begin{equation}
    \sigma_{\Delta R} = 1.35\,{\rm kpc} \Big(\frac{R_{\rm form}}{8\,{\rm kpc}}\Big)^{0.61} \Big(\frac{\tau}{1\,{\rm Gyr}}\Big)^{0.33}
    \label{eq:radial-migration}
\end{equation}
where $\tau$ is the age of the star particle. 

We fit a sech$^2$ function \citep{Spitzer1942} to the distribution of midplane distances $z$. Vertical migration away from the midplane does not affect the chemical evolution model, but we do use $z$ in our analysis. The probability density function (PDF) of $z$ given some scale height $h_z$ is
\begin{equation}
    {\rm PDF}(z) = \frac{1}{4 h_z} {\rm sech}^2\Big(\frac{z}{2 h_z}\Big)
    \label{eq:sech-pdf}
\end{equation}
and the corresponding cumulative distribution function (CDF) is
\begin{equation}
    {\rm CDF}(z) = \frac{1}{1 + e^{-z / h_z}}.
    \label{eq:sech-cdf}
\end{equation}
We fit the above function to the distributions of $z$ in h277 in varying bins of $\tau$ and $R_{\rm final}$ and found that $h_z$ is best described by the function
\begin{equation}
    h_z = (0.25\,{\rm kpc}) 
    e^{\frac{\tau-5\,{\rm Gyr}}{7.0\,{\rm Gyr}}}
    e^{\frac{R_{\rm final}-8\,{\rm kpc}}{6.0\,{\rm kpc}}}.
    \label{eq:scale-height}
\end{equation}

When a star particle is created by \vice at initial radius $R_{\rm form}$, we sample its total radial migration distance $\Delta R$ from a Gaussian with a width described by Equation \ref{eq:radial-migration}, and we sample its final midplane distance $z_{\rm final}$ from the distribution described by Equation \ref{eq:sech-pdf} with a width given by Equation \ref{eq:scale-height}. The star particle migrates to its final radius $R_{\rm final}$ in a similar manner to the ``diffusion'' case from \citetalias{Johnson2021-Migration}, but with a time dependence $\propto \Delta t^{1/3}$. This produces distributions of $R_{\rm final}$ and $z_{\rm final}$ which are similar to the analogue migration case for all but the oldest stars. 

\citet{Okalidis2022-AurigaMigration} studied the stellar migration in the Auriga simulations and found that the migration strength had both a dependence on age and formation radius. They also found that in strongly-barred galaxies, the migration is stronger overall but has a slower timestep evolution than diffusion ($\Delta t^{0.5}$).

Figure \ref{fig:radial-migration} compares the distributions of $R_{\rm final}$ in bins of $R_{\rm form}$ and age between the analogue and Gaussian migration schema. 

\begin{figure*}
    \centering
    \includegraphics[width=\linewidth]{figures/radial_migration.pdf}
    \caption{The distribution of final radius $R_{\rm final}$ as a function of formation radius $R_{\rm form}$ and age for the h277 analogue (top row) and Gaussian sampling scheme (bottom row). From left to right, star particles are binned by formation annulus, from the inner disk (far left column) to the solar annulus (center column) to the outer disk (far right column). Within each panel, colored curves represent the different age bins, ranging from the youngest stars (dark blue) to the oldest (dark red). In the top row, we exclude age bins with fewer than 100 unique analogue IDs for clarity. All distributions are normalized so that the area under the curve is 1, and have been smoothed by a boxcar function with a width of 0.5 kpc. The vertical dotted black lines indicate the bounds of each bin in $R_{\rm form}$; stars within that region of the distribution have not migrated outside their birth annulus over their lifetime.}
    \label{fig:radial-migration}
    \script{radial_migration.py}
\end{figure*}

\begin{figure*}
    \centering
    \includegraphics[width=\linewidth]{figures/midplane_distance.pdf}
    \caption{Similar to Figure \ref{fig:radial-migration} but for the distribution of final midplane distance $z_{\rm final}$ as a function of final radius and age. From left to right, star particles are binned by \textit{final} annulus. In the top row, we exclude age bins with fewer than 500 unique analogue IDs for clarity. All distributions have been smoothed by a boxcar function with a width of 0.1 kpc.}
    \label{fig:midplane-distance}
    \script{midplane_distance.py}
\end{figure*}

\bibliography{bib}

\end{document}
