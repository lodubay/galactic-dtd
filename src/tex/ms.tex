% Define document class
\documentclass[twocolumn,linenumbers,twocolappendix]{aastex631}

\usepackage{showyourwork}
\usepackage{amsmath}
\usepackage{amssymb}
% \usepackage{layouts}
\usepackage{xcolor}
% \usepackage{siunitx}

% custom unit definitions
% \DeclareSIUnit\year{yr}

% user-defined commands
\newcommand{\yes}{\textcolor{green}{\checkmark}}
\newcommand{\meh}{\textcolor{black}{$\sim$}}
\newcommand{\no}{\textcolor{red}{$\times$}}

\begin{document}

% Title
\title{The Galactic Delay-Time Distribution of Type Ia Supernovae:\\
       A Chemical Evolution Perspective}

% Author list
\author[0000-0003-3781-0747]{Liam O. Dubay}
\author[0000-0001-7258-1834]{Jennifer A. Johnson}
\author[0000-0002-6534-8783]{James W. Johnson}
% \author[0000-0001-7775-7261]{David H. Weinberg}

% Abstract from AAS 241
\begin{abstract}
    Type Ia supernovae (SNe Ia) produce most of Fe-peak elements in the Universe and therefore are a crucial ingredient in galactic chemical evolution models. SNe Ia do not explode immediately after star formation and the delay-time distribution (DTD) has not been definitively determined by supernova surveys or theoretical models. Because the DTD also affects the relationship among age, [Fe/H], and [$\alpha$/Fe] in chemical evolution models, comparison with observations of stars in the Milky Way is an important consistency check for any proposed DTD. We implement several popular forms of the DTD in combination with multiple star formation histories for the Milky Way in one-zone and multi-zone chemical evolution models using the Versatile Integrator for Chemical Evolution (VICE). VICE includes a prescription for radial stellar migration in multi-zone models. We compare our predicted interstellar medium abundance tracks, stellar abundance distributions, and stellar age distributions to the 17th data release of the Apache Point Observatory Galactic Evolution Experiment (APOGEE). We find that the SN Ia DTD has the largest effect on the [$\alpha$/Fe] distribution: a DTD with more prompt SNe Ia produces a stellar abundance distribution that is skewed toward a lower [$\alpha$/Fe] ratio, whereas a DTD with more delayed SNe Ia produces a distribution that is skewed toward higher [$\alpha$/Fe]. While the DTD alone cannot explain the observed bimodality in the [$\alpha$/Fe] distribution, in combination with an appropriate star formation history it affects the goodness of fit between the simulated and observed high-$\alpha$ sequence.
\end{abstract}

\section{Introduction}

\section{Methods}
\label{sec:methods}

\subsection{DTD Models}
\label{sec:dtd-models}

We explore four different functional forms for the SN Ia DTD: a power-law, a broken power-law with an initial flat plateau, an exponential, and an exponential with a prompt Gaussian component, with a few different parameterizations for each. Figure \ref{fig:dtds} presents these DTDs as a function of time since star formation.

\begin{figure}
    \centering
    \includegraphics{figures/delay_time_distributions.pdf}
    \caption{Models of the SN Ia delay time distribution used in this paper. All functions are normalized to $f_{\rm Ia}=1$ at $\tau=1$ Gyr.}
    \label{fig:dtds}
    \script{delay_time_distributions.py}
\end{figure}

% \citet{Stolger2020-ExponentialDTD} perform maximum likelihood estimations based on the cosmic star formation history and the star formation histories of individual galaxies. In both cases their best fit functions were approximately exponential, though they started with a skew-normal function so could not have ended up with a power law even if they wanted to. They don't actually report the timescale of this exponential, but based on the plots I'd say it's about 1.5 - 2 (1.7?) Gyr. Their best fit parameters make no sense to me as they don't report their units.

\paragraph{Power-law} A simple power law with slope $\alpha$:
\begin{equation}
    R(t) = A_\alpha (t/1\,\rm{Gyr})^\alpha
    \label{eq:powerlaw-dtd}
\end{equation}
where $A_\alpha = (\alpha+1)(t_{\rm max}^{\alpha+1} - t_{\rm min}^{\alpha+1})^{-1}$ is the normalization coefficient.
A declining power-law with $\alpha\sim-1$ is a standard assumption in many observational studies of the DTD as well as GCE simulations. \citet{Maoz2017-CosmicDTD} obtain a DTD with $\alpha=-1.07\pm0.09$ based on volumetric rates and an assumed cosmic SFH for field galaxies in redshift range $0\leq z\leq 2.25$. \citet{Wiseman2021-DESRates} obtain a similar slope of $\alpha=-1.13\pm0.05$ for field galaxies in the redshift range $0.2<z<0.6$. For galaxy clusters, \citet{Maoz2017-CosmicDTD} find a steeper DTD slope of $\alpha=-1.39^{+0.32}_{-0.05}$. \citet{Heringer2019-FieldGalaxyDTD} use a SFH-independent method to constrain the DTD for field galaxies within $0.01<z<0.2$ and find a steeper slope of $\alpha=-1.34^{+0.19}_{-0.17}$.
In this paper, we investigate the cases $\alpha=-1.1$ and $\alpha=-1.4$.

\paragraph{Plateau} A ``plateau'' model consisting of a flat slope of width $W$ followed by a power law decline with normalization $B_W$:
\begin{equation}
    R(t) = 
    \begin{cases}
        B_W, & t < W \\
        B_W (t/W)^\alpha, & t \ge W
    \end{cases}
    \label{eq:plateau-dtd}
\end{equation}
where $B_W=[(W-t_{\rm min}) + \frac{t_{\rm max}^{\alpha+1} - W^{\alpha+1}}{(\alpha+1)W^\alpha}]^{-1}$ is the normalization coefficient.
We investigate the cases $W=0.3$ Gyr and $W=1$ Gyr, taking $\alpha=-1.1$ for all plateau models. Figure \ref{fig:dtd-greggio05} illustrates the similarity between these models and two different treatments of the analytical double-degenerate DTD from \citet{Greggio2005-AnalyticalRates}.

\begin{figure}
    \centering
    \includegraphics{figures/dtd_greggio05.pdf}
    \caption{Analytical DTDs from \citet[][solid curves]{Greggio2005-AnalyticalRates} and close approximation simple functions (dashed curves). Some functions are presented with a constant multiplicative factor for visual clarity.}
    \label{fig:dtd-greggio05}
    \script{dtd_greggio05.py}
\end{figure}

\paragraph{Exponential} An exponentially declining DTD with timescale $\tau$ and normalization $C_\tau$:
\begin{equation}
    R(t) = \frac{1}{\tau} e^{-t/\tau}
    \label{eq:exponential-dtd}
\end{equation}
We investigate the case where $\tau=1.5$ Gyr, which has been used in previous studies \citep[e.g.,][]{Schonrich2009-RadialMixing,Weinberg2017-ChemicalEquilibrium} and is similar to the analytical single-degenerate DTD from \citet{Greggio2005-AnalyticalRates}. We also investigate a longer timescale of $\tau=3$ Gyr. \citet{Matteucci1986-SupernovaEnrichment}, \citet{Stolger2020-ExponentialDTD}

\paragraph{Prompt} A DTD in which 50\% of SNe Ia belong to a ``prompt'' Gaussian component at small $t$ and the other 50\% form an exponential tail at large $t$, normalized by $D_{t_p,\sigma,\tau}$:
\begin{equation}
    R(t) = D_{t_p,\sigma,\tau} \Big[e^{-\frac{(t-t_p)^2}{2\sigma^2}} + e^{-t/\tau}\Big]
    \label{eq:prompt-dtd}
\end{equation}
Following the two-population model from \citet{Mannucci2006-TwoPopulations}, we take $t_p=50$ Myr, $\sigma=15$ Myr, and $\tau=3$ Gyr, which results in roughly 50\% of SNe Ia exploding within $t<100$ Myr.

\paragraph{Triple} A DTD based on simulations of triple-system evolution by \citet{Rajamuthukumar2022-TripleEvolution}. We approximate their DTD as a special case of the plateau model (Equation \ref{eq:plateau-dtd}) where the initial rate is quite low until an instantaneous rise to the plateau value at time $t_{\rm rise}$:
\begin{equation}
    R(t) = 
    \begin{cases}
        \epsilon B_W, & t < t_{\rm rise} \\
        B_W, & t_{\rm rise} \leq t < W \\
        B_W (t / W) ^ \alpha, & t \geq W
    \end{cases}
    \label{eq:triple-dtd}
\end{equation}
We take $t_{\rm rise}=0.5$ Gyr, $W=0.5$ Gyr, $\alpha=-1.1$, and $\epsilon=0.05$ so the initial rate is 5\% of the peak rate.

Table \ref{tab:dtds} summarizes the model DTDs we investigate in this paper.

\begin{deluxetable*}{lll}
\tablecaption{Summary of SN Ia DTD models explored in this paper.\label{tab:dtds}}
\tablehead{
\colhead{Model} & \colhead{Parameters} & \colhead{Similar to}
}
\startdata
Power-law   & $\alpha=-1.1$                 & \citet[][field]{Maoz2017-CosmicDTD}; 
                                              \citet{Wiseman2021-DESRates}              \\
Power-law   & $\alpha=-1.4$                 & \citet[][cluster]{Maoz2017-CosmicDTD}; 
                                              \citet{Heringer2019-FieldGalaxyDTD}       \\
Plateau     & $W=0.3$ Gyr, $\alpha=-1.1$    & \citet[][CLOSE DD]{Greggio2005-AnalyticalRates} \\
Plateau     & $W=1.0$ Gyr, $\alpha=-1.1$    & \citet[][WIDE DD]{Greggio2005-AnalyticalRates} \\
Triple-system   & $f_{\rm init}=0.05f_{\rm peak}$, $t_{\rm rise}=0.5$ Gyr, & \citet{Rajamuthukumar2022-TripleEvolution} \\
            & $W=0.5$ Gyr, $\alpha=-1.1$ & \\
Exponential & $\tau=1.5$ Gyr    & \citet[][SD]{Greggio2005-AnalyticalRates};
                                  \citet{Schonrich2009-RadialMixing};       \\
            &                   & \citet{Weinberg2017-ChemicalEquilibrium}  \\
Exponential & $\tau=3.0$ Gyr    & --- \\
Prompt      & $t_{\rm max}=0.05$ Gyr, $\sigma=0.015$ Gyr, & \citet{Mannucci2006-TwoPopulations} \\
            & $\tau=3.0$ Gyr & \\
\enddata
\end{deluxetable*}

\subsection{Star Formation Histories}
\label{sec:sfh}

In this paper, we consider four models for the SFH: inside-out, late-burst, early-burst, and two-infall. The first two models are run in VICE's ``star formation mode,'' where the surface density of star formation rate $\dot\Sigma_*$ is prescribed along with the star formation efficiency timescale $\tau_*$, and the infall rate surface density $\dot\Sigma_{\rm in}$ and gas surface density $\Sigma_{\rm gas}$ are calculated from the specified quantities. The latter two models are run in ``infall mode,'' where we specify $\dot\Sigma_{\rm in}$  and $\tau_*$ and the other two quantities follow. 

The functions presented below are dimensionless. For a detailed look at the normalization of the star formation rate with $R_{\rm gal}$, we refer the reader to Appendix B of \citet{Johnson2021-Migration}.

\paragraph{Inside-out} Following \citet{Johnson2021-Migration}, this is our fiducial SFH:
\begin{equation}
    f_{\rm IO}(t|R_{\rm gal}) = (1 - e^{-t/\tau_{\rm rise}}) e^{-t/\tau_{\rm sfh}}
    \label{eq:insideout-sfh}
\end{equation}
where we assume $\tau_{\rm rise}=2$ Gyr for all radii. The SFH timescale $\tau_{\rm sfh}$ varies with $R_{\rm gal}$, with $\tau_{\rm sfh}(R_{\rm gal}=8\,\rm{kpc})\approx15$ Gyr at the Solar annulus and longer timescales in the outer galaxy. The $\tau_{\rm sfh} - R_{\rm gal}$ relation is determined from the data of \citet{Sanchez2020-StarFormationTimescales}; see Section 2.5 of \citet{Johnson2021-Migration} for the details.

\paragraph{Late-burst} A variation on the inside-out SFH with a Gaussian burst in the star formation rate:
\begin{equation}
    f_{\rm LB}(t|R_{\rm gal}) = f_{\rm IO}(t|R_{\rm gal}) \Big(1 + A_b e^{-(t-t_b)^2/2\sigma_b^2} \Big)
    \label{eq:lateburst-sfh}
\end{equation}
where $A_b$ is the dimensionless amplitude of the starburst, $t_b$ is the time of the peak of the burst, and $\sigma_b$ is the width of the Gaussian. Following \citet{Johnson2021-Migration}, we adopt $A_b=1.5$, $t_b=11.2$ Gyr, and $\sigma_b=1$ Gyr. The determination of $\tau_{\rm sfh}$ and the normalization of the SFR as a function of $R_{\rm gal}$ are the same as in the inside-out case.

\paragraph{Early-burst} The infall rate at a given radius declines exponentially with time:
\begin{equation}
    f_{\rm EB}(t|R_{\rm gal}) = e^{-t/\tau_{\rm sfh}}
    \label{eq:earlyburst-ifr}
\end{equation}
The time-dependence of the SFE timescale $\tau_*$ comes from \citet{Conroy2022-ThickDisk}:
\begin{equation}
    \frac{\tau_*(t|\Sigma_g)}{1\,\rm{Gyr}} =
    \begin{cases}
        50, & t < 2.5\,\rm{Gyr} \\
        \frac{50}{[1+3(t-2.5)]^2}, & 2.5\leq t \leq 3.7\,\rm{Gyr} \\
        2.36, & t > 3.7\,\rm{Gyr}
    \end{cases}
    \label{eq:earlyburst-conroy22}
\end{equation}
We combine this with the Kennicutt-Schmidt relation used by \citet{Johnson2021-Migration}, minus the dependence on the molecular timescale $\tau_{\rm mol}$:
\begin{equation}
    \dot \Sigma_* = 
    \begin{cases}
        \Sigma_g, & \Sigma_g \geq \Sigma_{g,2} \\
        \Sigma_g \Big(\frac{\Sigma_g}{\Sigma_{g,2}}\Big)^{2.6}, & \Sigma_{g,1} \leq \Sigma_g \leq \Sigma_{g,2} \\
        \Sigma_g \Big(\frac{\Sigma_{g,1}}{\Sigma_{g,2}}\Big)^{2.6} \Big(\frac{\Sigma_g}{\Sigma_{g,1}}\Big)^{0.7}, & \Sigma_g \leq \Sigma_{g,1}
    \end{cases}
    \label{eq:kennicutt-schmidt}
\end{equation}
Which leads to an overall SFE timescale of
\begin{equation}
    \tau_*(t,\Sigma_g) = \tau_*(t|\Sigma_g) \frac{\Sigma_g}{\dot \Sigma_*}
    \label{eq:earlyburst-taustar}
\end{equation}

\paragraph{Two-infall} 
We parameterize the infall rate as two successive, exponentially declining bursts as in \citet{Chiappini2001-AbundanceGradients}:
\begin{equation}
    \label{eq:twoinfall-ifr}
    f_{\rm TI}(t|R_{\rm gal}) = N_1(R_{\rm gal}) e^{-t/\tau_1} + N_2(R_{\rm gal}) e^{-(t-t_{\rm on})/\tau_2}
\end{equation}
In this model, the first infall produces the thick disk and the second infall produces the thin disk. The normalization ratio $N_2/N_1$ is calculated so that the thick-to-thin-disk surface density ratio $f_\Sigma(R)=\Sigma_2(R)/\Sigma_1(R)$ is given by
\begin{equation}
    f_\Sigma(R) = f_\Sigma(0) e^{R(1/R_2 - 1/R_1)}
\end{equation}
where we take the thick disk scale radius $R_1=2.0$ kpc, thin disk scale radius $R_2=2.5$ kpc, and $f_\Sigma(0)=0.27$ following \citet{BlandHawthornGerhard2016-MilkyWayReview}.

Figure \ref{fig:sfhs} presents an overview of these model star formation histories.

\begin{figure*}
    \centering
    \includegraphics{figures/star_formation_histories.pdf}
    \caption{The surface densities of star formation $\dot \Sigma_*$ (first row), gas infall $\dot \Sigma_{\rm in}$ (second row), and gas mass $\Sigma_{\rm gas}$ (third row), and the star formation efficiency timescale $\tau_*$ (fourth row) as functions of simulation time for our four model SFHs: inside-out (first column; see Equation \ref{eq:insideout-sfh}), late-burst (second column; see Equation \ref{eq:lateburst-sfh}), early-burst (third column; see Equations \ref{eq:earlyburst-ifr} and \ref{eq:earlyburst-taustar}), and two-infall (fourth column; see Equation \ref{eq:twoinfall-ifr}). In each panel, we plot curves for the zones which have inner radii of 4 kpc (yellow), 6 kpc (orange), 8 kpc (red), 10 kpc (violet), 12 kpc (indigo), and 14 kpc (blue).}
    \label{fig:sfhs}
    \script{star_formation_histories.py}
\end{figure*}

\subsection{Stellar Migration}
\label{sec:migration}

For each star particle in VICE, \citet{Johnson2021-Migration} randomly assign an analogue star particle from \texttt{h277} and adopt its radial migration distance $\Delta R$ and final midplane distance $z$. This allows VICE to adopt a realistic pattern of radial migration without needing to implement its own hydrodynamical simulation. However, in regions where the number of h277 star particles is relatively low, such as at large $R_{\rm gal}$ and small $t$, a single h277 star particle can be assigned as an analogue to multiple VICE stellar populations. These populations will have similar formation and migration histories and consequently similar abundances, which produces unphysical ``clumps'' of stars in the abundance distributions at high latitudes and large radii. 

We fit a Gaussian to the distribution of $\Delta R = R_{\rm final} - R_{\rm initial}$ from the \texttt{h277} output in bins of $R_{\rm initial}$ and age. Each Gaussian is centered at 0 and we find that the scale $\sigma_{\Delta R}$ is best described by the function

\begin{equation}
    \sigma_{\Delta R} = 1.35\,{\rm kpc} \Big(\frac{R_{\rm form}}{8\,{\rm kpc}}\Big)^{0.61} \Big(\frac{\tau}{1\,{\rm Gyr}}\Big)^{0.33}
    \label{eq:radial-migration}
\end{equation}

\noindent where $\tau$ is the age of the star particle. 

We fit a sech$^2$ function \citep{Spitzer1942} to the distribution of midplane distances $z_{\rm final}$. Vertical migration away from the midplane does not affect the chemical evolution simulation, but we do use $z_{\rm final}$ in our analysis. The probability density function (PDF) of $z_{\rm final}$ given some scale height $h_z$ is
\begin{equation}
    {\rm PDF}(z_{\rm final}) = \frac{1}{4 h_z} {\rm sech}^2\Big(\frac{z_{\rm final}}{2 h_z}\Big)
    \label{eq:sech-pdf}
\end{equation}
and the corresponding cumulative distribution function (CDF) is
\begin{equation}
    {\rm CDF}(z_{\rm final}) = \frac{1}{1 + e^{-z_{\rm final} / h_z}}.
    \label{eq:sech-cdf}
\end{equation}
We fit the above function to the distributions of $z_{\rm final}$ in h277 in varying bins of $\tau$ and $R_{\rm final}$ and found that $h_z$ is best described by the function
\begin{equation}
    h_z = (0.25\,{\rm kpc}) 
    e^{\frac{\tau-5\,{\rm Gyr}}{7.0\,{\rm Gyr}}}
    e^{\frac{R_{\rm final}-8\,{\rm kpc}}{6.0\,{\rm kpc}}}.
    \label{eq:scale-height}
\end{equation}

When a star particle is created by VICE at initial radius $R_{\rm form}$, we sample its total radial migration distance $\Delta R$ from a Gaussian with a width described by Equation \ref{eq:radial-migration}, and we sample its final midplane distance $z_{\rm final}$ from the distribution described by Equation \ref{eq:sech-pdf} with a width given by Equation \ref{eq:scale-height}. The star particle migrates to its final radius $R_{\rm final}$ in a similar manner to the ``diffusion'' case from \citet{Johnson2021-Migration}, but with a time dependence $\propto t^{1/3}$. This produces distributions of $R_{\rm final}$ and $z_{\rm final}$ which are similar to the analogue migration case for all but the oldest stars. 

Figure \ref{fig:radial-migration} compares the distributions of $R_{\rm final}$ in bins of $R_{\rm form}$ and age between the analogue and Gaussian migration schema. The 

\begin{figure*}
    \centering
    \includegraphics{figures/radial_migration.pdf}
    \caption{The distribution of final radius $R_{\rm final}$ as a function of formation radius $R_{\rm form}$ and age for the h277 analogue (top row) and Gaussian sampling scheme (bottom row). From left to right, star particles are binned by formation annulus, from the inner disk (far left column) to the Solar annulus (center column) to the outer disk (far right column). Within each panel, colored curves represent the different age bins, ranging from the youngest stars (dark blue) to the oldest (dark red). All distributions are normalized so that the area under the curve is 1, and have been smoothed by a boxcar function of width 0.5 kpc. The vertical dotted black lines indicate the bounds of each bin in $R_{\rm form}$; stars within that region of the distribution have not migrated outside their birth annulus over their lifetime.}
    \label{fig:radial-migration}
    \script{radial_migration.py}
\end{figure*}

\begin{figure*}
    \centering
    \includegraphics{figures/midplane_distance.pdf}
    \caption{Similar to Figure \ref{fig:radial-migration} but for the distribution of final midplane distance $z_{\rm final}$ as a function of final radius and age. From left to right, star particles are binned by \textit{final} annulus. All distributions have been smoothed by a boxcar function of width 0.1 kpc.}
    \label{fig:midplane-distance}
    \script{midplane_distance.py}
\end{figure*}

\subsection{Nucleosynthetic Yields}
\label{sec:yields}

Table \ref{tab:yields} summarizes the nucleosynthetic yields assumed in this paper, which we adopt from \citet{Johnson2021-Migration}.

\begin{deluxetable}{lll}
    \tablecaption{Nucleosynthetic yields assumed in our simulations.\label{tab:yields}}
    \tablehead{
        \colhead{Source} & \colhead{Element} & \colhead{Yield}
    }
    \startdata
    CCSN    & O     & 0.015     \\
    CCSN    & Fe    & 0.0012    \\
    \hline
    SN Ia   & O     & 0         \\
    SN Ia   & Fe    & 0.00214   \\
    \enddata
\end{deluxetable}

\subsection{Observational Sample}
\label{sec:observational-sample}

We compare our simulation results to abundance measurements from the 17th data release \citep[DR17;][]{Abdurro'uf2022-SDSSIV-DR17} of the Apache Point Observatory Galactic Evolution Experiment \citep[APOGEE;][]{Majewski2017-APOGEE}. APOGEE uses infrared spectrographs \citep{Wilson2019-APOGEE-Spectrographs} mounted on two telescopes: the 2.5-meter Sloan Foundation Telescope \citep{Gunn2006-SloanTelescope} at Apache Point Observatory, United States in the Northern Hemisphere, and the Ir{\'e}n{\'e}e DuPont Telescope \citep{BowenVaughan1973-DuPontTelescope} at Las Campanas Observatory, Chile in the Southern Hemisphere. After the spectra are passed through the data reduction pipeline \citep{Nidever2015-APOGEE-DataReduction}, the APOGEE Stellar Parameter and Chemical Abundance Pipeline \citep[ASPCAP;][]{Holtzmann2015-ASPCAP,GarciaPerez2016-ASPCAP} extracts chemical abundances using the model grids and interpolation method described by \citet{Jonsson2020-APOGEE-DR16}.

We limit our sample to red giant branch and red clump stars with high-quality spectra; Table \ref{tab:sample-selection} lists our selection criteria, which produce a final sample of \variable{output/sample_size.txt}stars.

\begin{deluxetable*}{lll}
    \tablecaption{Sample selection parameters from APOGEE DR17\label{tab:sample-selection}}
    \tablehead{
        \colhead{Parameter} & \colhead{Range} & \colhead{Notes}
    }
    \startdata
        $\log g$            & $1.0 < \log g < 3.8$          & Select giants only \\
        $T_{\rm eff}$       & $3500 < T_{\rm eff} < 5500$ K & Reliable temperature range \\
        $S/N$               & $S/N > 80$                    & Required for accurate stellar parameters \\
        ASPCAPFLAG Bits     & $\notin$ 23                   & Remove stars flagged as bad \\
        EXTRATARG Bits      & $\notin$ 0, 1, 2, 3, or 4     & Select main red star sample only \\
    \enddata
\end{deluxetable*}

\subsubsection{Stellar Ages}
\label{sec:stellar-ages}

191,179 stars in our sample have astroNN ages and 71,684 have ages from \citet{Leung2023-Ages}.

% For $19784$ stars, or $\sim3.5\%$ of the sample, the astroNN [Fe/H] abundance estimate is $\gtrsim0.5$ dex lower than the value provided by ASPCAP. The vast majority of these stars are on the lower-main sequence according to \textit{Gaia} photometry, so we exclude stars with $\log(g) > 4$ in ASPCAP. In addition, 197 or $\sim3\%$ of stars with APOKASC-2 asteroseismic ages are reported to be $>5$ Gyr younger by astroNN. The APOKASC-2 age is unphysically high for many of these stars, often exceeding 15 Gyr. We identify prior mass loss as the cause of this discrepancy, so the astroNN ages are likely closer to the truth.

\section{Results from One-Zone Models}
\label{sec:onezone-results}

All models are run for a duration of $t_{\rm max}=13.2$ Gyr with a time-step of $\Delta t=0.01$ Gyr. We adopt an outflow mass-loading factor $\eta\equiv \dot M_{\rm out}/\dot M_*=2.15$ \citep[see Equation 8 from][]{Johnson2021-Migration} and a star formation efficiency timescale $\tau_*\equiv M_{\rm gas}/\dot M_*=2$ Gyr. We adopt a continuous recycling prescription \citep[see Equation 2 from][]{JohnsonWeinberg2020-Starbursts}. Unless otherwise specified, we adopt the inside-out star formation rate (Equation \ref{eq:insideout-sfh}) evaluated at $R_{\rm gal}=8$ kpc, with timescales $\tau_{\rm rise}=2$ Gyr and $\tau_{\rm sfh}=15.1$ Gyr, and our models have an initial gas mass $M_{\rm gas,0}=0$. Unless otherwise specified, we assume a minimum SN Ia delay time of 40 Myr. A summary of these one-zone fiducial parameters is presented in Table \ref{tab:onezone-parameters}.

\begin{deluxetable}{Ccl}
    \tablecaption{A summary of parameters and their fiducial values for our one-zone chemical evolution models.\label{tab:onezone-parameters}}
    \tablehead{
        \colhead{Parameter} & \colhead{Value} & \colhead{Description}
    }
    \startdata
    \Delta t        & 10 Myr    & Time-step size \\
    t_{\rm max}     & 13.2 Gyr  & Maximum simulation time \\
    \eta            & 2.15      & Outflow mass-loading factor \\
    \tau_*          & 2.0 Gyr   & Star formation efficiency timescale \\
    \dot M_r        & continuous    & Recycling rate \\
    t_D             & 40 Myr    & Minimum SN Ia delay time \\
    M_{\rm gas,0}   & 0         & Initial gas mass \\
    \dot M_*(t)     & Equation \ref{eq:insideout-sfh}  & Star formation rate \\
    \tau_{\rm rise} & 2 Gyr     & SFR rise timescale \\
    \tau_{\rm sfh}  & 15.1 Gyr  & SFR exponential timescale
    \enddata
\end{deluxetable}

\subsection{The SN Ia Delay-Time Distribution}

Adjusting the parameters of the DTD affects how steeply the abundance track in [Fe/H]-[O/Fe] space falls from its initial ``plateau'' and the location of the ``knee.'' Figure \ref{fig:onezone-slope-timescale} shows abundance tracks from one-zone models which assume different slopes of a power-law DTD or different timescales of an exponential DTD. A steeper power-law slope results in a steeper decline in [O/Fe] and a lower knee, as does a shorter exponential timescale. The consequence is that a very shallow power-law produces a similar abundance track to a very steep exponential, which suggests these models may be difficult to distinguish in this parameter space. However, these cases can be distinguished by their MDFs and [O/Fe] distributions, where the power-law DTD produces a narrower, less skewed distribution while the exponential DTD produces a broader, more skewed distribution. In the [O/Fe] distribution especially, an exponential DTD with a 6 Gyr timescale produces a large number of high-$\alpha$ stars.

\begin{figure}
    \centering
    \includegraphics{figures/onezone_powerlaw_slope.pdf}
    \caption{\textit{Center:} Abundance tracks in the [Fe/H]-[O/Fe] plane for one-zone models with a power-law DTD. The dotted, dashed, and solid curves represent models with power-law slopes $\alpha=-0.8$, $-1.1$, and $-1.4$, respectively. In all cases a minimum SN Ia delay time of 40 Myr is assumed. \textit{Top:} Metalliticy distribution functions (MDFs) generated by the one-zone models. \textit{Right:} Distributions of [O/Fe] generated by the one-zone models.}
    \label{fig:onezone-powerlaw-slope}
    \script{onezone_powerlaw_slope.py}
\end{figure}

\begin{figure}
    \centering
    \includegraphics{figures/onezone_exponential_timescale.pdf}
    \caption{Similar to Figure \ref{fig:onezone-powerlaw-slope} but for one-zone models with an exponential DTD. The dotted, dashed, and solid curves represent exponential decay timescales $\tau=1.5$ Gyr, 3 Gyr, and 6 Gyr, respectively.}
    \label{fig:onezone-exponential-timescale}
    \script{onezone_exponential_timescale.py}
\end{figure}

\begin{figure}
    \centering
    \includegraphics{figures/onezone_plateau_width.pdf}
    \caption{Similar to Figure \ref{fig:onezone-powerlaw-slope} but for one-zone models with a plateau DTD. The solid, dashed, and dot-dashed pink curves represent plateau widths $W=1.0$, 0.3, and 0.1 Gyr, respectively. All plateau DTDs assume a power-law slope of $\alpha=-1.1$. For comparison, the blue dotted curve represents an exponential DTD with a timescale $\tau=3$ Gyr, and the black dotted curve represents a power-law DTD with the same slope but no plateau.}
    \label{fig:onezone-plateau-width}
    \script{onezone_plateau_width.py}
\end{figure}

Figure \ref{fig:onezone-dtd} compares the one-zone model outputs from the full range of DTDs we investigate in this paper. In general, the choice of DTD has the greatest effect on the location of the knee in [Fe/H]-[O/Fe] space (center panel) and the high-$\alpha$ end of the distribution of [O/Fe] (right panel). The DTD also has a smaller effect on the MDF in the range $-1.5\lesssim$ [Fe/H] $\lesssim-0.5$ (top panel).

\begin{figure}
    \centering
    \includegraphics{figures/onezone_dtd.pdf}
    \caption{Similar to Figure \ref{fig:onezone-powerlaw-slope} but with a different selection of DTDs. In all cases a minimum delay time of 40 Myr is assumed.}
    \label{fig:onezone-dtd}
    \script{onezone_dtd.py}
\end{figure}

\subsection{The Minimum SN Ia Delay Time}

Figure \ref{fig:onezone-minimum-delay} shows that the minimum SN Ia delay time has a much stronger effect on the abundance track and MDFs of models which assume a power-law DTD than models which assume a plateau or exponential DTD, due to the much higher SN Ia rate in the power-law models at early times (see Figure \ref{fig:dtds}). Moreover, a power-law DTD with a long minimum delay may be observationally hard to distinguish from a plateau model. In Figure \ref{fig:onezone-minimum-delay}, the abundance track for the model with a power-law DTD and $t_D=150$ Myr (red dashed line) is similar to that of the plateau DTD with $W=0.3$ Gyr and $t_D=40$ Myr (blue solid line), and the [O/Fe] distributions are virtually identical.  

\begin{figure}
    \centering
    \includegraphics{figures/onezone_minimum_delay.pdf}
    \caption{Similar to Figure \ref{fig:onezone-powerlaw-slope} but with varying DTD models and minimum SN Ia delay time ($t_D$). Solid curves represent models with a minimum delay time of 40 Myr, while dashed curves represent models with a minimum delay time of 150 Myr. The red curves represent a power-law DTD with slope $\alpha=-1.1$; blue curves represent an plateau DTD with width $W=0.3$ Gyr and slope $\alpha=-1.1$; and orange curves represent an exponential DTD with timescale $\tau=3$ Gyr. In all cases an exponential rise-fall SFR is assumed.}
    \label{fig:onezone-minimum-delay}
    \script{onezone_minimum_delay.py}
\end{figure}

A 150 Myr minimum delay time is incompatible with the prompt DTD model, which has $\sim 50$\% of SNe Ia explode in the first 100 Myr, and would have a negligible effect on the triple-system DTD as its relative SN Ia rate is so low at early times.

\subsection{The Star Formation History}

Figure \ref{fig:onezone-delay-taustar} shows results from one-zone models with a power-law DTD which differ in their assumed minimum SN Ia delay time ($t_D$) or star formation efficiency timescale ($\tau_*$). In [Fe/H]-[O/Fe] space, doubling or halving the minimum delay time produces similar results to doubling or halving the SFE timescale. A shorter minimum delay time means SNe Ia will start to explode sooner after the star formation burst, giving less time for CCSNe to explode at the plateau of the abundance track. Similarly, a longer SFE timescale means fewer massive stars will be produced by the initial star formation burst, resulting in fewer CCSNe before the first SNe Ia explode. 

The minimum delay and SFE timescale have similar effects on the abundance tracks but very different effects on the MDFs. Altering the SFE timescale changes the stellar distribution of [Fe/H], especially at the low-metallicity end, as shown in the top panel of Figure \ref{fig:onezone-delay-taustar}. However, as a change in the SFE timescale affects both the number of high-mass and low-mass stars produced, it does not affect the distribution of [O/Fe], as shown in the right-hand panel of Figure \ref{fig:onezone-delay-taustar}. Meanwhile, a change in the minimum SN Ia delay time does not impact the total number of SNe which explode, so the distribution of [Fe/H] is largely unaffected, while it does result in a change in the number of high-$\alpha$ stars formed and therefore the distribution of [O/Fe].

\begin{figure}
    \centering
    \includegraphics{figures/onezone_delay_taustar.pdf}
    \caption{Similar to Figure \ref{fig:onezone-powerlaw-slope} but with varying minimum SN Ia delay time ($t_D$) and star formation efficiency timescale ($\tau_*$). Tracks with $\tau_*=2$ Gyr are represented in black, while a track with a shorter $\tau_*=1$ Gyr is shown in red and a track with a longer $\tau_*=4$ Gyr is in yellow. The dotted, solid, and dashed black curves represent a minimum delay time of 40 Myr, 80 Myr, and 160 Myr, respectively. A power-law DTD with $\alpha=-1.1$ is assumed in all models.}
    \label{fig:onezone-delay-taustar}
    \script{onezone_delay_taustar.py}
\end{figure}

\begin{figure}
    \centering
    \includegraphics{figures/onezone_sfh.pdf}
    \caption{Similar to Figure \ref{fig:onezone-powerlaw-slope} but with varying star formation history (SFH). An exponential DTD with $\tau=1.5$ Gyr is assumed in all models.}
    \label{fig:onezone-sfh}
    \script{onezone_sfh.py}
\end{figure}

\subsection{Analytical Formulations}

\citet{Greggio2005-AnalyticalRates} derives analytical DTDs for SD and DD progenitor systems from assumptions about binary stellar evolution and outcomes of mass exchange. They find that the parameters which have a large effect on the shape of the DTD are the distribution and range of stellar masses in progenitor systems; the efficiency of accretion in the SD scenario; and the distribution of separations at birth in the DD scenario. Figure \ref{fig:dtd-greggio05} shows the analytical DTDs for SD progenitors and two different prescriptions for DD progenitors (``WIDE'' and ``CLOSE''). In the ``WIDE'' scheme, it is assumed that there is a wide distribution of ratios $A/A_0$ of the separation of the DD system to the initial separation of the binary, and that the distributions of $A$ and total mass of the system $m_{\rm DD}$ are independent, so one cannot necessarily predict the total merge time of a system based on its initial parameters. In the ``CLOSE'' scheme, there is assumed to be a narrow distribution of $A/A_0$ and a correlation between $A$ and $m_{\rm DD}$, so the most massive binaries tend to merge quickly and the least massive merge last.

% \begin{figure}
%     \centering
%     \includegraphics{figures/greggio05_dtd.pdf}
%     \caption{Analytical DTDs computed by \citet{Greggio2005-AnalyticalRates}. The solid red curve represents the SD DTD for Chandrasekhar explosions, while the green and blue solid curves represent, respectively, the CLOSE and WIDE formulations of a DD DTD. The best-fit approximations for the DD DTDs (see Equation \ref{eq:greggio05-approx}) are represented by the dashed olive and cyan curves, respectively. All DTDs are normalized to unity.}
%     \label{fig:greggio05-dtd}
%     \script{greggio05_dtd.py}
% \end{figure}

Figure \ref{fig:onezone-greggio05} shows the results of one-zone chemical evolution models with the \citet{Greggio2005-AnalyticalRates} DTDs. We assume $\eta=2.5$, $\tau_*=2$ Gyr, an inside-out SFH evaluated at a radius of 8 kpc, a continuous recycling approximation, and a minimum SN Ia delay of 40 Myr. We compare the SD, DD WIDE, and DD CLOSE schemes to standard power-law, broken power-law, and exponential DTDs. The SD and DD CLOSE DTDs follow nearly identical tracks in [O/Fe] vs [Fe/H]; however, their distributions on [O/Fe] differ at the low end. The SD DTD follows an exponential with a 1.5 Gyr timescale, whereas the DD CLOSE DTD is well-approximated by a broken power-law with an initial plateau of 350 Myr and a subsequent declining slope of -1.1. The WIDE prescription is likewise best approximated by a broken power-law, but with a longer plateau width of 1 Gyr. In all cases, the difference between the analytical DTD and the nearest broken power-law or exponential is likely too small to be observationally detectable, so in our multi-zone models we implement the more generic functions in lieu of the analytical DTDs.

A multi-zone with the \citet{Greggio2005-AnalyticalRates} SD DTD produced nearly identical results to an exponential DTD with a 1.5 Gyr timescale.

\begin{figure*}
    \centering
    \includegraphics{figures/onezone_greggio05_single.pdf}
    \includegraphics{figures/onezone_greggio05_double.pdf}
    \caption{Abundance tracks and distributions from one-zone models with the analytical DTDs from \citet{Greggio2005-AnalyticalRates}. \textit{Left:} a comparison between the analytical SD DTD (solid magenta curve), an exponential DTD with a 1.5 Gyr timescale (dashed pink curve), and a power-law DTD with a slope of -1.1 (dotted black curve). \textit{Right:} a comparison between WIDE and CLOSE prescriptions for an analytical DD DTD (solid blue and green curves, respectively), power-law DTDs with an initial plateau of 1 Gyr and 300 Myr (dashed cyan and yellow curves, respectively), and a power-law DTD with no plateau and a slope of -1.1 (dotted black curve).}
    \label{fig:onezone-greggio05}
    \script{onezone_greggio05.py}
\end{figure*}

\begin{figure}
    \centering
    \includegraphics{figures/onezone_twopopulation.pdf}
    \caption{One-zone model abundance tracks and distributions for variants of the prompt DTD.}
    \label{fig:onezone-prompt}
    \script{onezone_twopopulation.py}
\end{figure}

\section{Results from Multi-Zone Models}
\label{sec:multizone-results}

\begin{deluxetable}{lll}
    \tablecaption{Median Parameter Uncertainties\label{tab:uncertainties}}
    \tablehead{
        \colhead{Parameter} & \colhead{Median Uncertainty} & \colhead{Source}
    }
    \startdata
        [Fe/H] & $9.2\times10^{-3}$ & APOGEE DR17 \\
        $\rm [O/Fe]$ & $1.8\times10^{-2}$ & APOGEE DR17 \\
        log(Age/Gyr) & $0.12$ & \citet{Leung2023-Ages}
        % Age/Gyr & 29\% & astroNN
    \enddata
\end{deluxetable}

\subsection{The distribution of [Fe/H]}

To quantify the similarity between the MDFs generated by VICE and the observed APOGEE MDF, we compute the Kullback-Leibler (KL) divergence \citep{KullbackLeibler1951}, defined as
\begin{equation}
\label{eq:kl-divergence}
D_{\rm{KL}}(P \parallel Q) = \int_{-\infty}^{\infty} p(x) \log\Big(\frac{p(x)}{q(x)}\Big) dx
\end{equation}
for distributions $P$ and $Q$ with probability density functions $p(x)$ and $q(x)$. In this case, $P$ is the APOGEE MDF, $Q$ is the model MDF, and $x$ is [Fe/H]. For each model SFH and DTD, we compute $D_{\rm{KL}}$ in 18 different galactocentric zones defined by radial bins $3-5$ kpc, $5-7$ kpc, $7-9$ kpc, $9-11$ kpc, $11-13$ kpc, and $13-15$ kpc and absolute $z$-height bins $0-0.5$ kpc, $0.5-1$ kpc, and $1-2$ kpc. The KL divergence for the entire model is taken to be the average of $D_{\rm{KL}}$ for each zone, weighted by the number of APOGEE stars in each zone.

Figure \ref{fig:feh-df-sfh} plots the distribution of [Fe/H] in multiple regions across the disk for the four SFHs in this paper.

\begin{figure*}
    \centering
    \includegraphics{figures/feh_df_sfh.pdf}
    \caption{Distributions of [Fe/H] for multi-zone simulations with different SFHs. Each row presents distributions of stars within a range of absolute midplane distance: $1\leq|z|<2$ kpc (\textit{top}), $0.5\leq|z|<1$ kpc (\textit{middle}), and $0\leq|z|<0.5$ kpc (\textit{bottom}). Within each panel, curves of different color represent the distributions of stars binned by Galactocentric radius $R_{\rm gal}$, from $3\leq R_{\rm gal}<5$ kpc (yellow) to $13\leq R_{\rm gal}<15$ kpc (blue). Each distribution is normalized so the area under the curve is 1, and the vertical scale is consistent across each row. All distributions are convolved with observational uncertainties in APOGEE DR17 (see Table \ref{tab:uncertainties}) and smoothed with a box-car width of 0.2 dex. In all cases an exponential DTD with timescale $\tau=1.5$ Gyr is assumed. Distributions from APOGEE DR17, binned similarly, are presented in the right-most column for reference.}
    \label{fig:feh-df-sfh}
    \script{feh_df_sfh.py}
\end{figure*}

Holding the SFH fixed, varying the DTD has only a minor effect on the MDF. Figure \ref{fig:feh-df-dtd} plots the [Fe/H] distributions for two multi-zone simulations which both assume an inside-out SFH but different DTDs: a power-law with slope $\alpha=-1.4$, and an exponential with timescale $\tau=3$ Gyr. The balance between prompt and delayed SNe Ia is starkly different between the two models, with $\sim 80\%$ of SNe Ia exploding within 1 Gyr in the former model but only $\sim 30\%$ in the latter. Nevertheless, the effect of the DTD on the MDF is second-order compared to the effect of the SFH as seen in Figure \ref{fig:feh-df-sfh}.

\begin{figure}
    \centering
    \includegraphics{figures/feh_df_dtd.pdf}
    \caption{Distributions of [Fe/H] for multi-zone simulations with different DTDs: a power-law with slope $\alpha=-1.4$ (\textit{left}), and an exponential with timescale $\tau=3$ Gyr (\textit{right}). In both cases an inside-out SFH is assumed. The plot format is similar to Figure \ref{fig:feh-df-sfh}.}
    \label{fig:feh-df-dtd}
    \script{feh_df_dtd.py}
\end{figure}

\subsection{The distribution of [O/Fe]}

\begin{figure*}
    \centering
    \includegraphics{figures/ofe_df_dtd.pdf}
    \caption{Distributions of [O/Fe] for multi-zone simulations with different DTDs. In all cases an inside-out SFH is assumed. The plot format is similar to Figure \ref{fig:feh-df-sfh}. All distributions are smoothed with a box-car width of 0.05 dex.}
    \label{fig:ofe-df-dtd}
    \script{ofe_df_dtd.py}
\end{figure*}

\begin{figure*}
    \centering
    \includegraphics{figures/ofe_df_sfh.pdf}
    \caption{Distributions of [O/Fe] for multi-zone simulations with different SFHs. The plot format is similar to Figure \ref{fig:feh-df-sfh}. In all cases an exponential DTD with timescale $\tau=1.5$ Gyr is assumed.}
    \label{fig:ofe-df-sfh}
    \script{ofe_df_sfh.py}
\end{figure*}

\subsection{The [O/Fe]--[Fe/H] Plane}
\label{sec:ofe-feh}

\begin{figure}
    \centering
    \includegraphics{figures/ofe_feh_sfh.pdf}
    \caption{A comparison of the [O/Fe]--[Fe/H] plane between the four SFH models in our multi-zone simulations. All assume an exponential DTD with $\tau=1.5$ Gyr. Each point plots the abundances of a single star particle and is color-coded according to its Galactocentric radius at birth. Each panel plots a random sample of 10,000 star particles in the Solar neighborhood ($7\leq R_{\rm gal}<9$ kpc, $0\leq|z|<0.5$ kpc) weighted by their mass. A Gaussian scatter has been applied to all points based on the median abundance errors in APOGEE DR17. The black curves represent the ISM abundance tracks in the 8 kpc zone. The red contours represent the APOGEE abundance distribution, with the solid and dashed contours enclosing 68\% and 95\% of stars, respectively.}
    \label{fig:ofe-feh-sfh}
    \script{ofe-feh-sfh.py}
\end{figure}

\begin{figure*}
    \centering
    \includegraphics{figures/ofe_feh_dtd.pdf}
    \caption{Similar to Figure \ref{fig:ofe-feh-sfh} but comparing multi-zone simulations with different DTD models. All assume an inside-out SFH. Each row contains star particles from a different bin in $|z|$, with stars closest to the midplane in the bottom row and stars farthest from the midplane in the top row.}
    \label{fig:ofe-feh-dtd}
    \script{ofe-feh-dtd.py}
\end{figure*}

\subsection{The Age--[O/Fe] Plane}
\label{sec:age-ofe}

\begin{figure}
    \centering
    \includegraphics{figures/age_ofe_sfh.pdf}
    \caption{Caption}
    \label{fig:age-ofe-sfh}
    \script{age_ofe_sfh.py}
\end{figure}


\section{Discussion}
\label{sec:discussion}

Picking the team.

% TODO make table with numerical scores (KL divergences, etc) for supplementary material
% TODO explain that precise statistical understanding of these scores is outside the scope of the paper

\variable{output/summary_table.tex}

\subsection{Radial Migration \& Bimodality}

Something about \citet{Johnson2021-Migration} and \citet{Schonrich2009-RadialMixing}.

\section{Conclusions}
\label{sec:conclusions}

\begin{itemize}

    \item Our diagnostics consistently favor a DTD with a large fraction of delayed SNe Ia, such as the 3 Gyr exponential, 1 Gyr plateau, or triple system evolution models, over a DTD with a high number of prompt SNe Ia, such as the $t^{-1.4}$ power law or prompt models. [Discuss the disconnect and/or complement with \citet{Maoz2017-CosmicDTD}, etc.; something about different timescales probed by volumetric surveys vs. us]
    
    \item While the choice of DTD cannot produce a bimodal distribution of [O/Fe], it can affect the shape of the distribution, the ratio of high-$\alpha$ to low-$\alpha$ stars, and the location of the high-$\alpha$ sequence if it exists. The SFH is the critical determining factor in bimodality.

    \item The distribution of [Fe/H] depends strongly on the chosen SFH, but there is only a weak effect from the DTD.

    \item The early-burst SFH model is the only one which produced a bimodal distribution of [O/Fe] across the disk resembling APOGEE, from the models we investigate. [Add: summary of why everything else fails.]

    \item Something about how radial migration can't produce bimodality on its own.
    
\end{itemize}

\begin{acknowledgments}

Funding for the Sloan Digital Sky 
Survey IV has been provided by the 
Alfred P. Sloan Foundation, the U.S. 
Department of Energy Office of 
Science, and the Participating 
Institutions. 

SDSS-IV acknowledges support and 
resources from the Center for High 
Performance Computing  at the 
University of Utah. The SDSS 
website is www.sdss4.org.

SDSS-IV is managed by the 
Astrophysical Research Consortium 
for the Participating Institutions 
of the SDSS Collaboration including 
the Brazilian Participation Group, 
the Carnegie Institution for Science, 
Carnegie Mellon University, Center for 
Astrophysics | Harvard \& 
Smithsonian, the Chilean Participation 
Group, the French Participation Group, 
Instituto de Astrof\'isica de 
Canarias, The Johns Hopkins 
University, Kavli Institute for the 
Physics and Mathematics of the 
Universe (IPMU) / University of 
Tokyo, the Korean Participation Group, 
Lawrence Berkeley National Laboratory, 
Leibniz Institut f\"ur Astrophysik 
Potsdam (AIP),  Max-Planck-Institut 
f\"ur Astronomie (MPIA Heidelberg), 
Max-Planck-Institut f\"ur 
Astrophysik (MPA Garching), 
Max-Planck-Institut f\"ur 
Extraterrestrische Physik (MPE), 
National Astronomical Observatories of 
China, New Mexico State University, 
New York University, University of 
Notre Dame, Observat\'ario 
Nacional / MCTI, The Ohio State 
University, Pennsylvania State 
University, Shanghai 
Astronomical Observatory, United 
Kingdom Participation Group, 
Universidad Nacional Aut\'onoma 
de M\'exico, University of Arizona, 
University of Colorado Boulder, 
University of Oxford, University of 
Portsmouth, University of Utah, 
University of Virginia, University 
of Washington, University of 
Wisconsin, Vanderbilt University, 
and Yale University.

\end{acknowledgments}

\appendix

\section{Reproducibility}
\label{app:reproducibility}

This study was carried out using the reproducibility software
\href{https://github.com/showyourwork/showyourwork}{\showyourwork}
\citep{Luger2021-showyourwork}, which leverages continuous integration to
programmatically download the data from
\href{https://zenodo.org/}{zenodo.org}, create the figures, and
compile the manuscript. Each figure caption contains two links: one
to the dataset stored on zenodo used in the corresponding figure,
and the other to the script used to make the figure (at the commit
corresponding to the current build of the manuscript). The git
repository associated to this study is publicly available at
\url{\GitHubURL}, and the release v.X.X allows anyone to re-build the entire 
manuscript. The datasets are stored at [URL].

\bibliography{bib}

\end{document}
